\part{Miscellanea Notes}

%%%%%%%%%%%%%%%%%%%%%%%%%%%%%%%%%%%%%%%%%%%%%%%%%%%%%%%%%%%%%%%%%%%%%%%%%%%%%%%%
% bldoc introduction
%%%%%%%%%%%%%%%%%%%%%%%%%%%%%%%%%%%%%%%%%%%%%%%%%%%%%%%%%%%%%%%%%%%%%%%%%%%%%%%%
\chapter{Introduction}
Bldoc主要是学习工作中的随记,大多来自HOWTO、man手册、各种软件自带的帮助文档,技术书籍,
以及网上看到的CS相关的技术文章或逸闻趣事,另外还有极少的个人感悟。
不论程序设计还是文档写作,一个诀窍是Copy and Plaster。
然而,俗话说吃亏就是占便宜,反过来讲,便宜不好占;因此本文绝大多数的内容,虽非原创,但不复制粘贴。
这样就可以避免使文档内容呈爆炸式的增长,也可以在键入每个字的时候,加深印象。
好记性不如烂笔头,书不厌读,识不厌记。

文档的内容主要是:
\begin{itemize}
\item CS随记,包括CS逸闻轶事逸言,CS读书笔记等。
\item C/C++程序设计,包括语言、库、调试技巧等。
\item UNIX程序设计,包括编译系统、系统编程、网络编程等。
\item UNIX环境,包括Bash、系统命令等。
\item UNIX网络,如iptable/netstate/tcpdump等各种网络工具的使用。
\item 网络协议,如TCP/IP/HTTP/FTP/BT等。
\item 数据结构和算法。
\item 硬件相关的笔记。
\item Python程序设计。
\item Vim编辑器。
\item 使用\LaTeX或者DocBook排版。
\end{itemize}

整个文档是在学习\LaTeX{}的过程中完成的。也许永不完成。因为只要还在这个行业,就不可能停止记录。
文档的质量良莠不齐,但我也并不追求完美,这份文档的初衷只是个人的一个学习笔记。

%%%%%%%%%%%%%%%%%%%%%%%%%%%%%%%%%%%%%%%%%%%%%%%%%%%%%%%%%%%%%%%%%%%%%%%%%%%%%%%%
% reading notes
%%%%%%%%%%%%%%%%%%%%%%%%%%%%%%%%%%%%%%%%%%%%%%%%%%%%%%%%%%%%%%%%%%%%%%%%%%%%%%%%
\chapter{Reading Notes}
\section{Master}
\blcomment{本节记录计算机领域的传奇式人物,他们的事迹,言论等。}

%%%%%%%%%%%%%%%%%%%%%%%%%%%%%%%%%%%%%%%%%%%%%%%%%%%%%%%%%%%%%%%%%%%%%%%%%%%%%%%%
% doouglas mcilroy
%%%%%%%%%%%%%%%%%%%%%%%%%%%%%%%%%%%%%%%%%%%%%%%%%%%%%%%%%%%%%%%%%%%%%%%%%%%%%%%%
\subsection{Douglas McIlroy}
\href{http://www.cs.dartmouth.edu/~doug/}{Doug McIlroy's Homepage}。
\href{http://www.cs.dartmouth.edu/~doug/biography}{Doug McIlroy's Biography}。
McIlroy发明了管道。是echo, spell, diff, sort, join, graph, speak, tr的作者。

Those types are note ``abstract'', they are as real as int and float.

但是,Linux给出``因为没安装对应的软件,所以打不开文件''这种Mac式诊断之时,
就是Linux不再是UNIX之日。

[McIlroy78]中揭露的UNIX哲学:
\begin{itemize}
\item 让每个程序就做好一件事。如果有新任务,就重新开始,不要往原程序中加入新功能而搞得复杂。
\item 假定每个程序的输出都会成为另一个程序的输入,哪怕那个程序还是未知的。
输出中不要有无关的信息干扰。避免使用严格的分栏格式和二进制输入。不要坚持使用交互式输入。
\item 尽可能早地将设计和编译的软件投入试用,哪怕是操作系统也不例外,理想情况下,
应该是在几星期内。对拙劣的代码别犹豫,扔掉重写。
\item 优先使用工具而不是拙劣的帮助来减轻编程任务的负担。工欲善其事,必先利其器。
\end{itemize}

\href{http://www.cs.dartmouth.edu/~sinclair/doug/?doug=mcilroy}{这里}有许多关于McIlroy
的笑话,摘录数条于下:
\begin{itemize}
\item Doug McIlroy doesn't make system calls. System calls call Doug McIlroy.
\item Doug McIlroy doesn't use malloc to allocate memory. He uses his bare hands.
\item Doug McIlroy doesn't debug. He stares at tty0 until it fixes the problem.
\item Alan Turing always wanted to win a McIlroy Award, but didn't qualify. No one has.
\item In 1984, the Department of Justice broke up AT\&T because they had a monopoly(垄断). On Doug McIlroy.
\item Doug McIlroy supervises the hypervisor.
\end{itemize}

%%%%%%%%%%%%%%%%%%%%%%%%%%%%%%%%%%%%%%%%%%%%%%%%%%%%%%%%%%%%%%%%%%%%%%%%%%%%%%%%
% c++ d&e
%%%%%%%%%%%%%%%%%%%%%%%%%%%%%%%%%%%%%%%%%%%%%%%%%%%%%%%%%%%%%%%%%%%%%%%%%%%%%%%%
\section{C++ D\&E}
\blcomment{The Design and Evolution of C++,昵称C++ D\&E,中文名《C++语言的设计和演化》,
是C++语言的创始人Bjarne Stroustrup的著作。}

尊重人群而不尊重人群中的个体实际上就是什么也不尊重。
C++的许多设计决策根源于我对强迫人按某种特定方式行事的极度厌恶。
在历史上,一些最坏的灾难就起因于理想主义者们试图强迫人们``做某些对他们最好的事情”。
这种理想主义不仅导致了对无辜受害者的伤害,也迷惑和腐化了施展权利的理想主义者们。

我不认为自己有权利把个人观点强加给别人。不同的人们确实会按不同的方式思考,
喜欢不同的方式做事情,对于这些情况的高度容忍和接受是我最愿意做的事情。

经常的情况是,如果一个人可以很容易地转变到``信仰”X,那么进一步转变到``信仰”Y也是很有可能的。
我喜欢怀疑论者而不是``真实的信徒”。我把一点点实在的证据看得比许多理论更有价值,
把实际经验结果看得比许多逻辑论述更重要。

我绝不想通过一种有局限性的程序设计语言定义去推行某种唯一的设计理念。

一个程序设计语言只是这个世界中微乎其微的一个部分,因此也不应该把它看得太重了。
要保持一种平衡的心态,特别重要的是应该维持自己的幽默感。

C++语言在1985年之后的演变,就说明了来自Ada(模板、异常、名字空间),Clu(异常),
以及ML(异常)的思想的影响。

如果地图与地表不符,要相信地表。 -瑞士军队格言

我认为原理这个词在一个真正科学原理非常贫乏的领域中显得过于自命不凡了,
而程序设计语言设计就是这样的一个领域。

你可以在任何语言里写出很坏的程序。

委员会的每个成员都需要学会尊重那些看起来是异己的观点,学会妥协。
这些倒是很符合C++的精神。

Samlltalk鼓励人们把继承看成是唯一的,或者至少是最基本的程序组织方式,
并鼓励人们把类组织到只有一个根的层次结构中。在C++里,类就是类型,并不是组织程序的唯一方式。
我也极端怀疑一种论断,说是需要强迫人们去采用面向对象的风格写程序。
除非你能把握住如何表现隐藏在数据抽象和面向对象的程序设计后面的原理,
否则你能做的不过是错误的使用支持这些概念的语言特征。

有多少天赋,也打不败细节的纠缠。 -古语

魔鬼隐藏在细节之中。    -古语

\section{The long tail}
作者:Chris Anderson

广播电视有一个了不起的地方:它可以用无可匹敌的效率将一个节目送到数百万人面前。
但是,相反的事情它却做不到--将数百万节目送到同一个人面前。而这一点正是互联网的强项。

21世纪经济学的秘密就藏在企业的服务器中。

19世纪意大利经济学家帕累托提出的80/20原则。

在传统零售经济学已经举步维艰的地方,网络零售经济学仍然能够高歌猛进。

最大的财富孕育自最小的销售。

这些“货架空间无穷无”的企业已经领悟了数学集合论的一个原理:
一个极大极大的数(长尾中的产品)乘以一个相对较小的数(每一种长尾产品的销量),仍然等于一个极大极大的数。而且,这个极大的数只会变的越来越大。

电磁波有一种无与伦比的威力:它可以毫无成本的向各个方向传播。

“饮水机效应”指的是办公室里围绕某个大众文化事件的热烈讨论。

下水道的最高排放量通常是在Super Bowl的中场休息时测量到的。

把魅力四射的年轻男人卖给年轻的女人。

一些根本性的东西已经在2000年改变了。

每一个热门都拥有数量虽少但却更加执着的拥?

年轻人不会等待某个神圣的数据来告诉他们什么东西是最重要的,他们想控制他们的媒体而不是被媒体控制。

他们给一件不可预见的事带来了一点点可预见性。

货架空间的分配就是一个零和游戏:一种产品取代另一种产品。

我们正在从一个大规模市场退回到利基市场,只不过,定义不同市场的不再是地理位置,而是我们的兴趣爱好。

在经济学中,搜索成本是指任何妨碍你寻找目标的东西。

其他消费者的行动往往是最有用的信号,因为他们的动机与我们最为统一。

现在,专业-业余写作也可以成就伟业。

面对茫茫太空,你唯有在绝对正确的时间去看绝对正确的方位才能观察到那些最有趣的新现象,
比如小行星或星体演化。这不是一台望远镜有多大,多贵的问题,而是在某一个特定时刻能有多少双眼睛盯着太空的问题。

如果有足够多的眼睛,所有bug都不在话下。

复印机率先揭穿了“媒体的利用永远属于拥有媒体的人”这句谎言。

业余者(amateur)这个词本身就来自拉丁语的爱人(amator)一词,动词是去爱(amare)。

正在成长的一代人目睹他们的同龄人制作出这样动人的创造性杰作,必然会被深深地触动。
某些才华横溢的人和某些了不起的设备共同造就了这些令我们神魂颠倒的艺术经典。
但是,一旦你了解了幕后的玄机,你就会意识到你也能成为这样的“天才”。

Jimmy Wales,最富有的期权交易商。维基百科全书没有标榜权威二字。
维基的真正非凡之处在于有机的治疗自己,自然选择那些必要的特征以避开本生态系统内的食肉动物和病原体的侵袭。

长尾中的事情有许多都不是以商业利益为目标的。在需求的头部和尾部,创造的动机截然不同。需求曲线开始于头部的传统货币经济,终结于尾部的非货币经济。

lulu.com - 一个新型的DIY出版商。

因为说完整个词太费时间,等三个字说完,你也就不再年轻了。

朋克摇滚的精神就是:没错,你有你的吉他,但你不一定要做正确的事!你可以做错!你是一个好音乐家一点意义也没有,唯一最要的是-你有话要说。

%%%%%%%%%%%%%%%%%%%%%%%%%%%%%%%%%%%%%%%%%%%%%%%%%%%%%%%%%%%%%%%%%%%%%%%%%%%%%%%%
% c++ coding standards
%%%%%%%%%%%%%%%%%%%%%%%%%%%%%%%%%%%%%%%%%%%%%%%%%%%%%%%%%%%%%%%%%%%%%%%%%%%%%%%%
\section{C++ Coding Standards}
C++ Coding Standards, 101 Rules, Guidelines, and Best Practices.\\
作者:Herb Sutter, Andrei Alexandrescu。\\
译者:刘基诚


小类只体现一个概念,承担一个责任。巨类,削弱封装性。

编译时隔离。

除非需要继承,否则不要忍受其弊端。

有些人不想生孩子。勿将独立类用作基类。

利器在手,勿再徒手为之。

策略应该上推,而实现应该下放。

健壮的设计就是能将修改限于局部的设计。新的需求不应该引起对已有代码的重新改写。
设计如果含有混合了实现细节的接口,就很可能会出现复杂的依赖网。

passthrough function, 通道函数,常见如get/set函数。
instrumentation: 度量性。

DIP: Dependency Inversion Priciple.

Law of Second Chances: 需要保证正确的最重要的东西是接口,其他所有东西以后都可以修改,
如果接口弄错了,可能再也不允许修改了。

Liskov: Substitution Priciple. 公用继承的目的是实现可替换性。

虽然派生类通常会增加更多的状态(即数据成员),但它们所建模的是其基类的子集而非超集。
派生类是更一般的基础概念的一个特例。

一个函数无法很好的履行两种职责。公用虚拟函数具有两个职责:指定了接口,指定了实现细节。
NVI: Nonvirtual Interface。

并非所有变化都是进步。

隐藏数据却又暴露句柄的做法是一种自欺欺人,就像你锁上了自己家的门,却把钥匙留在了锁里。

C++将私有成员指定为不可访问的,但没有指定为不可见的。
可访问性:是否能够被调用或者使用某种东西。可见性:是否能看到他从而依赖他的定义。
类的私有成员在成员函数和友元之外是不可访问的,但对整个世界,即所有看到类定义的代码而言,
都是可见的。使用Pimpl。但是,只有在弄清了增加间接层次确实有好处之后,才能添加复杂性,Pimpl也是一样。

\section{Effective C++ 3/e}
作者:Scott Meyers \\
译者:侯捷

一位女性若非怀孕,就是没有怀孕。不可能说她部分怀孕。同样道理,一个软件系统要不就具备异常安全性,
要不就全然否决,没有所谓的局部异常安全系统。

pimpl idiom. copy-and-swap.

RCSP, Reference-Counting Smart Pointer, 但RCSP无法打破循环引用。\ shared\_ptr 就是RCSP。

shared\_ptr有一个特别好的性质:它会自动使用它的每个指针专属的deleter,因而消除另一个潜在的客户错误,
即所谓的cross-DLL problem。问题发生于对象在某一DLL中被new创建,却在另一个DLL内被delete。
许多平台上,这类跨DLL之new/delete成对应用会导致运行期错误。




%%%%%%%%%%%%%%%%%%%%%%%%%%%%%%%%%%%%%%%%%%%%%%%%%%%%%%%%%%%%%%%%%%%%%%%%%%%%%%%%
% working notes
%%%%%%%%%%%%%%%%%%%%%%%%%%%%%%%%%%%%%%%%%%%%%%%%%%%%%%%%%%%%%%%%%%%%%%%%%%%%%%%%
\chapter{Working Notes}
\section{Working In Funshion}
\bldate{2009/07/09} 阅读HS源码时看到的hash算法,有来自UNIX system V的ELFhash,
来自阎宏飞、谢正茂在天网搜索引擎中使用的hash算法,另有一种是谢正茂的hash算法。
下载了个一致性hash(consistent hash)的实现,据说是memcache client的一个实现,
由ketama实现。

\bldate{2009/07/08} BS地址service-bs.funshion.com

\bldate{2009/07/08} fget公网运行的一个问题。fget单独运行正常,被MS调用则出错。
因为MS fork()/exec()之前,已有大量打开的文件描述(数千个),fget继承父进程打开的文件描述符,
因此socket()创建的sockfd的值>1024,用非阻塞connect()后调用select()判断sockfd是否可写,
而select()系统限制为1024,因此select这里即出错。解决方案是在fget启动时,关闭文件描述符。

\bldate{2009/06/19} web查看homeserver信息:admin.funshion.com

\bldate{2009/06/19} 测试了socket选项SO\_REUSEADDR。在Fedora9上测试,在一个进程内部,
不能对同一个端口有两次bind()调用,也不能有两个进程同时绑定到同一个端口。
在CentOS 5.2上测试,一个进程内部,两个socket可以bind到同一个端口,
但两个进程不能同时bind到同一个端口。
man 7 socket上关于SO\_REUSEADDR的说明中有提及不能bind到同一个端口。
在Windows上,一个进程内部可以多次bind同一端口,也可两个进程bind同一端口,只是结果是未知的。

\bldate{2009/06/17} 代码审查时发现的问题。send失败,可休眠1秒再次尝试。
accept如果需要限制被动连接数,可先accept,再close。

\bldate{2009/06/17} 调试发现的问题。对本的机器做了对等端口映射(192.168.16.252:9501映射公网端口9501)。
fget接受被动连接。发现有被动连接进来的peer发送bitfield的长度和本地任务不匹配。
但只有被动连接的peer有此问题,主动连接则无。缘故是:在tracker上announce后,
客户端可能已存储fget信息,当fget切换另一任务下载时,客户端可能连接fget并交互旧任务的bitfield。
解决方法是在handshake时,判断任务hash,如果不是当前任务hash,则中断连接。

\bldate{2009/06/15} 调试是发现,两个函数调用之间参数被改变,如foo(par) \{ boo(par); \},
函数声明foo, boo的参数都为unsigned int,且par的类型也为unsigned int,但调试时仍发现,
在foo(par)时par值为VAL1(10485760),在boo(par)时par值为VAL2(4096)。但重新编译后,再次运行,无此问题。
非常诡异。

\bldate{2009/06/03} 此前以往八月,皆属无稽之谈。

\bldate{2009/05/31} 连接公网peer,连接速度很慢,不容易连接上。因此如果连接成功,则不要轻易断开。
另一则是利用被动连接。

\bldate{2009/04/27} 最近希望看一些高端服务器设计方面的论文或者书籍。不过资料似乎不多。
从C10K等看起,但貌似资料有些旧了,都是好些年前。看TAoUP,上面对线程编程评价似乎很低,
在High-Performance-Server-Design上也看到关于线程的代价评论,在另一篇Why-Threads-Are-Bad-Idea里也有
把线程与event编程模型的对比。继续深入学习。

\bldate{2009/04/23} fget项目中做了一个修改。将utility.h修改为fget\_utility.h,然后单独建立一个utility目录,
其中又有一个utility.h文件,并在fget\_utility.h中包含utility.h。经过一些简单的试验,
如gcc -E a.cpp输出没有包含utility.h中声明的函数,在gcc -E b.cpp的输出却包含utlity.h中声明的函数。
比较奇怪的问题。查明结果是在把原始的utility.h改为fget\_utility.h时,忘了修改头文件的宏定义,即\#ifndef \_UTILITY\_H。
再次说明一个问题,切勿冒冒然修改程序,想清楚前后关联影响,再修改。

\bldate{2009/04/23} 当机器有多个网口时,一般eth0配置为外网地址,eth1配置为内网地址。当bind时,
如果不指定地址,用INADDR\_ANY,则由内核选择一个网口。如果bind到指定网口,可能路由时不能抵达目标地址(未验证)。

\bldate{2009/04/07} select()返回-1,表示有连接坏掉了,此时也可能受到了SIGPIEPE信号。此时sockfd $>$ 0还成立,
因此要把这个坏掉的连接抓出来,有几种方法。1,用select依次调用每个sockfd,如果返回-1,则此连接已坏。
2,调用一些socket函数,如果返回错误,则此连接坏掉。注意如果调用send(), recv()等函数,可能另有问题,
如,sockfd是非阻塞的,但连接已坏,此时可能被send, recv阻塞。这种情况下,可以尝试调用getsockopt等函数判断。
另外,select()返回-1,如果errno为EINTR,则此错误可以忽略。

\bldate{2009/03/31} 测试fget时,如果连接公网Tracker,则fget也需要有公网ip和port,否则fget之间不能互联。
对于非阻塞socket的connect()调用,在返回EINPROGRESS后,select()的等待时间可为4秒。

\bldate{2009/03/26} 今天得知运营是用PRTG的工具监视流量的。当然是UI是web页面。

\bldate{2009/03/12} 系统调用的错误处理。在send/write后,应该仔细检查返回值是否等于发送缓冲的长度。
如果缓冲区已满,则应保留未发送的数据,等待下一次send。

\bldate{2009/03/10} 公网调试MediaServer时发现的问题。kill了mediaserver,但很快它自己又运行了。
注意检查deamon。如果有deamon,先killdeamon。另外,用valgrind检查MediaServer内存时,发现没有MediaServer
吐不出流量,很奇怪。

\bldate{2009/03/10} 检查函数返回值时的低级错误。在网络编程时,如果是非阻塞socket,
则send/recv返回-1,且errno是EWOULDBLOCK(或者EAGAIN),则应该忽略这个错误。
但注意与EWOULDBLOCK比较的对象是errno,而非send/recv函数的返回值(-1)。

\bldate{2009/02/26} 网络速度的快慢。但发现自己写的网络程序速度慢,要主要检查两个方面,
一是send速度,一是request速度。如BT程序,如果Client请求发的频率很低(请求慢),
则Server发送的数据自然就少,反应到Monitor上,就表象为速度慢。
而且网络程序,如果发现问题,需要检查连接两端,先确保一端没有问题,再调试另一端。
所以发现“速度慢”之类的问题,不要忽略了请求速度的控制。

\bldate{2009/02/26} TCP发送窗口的问题。调试时发现一条TCP连接速度较快,
但用抓包器观察,连接双方的TCP窗口却很小,特别是TCP三次握手期间,第1和第2次握手交换的窗口还较为合理,
但第3次握手的窗口就非常低。这个问题纠缠了数天,后来查明原因是:
TCP最初设计窗口大小为16bit的值,即最大为65535字节;后来对TCP协议的扩展中,增加了一个
window scale,是16bit,可用其中的[0,14],wscale在SYN报文(SYN报文在3次握手的前两次中出现,
第3次是ACK报文)中声明,之后发的窗口,都要经过这个wscale的计算。如wsacle为7,window为56,
则实际的window大小为:$56 * 2^{7}$,即7168。

\bldate{2009/02/24} 调试程序最怕遇见的时随机出现的问题。memcpy时常导致core dump,
但查不出究竟。在memcpy前做了非常详细的参数检查,许多的assert。后来发现这样做其实有些无济于事,
因为assert在这里不能找出问题根源(或者应该用异常?)。数小时的努力后,
放弃调试,开始源码阅读(write by myself -\_-!!),数分钟后发现了问题所在。
精简的代码逻辑如下:
\noindent\begin{lstlisting}[language=C++]
pointer* p = arr_have_long_name.find();
for(int i = 0; i < arr_have_long_name.size(); ++i){
    // do some check, but unfortunate, p is changed
    p = arr_have_long_name[i];
    check(p);
}
return p;
\end{lstlisting}
教训是,最经济的排错方法是仔细阅读源码,而非利用各种工具或者不适用的技巧。

\bldate{2009/02/10} 用GDB调试的一个问题。编译时加-g参数,编译生成a.out。
在serv-a上编译,在serv-b上调试。在GDB中用edit查看源文件,发现源文件不同;
用系统命令md5sum计算serv-a/a.out和serv-b/a.out,发现两个的MD5值却完全相同。
导致此问题的原因是:-g参数并不将源代码嵌入可执行程序,而只是在可执行程序中
嵌入调试信息,与源代码关联起来。因此在serv-b上看到程序源码,完全出于巧合,
即假设在serv-a的/home/bailing/编译./hello.cpp,而在serv-b的/home/bailing/目录下,
恰好存在一个名为hello.cpp的文件,因此被GDB发现;如果没有hello.cpp,
则GDB会抱怨说找不到源码。

%%%%%%%%%%%%%%%%%%%%%%%%%%%%%%%%%%%%%%%%%%%%%%%%%%%%%%%%%%%%%%%%%%%%%%%%%%%%%%%%
% other notes
%%%%%%%%%%%%%%%%%%%%%%%%%%%%%%%%%%%%%%%%%%%%%%%%%%%%%%%%%%%%%%%%%%%%%%%%%%%%%%%%
\chapter{Other Notes}
\noindent\href{http://lamp.linux.gov.cn/jinbuguo\_florilegium.html}{金步国的主页} 
严谨的翻译和原创作品,实用。\\
\noindent\href{http://www.ringkee.com/}{肥肥世家}
丰富的学习笔记。\\
\noindent\href{http://docs.huihoo.com/homepage/shredderyin/}{王垠的个人主页}
\LaTeX{} 等介绍。\\
\noindent\href{blog.csdn.net/haoel}{陈皓} CSDN博客。Makefile、GDB调试等文章。\\
\noindent\href{http://blog.youxu.info/}{4G Spaces} 计算机八卦等。\\
\noindent\href{http://blog.csdn.net/g9yuayon/}{袁泳 \textbar{} 负暄琐话} \\
\noindent\href{http://blog.csdn.net/pongba/}{刘未鹏 \textbar{} C++的罗浮宫} \\
\noindent\href{http://mindhacks.cn/}{刘未鹏 \textbar{} Mind Hacks} \\

\bldate{2009/02/24} 搜索Linux下的网络监测工具很是费劲,最后发现的一个较好的搜索词为:
Linux network traffic monitor tool。

listings\footnote{\LaTeX{} 的一个宏包,用它排版计算机程序。}的手册介绍了一个让\TeX{} crash的技巧。
在\LaTeX{} 中引入以下源码:
\begin{lstlisting}
\lstdefinestyle{crash}{style=crash}
\lstset{style=crash}
\end{lstlisting}
然后手册的作者说:Only bad boys use such recursive calls, 
but only good girls use this package.

逸闻趣事,从《计算机网络4/e》8.3.2节上看到的:
第一个公开密钥算法是背包算法(MerKle and Hellman,1978),
算法的发明者Ralph Merkle对自己的算法非常自信,因此他悬赏100美金给破解算法的人。
Adi Shamir(即RSA中的S)迅速的破解了算法,领取了奖金。
但Merkle并不气馁,又加强算法,并悬赏1000美金给破解算法的人。
这次Ronald Rivest(即RSA中的R)也迅速地破解了该算法,并领取了奖金。
Merkle不敢再为下一个版本悬赏10,000美金了,所以“A”(Leonard Adleman)很是不幸,无法领取奖金了。
背包算法不再被认为是安全的,也没在实践中使用。

\href{http://developers.solidot.org/article.pl?sid=09/03/24/0859257}{这里}看到的关于Vi和Emacs的趣闻。
说是海盗用Emacs,忍者用Vi。总所周知Emacs拥有强大无比的定制和扩展能力,而海盗也无时不在定制化他们的趁手工具,
Emacs确实比Vi慢,但海盗并不在意,因为他们经常喝的醉醺醺的。
英文原文在\href{http://philosecurity.org/2009/03/23/pirates-and-ninjas-emacs-or-vi}{这里}。

John W. Backus: You need the willingness to fail all the time\ldots{} You have to generate
many ideas and then you have to work very hard only to discover that they don't work.
And you keep doing that over and over until you find one that does work.

Edmund Burke: All that is needed for the triumph of misguided cause is that good people do nothing.
[谬误想要获得胜利,只需好人袖手旁观。]

BBN had a big contract to implement TCP/IP, but their stuff didn't work, and Joy's grad student
stuff worked. So they had this big meeting and this grad student in a T-shirt shows up, and they said,
"How did you do this?" And Bill said, ``It's very simple, you read the protocol and write the code."

Don says that he chose the term in hopes of making the originators of the term ``structured programming"
feel as guilty when they write illiterate programs as he is made to feel when he writes unstructured programs.

Who do you think is the best coder, and why?

Leonardo Da Vinci是``从Vinci来的Leonardo"之意。

John Carmack, Castle Wolfstein, doom, doomII, Quake的作者。其简历说自己的专长是``Exhaust 3-D technology"。

C.A.R.Hoare: Premature optiomization is the root of all evil.[提前优化是万恶之源。]

只有向后看才能理解生活,但是要生活好,则必须向前看。 -SREN AABYE KIERKEGAARD 日记(1843)

%%%%%%%%%%%%%%%%%%%%%%%%%%%%%%%%%%%%%%%%%%%%%%%%%%%%%%%%%%%%%%%%%%%%%%%%%%%%%%%%
% temp notes
%%%%%%%%%%%%%%%%%%%%%%%%%%%%%%%%%%%%%%%%%%%%%%%%%%%%%%%%%%%%%%%%%%%%%%%%%%%%%%%%
\chapter{Temp Notes}
\blcomment{
本章为网络浏览时随手之笔记,大多为外文资料随译之笔,并未修饰言辞。
且随意放置,未加管束。其中所记,多未加实践实验,权当开阔眼界。
待时日既久,材料既殷,再分门别类。}

Linux内核源码在如\blscmd{/usr/src/kernels/2.6.18-92.el5-x86\_64}的目录,但似乎其中没有源文件。

DCCP, Datagram Congestion Control Protocol,数据报拥塞控制协议。
\href{http://www.linuxfoundation.org/en/Net:DCCP}{这里}有比较全的介绍。
dccp在Linux内核的\blscmd{net/dccp}这个目录。DCCP是一个传输层协议,
在RFC 4340-4342中有其说明。

CDN, Content Delivery(or Distribution) Network,
有两种类型的content,下载和流。下载类的如页面访问和视频。流如在线视频。
应该是服务器均衡负载的一种实现网络。wiki上介绍说,策略性的放置服务器,
可以获取吞吐量大于backbone主干线的能力。CND减少了洲际(interconnects)的,
public peers, private peers和bacbones的负担,降低了delivery的开销。
避免了负载于backbone或peer link的压力,而将之重定向且分摊于edge servers。

TCP因丢包和时延而受损,而CDN置服务器于edge networks,并使用户易于访问。
某种CDN,如高速web页面的cache。当靠近server的client访问web时,此server查询cache
中是否存在此页面,若有直接返回用户;若无,则从其他server获取,并cache之。

CRT(C Runtime Library)。MSDN上看到的,VC编译时的C运行时库,有多种选项,其说明如下:\\
\begin{tabular}{|l|l|l|l|}\hline
编译选项    & 关联的DLL     & C运行时库         & 说明  \\\hline
/MT         & 无。          & libcmt.lib        & 多线程,静态链接 \\\hline
/MD         & msvcr80.dll   & msvcrt.lib        & 多线程,动态链接 \\\hline
/MTd        & 无。          & libcmtd.lib       & 多线程,静态链接,调试版本 \\\hline
/MDd        & msvcr80d.dll  & msvcrtd.lib       & 多线程,动态链接,调试版本 \\\hline
\end{tabular}

之前有单线程的C运行时库(libc.lib, libcd.lib,用/ML, /MLd开启),但现在已不支持。
对于C++运行时库,把以上libc都变为libcp即是。编译选项相同。

wiki上看到说lihttpd可以承受每秒1000次的访问。在Lighttpd的主页上见其1.5版使用aio,
据称效率有极大的提高。

用\blscmd{F-Secure SSH Client Trial}在Windows和Linux间通过SSH传输文件。

用\blscmd{iftop}可以查看网络流量,比\blscmd{vnstat}更为细致的输出。不过似乎有些字体问题。

\blscmd{COW, Copy-On-Write}技术用在\blscmd{fork()}上,当父进程调用fork()系统调用后,
并不直接将父进程的页面复制到子进程,而是两者共享地址空间。只有当其中一方对页面执行写操作时,
才复制其副本给修改者。

\blscmd{vfork}创建新进程后,父进程挂起,指导子进程执行完成退出,或执行\blscmd{execve()}系列函数。

\blscmd{UPnP}(Universal Plug and Play),简单的理解可以是:自动端口映射。这样内网的端口,
即可在通过NAT时做自动端口映射,即对公网开放了这个端口。要使用UPnP,需要Modem,OS和软件的支持。
在Windows中开启UPnP需要以下设置:
\begin{itemize}
\item 控制面板,添加删除程序,添加/删除Windows组件,网络服务,UPnP用户界面。
\item Windows防火墙,例外选项中选中UPnP框架。以上两步,可以通过:网上邻居,
显示联网的UPnP设备的图标,一次完成。
\item 控制面板,管理工具,服务,启动:SSDP Discovery Service和Universal Plug and Play Device Host
两项服务。
\end{itemize}

\href{http://www.usenix.org/about/flame.html}{USENIX frame award}。

斯德哥尔摩综合症又称为人质情绪、人质综合症,是指犯罪的被害者对于犯罪者产生情感,
甚至反过来帮助犯罪者的一种情绪。

浏览DHT时看到的外部链接,一个名为memcached的系统,用在加速需要与数据库交互动态页面访问场合,
被youtube, twitter等网站使用。\href{http://www.danga.com/memcached/}{这里}。

\href{http://www.xmlbar.com/}{稞麦网}提供下载视频网站的工具。如下载优酷视频。

\href{http://www.devtopics.com/101-great-computer-programming-quotes/}{101条计算机妙语}。
\href{http://news.csdn.net/a/20090522/211469.html}{中文版}。

ip转发,查看:
\bllcmd{cat /proc/sys/net/ipv4/ip\_forward}
也可修改\blscmd{/etc/sys}中的\blscmd{net.ipv4/ip\_forward}值。

\href{http://www.planet-lab.org/}{planet-lab}计算机网络实验。

安装\blscmd{wordpress},用作本地博客系统。依赖PHP, MySQL。
单独安装PHP, MySQL, Apache组件较为繁琐,可用\href{http://www.apachefriends.org/en/xampp.html}{xampp}软件。
此软件携带以上程序。安装xamp后,设定Apache监听端口,编辑\blscmd{/apahche/conf/httpd.conf},
如果默认80端口被占用,可改用8080等其他端口。再次启动Apache。用浏览器访问
\blscmd{http://localhost:8080/phpmyadmin},选择左侧之mysql,继而可查看user。默认有root用户,且密码为空。

MySQL的默认名为“localhost”,在wordpress的解压目录中,修改wp-config-sample.php,填入数据库信息。
其中已有注释提示,修改数据库名为wordpress,填入用户名(root),密码(),MySQL的名称(localhost)。
修改wp-config-sample.php为wp-config.php,并将整个wordpress目录放在\blscmd{/xampp/htdocs/}下。
或者如果单独安装apache,则放在\blscmd{/apache/htdocs}目录。
注意,要现在MySQL中建立wordpress数据库,不用见表等操作,但需要先建立此库。

\href{http://gigamonkeys.com/book/}{Practical Common Lisp}。



if browser were woman.


