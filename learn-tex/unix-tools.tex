\part{UNIX Tools}

UNIX Tools主要记录以下一些类别的UNIX工具:
\begin{itemize}
\item 常用工具。各种各样的常用工具,以及一些使用技巧。
\item 编译系统的主要工具。包括gcc, gdb, make。另有GNU出品的coreutil,包括strip等。
\item 编译系统的其他工具。如valgrind, gprof, objdump、nm等。
\item 网络工具。如traceroute, tcpdump, wget, curl,iptables等。
\item 其他工具。如perforce/p4源码管理客户端等。
\end{itemize}

UNIX变种的各种平台都大量使用GNU软件,\href{http://www.gnu.org/manual/manual.html}{这里}是GNU
软件的在线手册列表,包括gcc, gdb, libc, stdc++等。

%%%%%%%%%%%%%%%%%%%%%%%%%%%%%%%%%%%%%%%%%%%%%%%%%%%%%%%%%%%%%%%%%%%%%%%%%%%%%%%%
% UNIX environment tools
%%%%%%%%%%%%%%%%%%%%%%%%%%%%%%%%%%%%%%%%%%%%%%%%%%%%%%%%%%%%%%%%%%%%%%%%%%%%%%%%
\chapter{UNIX Enivronment Tools}

%%%%%%%%%%%%%%%%%%%%%%%%%%%%%%%%%%%%%%%%%%%%%%%%%%%%%%%%%%%%%%%%%%%%%%%%%%%%%%%%
% #environment basic
%%%%%%%%%%%%%%%%%%%%%%%%%%%%%%%%%%%%%%%%%%%%%%%%%%%%%%%%%%%%%%%%%%%%%%%%%%%%%%%%
\section{Basic}
\blscmd{adduser}添加用户,如:\blscmd{useradd bailing}。
\blscmd{passwd}为用户设置密码,如:\blscmd{passwd bailing}。
\blscmd{userdel}删除用户,如\blscmd{userdel bailing}。

修改\blscmd{/etc/sudoers}文件,为一般用户赋予需要root执行权限的命令,添加如下一行:
\bllcmd{bailing ALL=/usr/bin/p4}
这样一般用户bailing就具备了执行p4命令的权限,使用\blscmd{sudo p4}即可。
注意即使root用户也不能修改/etc/sudoers文件,使用命令\blscmd{visodu}即可。
如果要完全开放一个命令的执行权限,可以用\blscmd{chmod a+x}即可。

\blscmd{sudo}的操作会被日志记录。

\blscmd{wc}用于统计文件的行数,字符串个数等。
\blscmd{wc -l}表示输出行数。

\blscmd{sort}用于将文件每一行进行排序,默认按字典序排序。
\blscmd{sort -n}表示按数值排序。
\blscmd{wc f1 f2 f3 -l \textbar sort -n}按行数对f\{1-3\}文件排序。

\bllcmd{\$ netstat --tcp -n \textbar grep ``6699.*ES'' \textbar sort -r -n -k 3 \textbar less}
对netstat输出的第三列排序,从大到小。\blscmd{-k}指定目标排序域。

\blscmd{dd}命令可用于转储文件,man手册解释:convert and copy a file.
常用的切割文件的需求可用dd完成。如:
\bllcmd{dd if=fin of=fout skip=500 ibs=1 count=500}
以上\blscmd{ibs}指定了输入文件的切割单位是字节,\blscmd{count}指定了长度,以ibs为单位。

%%%%%%%%%%%%%%%%%%%%%%%%%%%%%%%%%%%%%%%%%%%%%%%%%%%%%%%%%%%%%%%%%%%%%%%%%%%%%%%%
% #environment tricks
%%%%%%%%%%%%%%%%%%%%%%%%%%%%%%%%%%%%%%%%%%%%%%%%%%%%%%%%%%%%%%%%%%%%%%%%%%%%%%%%
\section{Tricks}



使用Debian时,\blscmd{apt-get install ssh}后,发现用SecureCRT,以ssh连接debian,只要一段时间空闲,
就会被自动断开。因此需要设置:\blscmd{/etc/ssh/sshd\_config}文件,把\blscmd{LoginGraceTime}选项注释掉即可。

\blscmd{info core.util}查看系统所有的核心命令。

\blscmd{date}命令设置时间,可以用\blscmd{-d}查看字符串对应的时间是否正确,再用\blscmd{--set}设置时间。
\bllcmd{\$ date -d ``03/18/07 15:04''}, 输出\bllcmd{ 2009年 03月 18日 星期三 15:04:00 CST }

\blscmd{tee}read from standard input and write to standard output and files.
\blscmd{tcpdump}的手册中介绍的一个应用:
\bllcmd{tcpdump -l \textbar tee dat}
将\blscmd{tcpdump}输出行缓冲,并输出到stdout和dat文件。

%%%%%%%%%%%%%%%%%%%%%%%%%%%%%%%%%%%%%%%%%%%%%%%%%%%%%%%%%%%%%%%%%%%%%%%%%%%%%%%%
% #xargs
%%%%%%%%%%%%%%%%%%%%%%%%%%%%%%%%%%%%%%%%%%%%%%%%%%%%%%%%%%%%%%%%%%%%%%%%%%%%%%%%
\blscmd{xargs}build and execute command lines from standard input.
man xargs上的例子很值得学习。如:
\bllcmd{find /tmp -name core -type f -print \textbar xargs /bin/rm -f }
这个命令可能出现的错误是,如果文件名包含换号符(newlines)或空格(spaces),则会出错。因为\blscmd{xargs}以空格或换号作为分隔符。
\bllcmd{find /tmp -name core -type f -print0 \textbar xargs -0 /bin/rm -f }
这个命令则修复了上一个命令的错误。
\bllcmd{find /tmp -depth -name core -type f -delete}
与上一个命令效果相同,但效率更高,因为没有fork和exec的开销,也不用xargs的调用。
\bllcmd{cut -d: -f1 < /etc/passwd \textbar sort \textbar xargs echo}
产生一个紧凑的输出。如果没有xargs echo,则每个名字跟一个换行。

Bash的快捷操作:\\
\begin{tabular}{ll}
Ctrl-r  & 历史搜索模式。\\
Ctrl-a  & 将光标定位到命令开头。\\
Ctrl-e  & 将光标定位到命令结束。\\
Ctrl-u  & 剪切光标之前的内容。\\
Ctrl-k  & 剪切光标之后的内容。\\
Ctrl-y  & 粘贴由Ctrl-u/k剪切的内容。\\
Ctrl-t  & 交换光标之前两个字符的顺序,并将光标后移一位。\\
Ctrl-w  & 删除光标左边的一个word。\\
Ctrl-l  & 清屏。\\
\end{tabular}

在bash比较字符串的两个方法。用if语句:
\bllcmd{\$ if [ 1 = 3 ]; then echo ok; fi}
另一种方法是:
\bllcmd{\$ [ 1 = 3 ];ret=\$?;echo \$ret}

数学表达式:\blscmd{echo \$((1+3))}。

%%%%%%%%%%%%%%%%%%%%%%%%%%%%%%%%%%%%%%%%%%%%%%%%%%%%%%%%%%%%%%%%%%%%%%%%%%%%%%%%
% #ps, #top, #fuser, #lsof
%%%%%%%%%%%%%%%%%%%%%%%%%%%%%%%%%%%%%%%%%%%%%%%%%%%%%%%%%%%%%%%%%%%%%%%%%%%%%%%%
\section{Process Monitor}

%%%%%%%%%%%%%%%%%%%%%%%%%%%%%%%%%%%%%%%%%%%%%%%%%%%%%%%%%%%%%%%%%%%%%%%%%%%%%%%%
% #ps
%%%%%%%%%%%%%%%%%%%%%%%%%%%%%%%%%%%%%%%%%%%%%%%%%%%%%%%%%%%%%%%%%%%%%%%%%%%%%%%%
\subsection{ps}
\blscmd{ps -C fget -o pid=}输出进程fget的pid。可用在如top查看指定进程的运行时状态,
或用gdb挂载运行时进程:\blscmd{gdb fget `ps -C fget -o pid=`}。
以上命令更简单的模式是:\blscmd{gdb fget `pgrep fget`}。
如果有\blscmd{pidof}命令,则更为简单,\blscmd{gdb fget `pidof fget`}。

\blscmd{ps aux}输出进程的详细信息,其中进程状态一列的格式如下:\\
\begin{tabular}{ll} \hline
R   & Runing or runable(on run queue)\\
S   & Interruptible sleep(waiting for an event to complete) \\
Z   & defunct(zombie) process, terminated but not reaped by its parent \\\hline
<   & high-priority\\
N   & low-priority\\
L   & has pages locked into memory\\
s   & is a session leader\\
l   & is multi-threaded\\
+   & is in the foreground process group\\
\end{tabular}

查看mediaserver的进程状态输出是:\blscmd{SLl},则表明该进程在sleep,执行了锁页调用,
且是一个多线程进程。

\blscmd{-L} 显示线程信息,包括NLWP(number of thread)和LWP(Thread ID)。
\blscmd{-f} full-format listing。
\blscmd{-p PID} 输出PID进程的信息。
\blscmd{--no-headers} 不输出说明行。

%%%%%%%%%%%%%%%%%%%%%%%%%%%%%%%%%%%%%%%%%%%%%%%%%%%%%%%%%%%%%%%%%%%%%%%%%%%%%%%%
% #fuser
%%%%%%%%%%%%%%%%%%%%%%%%%%%%%%%%%%%%%%%%%%%%%%%%%%%%%%%%%%%%%%%%%%%%%%%%%%%%%%%%
\subsection{fuser}
\blscmd{fusr}查看使用某个文件/路径的所有进程。
\blscmd{-k}

%%%%%%%%%%%%%%%%%%%%%%%%%%%%%%%%%%%%%%%%%%%%%%%%%%%%%%%%%%%%%%%%%%%%%%%%%%%%%%%%
% #find, #grep
%%%%%%%%%%%%%%%%%%%%%%%%%%%%%%%%%%%%%%%%%%%%%%%%%%%%%%%%%%%%%%%%%%%%%%%%%%%%%%%%
\section{find and grep}
find是在目录中搜索文件。
grep是在文件中搜索字符串。
grep的名字源自ed行编辑器中的“全局正则表达式打印”\blscmd{g/re/p},
这个命令中的g表示在全局范围,/re/表示查找对象(正则表达式),p表示打印。

\blscmd{pgrep}和\blscmd{pkill},man手册中这两个工具是并列的,因此也一并记录。
\blscmd{pgrep}用来查找程序对于的PID,如\blscmd{pgrep fget},查找fget的PID。
\blscmd{pkill}向匹配的程序发送默认的SIGTERM信号。
\bllcmd{pgrep -u root sshd}列出所有属于root的sshd进程(AND)。
\bllcmd{pgrep -u root,daemon}列出所有属于root的进程,或deamon进程(OR)。

\verb|find ./ -name "*.h"  -exec grep setsock "{}" \;|

\blscmd{pgrep fget}则会搜索并输出fget的pid。
\blscmd{top -p `pgrep fget`}则可单只查看fget的信息。

``逻辑与"与``逻辑或"的grep表达式
\bllcmd{cat f \textbar grep ``word1 $\backslash$\textbar word2''}
\bllcmd{cat f \textbar grep ``word1.*word2''}

grep命令行参数:\\
\begin{tabular}{ll}
-i          & 不区分大小写\\
-v          & 反向搜索,不显示匹配项\\
-c          & 打印匹配行数,不显示具体的匹配内容\\
-I          & 不搜索二进制文件\\
\end{tabular}

\bllcmd{\$ netstat --tcp -n \textbar grep 6699 \textbar grep ES -c}

\blscmd{-I}不搜索二进制文件,同等\blscmd{--binary-files=without-match}。

%%%%%%%%%%%%%%%%%%%%%%%%%%%%%%%%%%%%%%%%%%%%%%%%%%%%%%%%%%%%%%%%%%%%%%%%%%%%%%%%
% bc, dc
%%%%%%%%%%%%%%%%%%%%%%%%%%%%%%%%%%%%%%%%%%%%%%%%%%%%%%%%%%%%%%%%%%%%%%%%%%%%%%%%
\section{bc and dc}
任意进制间的转换:
\bllcmd{ \$ echo 'ibase=10;obase=2;207' \textbar bc }
输出\blscmd{11001111}



%%%%%%%%%%%%%%%%%%%%%%%%%%%%%%%%%%%%%%%%%%%%%%%%%%%%%%%%%%%%%%%%%%%%%%%%%%%%%%%%
% #gcc
%%%%%%%%%%%%%%%%%%%%%%%%%%%%%%%%%%%%%%%%%%%%%%%%%%%%%%%%%%%%%%%%%%%%%%%%%%%%%%%%
\chapter{GCC}
\noindent\href{http://gcc.gnu.org/onlinedocs/}{GCC online documentation}GCC所有版本的手册列表。
\noindent\href{http://www.gnu.org/software/hello/manual/libc/}{GNU C Library Manual}其中有关于GNU C库
的细节说明。并且给出了一些编程技巧,如malloc hook,debug traceback(), 锁页mlock()等的说明。其中还有一些关于GNU
平台上特殊的函数等介绍。

\section{Compile Parameter}


\paragraph{-g}
生成调试信息。-glevel指定调试级别,默认为-g2。-g0表示没有任何调试信息,即-g0是-g取反(negates)。
-g3附加更多的调试信息,如宏定义/展开等。查看宏用\blscmd{p macroname},查看宏的展开形式用:
\blscmd{macro expand macroname}。需要查看宏时,还可加gcc选项-gdwarf-2。
    
\blscmd{man gcc}上说-g参数会为GDB生成专用调试信息,
这些调试信息会让GDB更好工作,但如果使用其他调试器,则可能因此而崩溃。
所以可以使用其他命令生成指定的调试信息,如\blscmd{-gstabs+},\blscmd{-gstabs},
\blscmd{-gxcoff},\blscmd{-gvms}等。

\paragraph{-D}
在命令行提供宏定义,如:
\bllcmd{g++ -DVERSION="\"fget 0.1.0\"" -DBUILD\_HOST=\"`uname -r`\" test.cpp}
需要注意引号的使用。双引号防止shell将参数里的空格作为命令行分隔符;
斜杠双引号,即双引号之间的内容,表示宏的值,包括双引号本身,
使用斜杠是防止shell解释引号。

\blitem{-fbounds-checking}{在代码中插入数组边界检查。man gcc说目前只对Java和Fortran管用。}
\blitem{-fstack-protector/-fstack-protector-all}{检查栈上数据。参考\ref{cpp-buffer-overflow-protection}的说明。}

\blitem{-static}{静态链接。man gcc的解释是:prevents linking with the shared libraries.
如:\blscmd{g++ -static hello.cpp}生成的a.out文件约1.3M,而不加此参数生成文件约6K。
也不能用\blscmd{ldd}查看a.out文件了,因为不是动态可执行文件。
}

\blitem{-fno-builtin/-fno-builtin-function}{不识别那些不以\_\_builtin\_作为前缀的内建函数。
GCC通常会为内建函数生成更有效率的代码,例如\blscmd{alloca}(在栈上分配空间)会成为一条指令,
而\blscmd{memcpy}则可能成为inline copy loops。这样做的结果就是生成代码更小更快,但这些函数
也许就不再存在,你不能设置断点或通过链接外部库来修改它们的行为。在使用\cppheader{math.h}
中的某些函数时,就可能遇见这些内建函数。
}

g++可以通过\blscmd{-fabi-version}控制ABI版本。另一些具体细节的控制,如\blscmd{-fpack-struct},
\blscmd{-fno-exception}等。

\blscmd{gcc --help=warning}查看关于warning的说明。

\section{GCC Assemble} \label{language-gcc-assemble}
\%ebp: frame pointer, 一般在所有函数调用第一行都是\blscmd{push \%ebp}。

\%eax: 用于存储函数返回值。GDB调试时查看函数返回值用命令\blscmd{p \$rax}。

\section{Ccache}
gcc编译时,用\blscmd{ccache},在编译时生成cache,以加速再次编译C/C++程序的速度,不过生成的文件也是颇大。
CCACHE\_DIR的默认值是\blscmd{\$HOME/.ccache}。可以直接删除此目录,也可通过\blscmd{ccache -C}删除编译缓存。
\blscmd{-F max-file-size}和\blscmd{-M maxsize}设置最大缓存文件大小,和缓存最大占用空间。
\blscmd{-s}查看缓存统计信息,包括命中次数,cache大小等。

除以上命令,还有许多环境变量控制ccache。

%%%%%%%%%%%%%%%%%%%%%%%%%%%%%%%%%%%%%%%%%%%%%%%%%%%%%%%%%%%%%%%%%%%%%%%%%%%%%%%%
% #gdb 
%%%%%%%%%%%%%%%%%%%%%%%%%%%%%%%%%%%%%%%%%%%%%%%%%%%%%%%%%%%%%%%%%%%%%%%%%%%%%%%%
\chapter{Debuging with GDB}
\section{Thinking in Debug}
常用的调试技巧有:

\paragraph{使用日志系统}
日志需要可以分级控制,可以用某种简单的开关控制,如C/C++的日志宏。
当出现系统级错误时,需要检查系统日志,如\blscmd{/var/log/*}, \blscmd{dmesg}等。

\paragraph{调试内核}
调试内核用的\blscmd{kgdb}, \blscmd{kdb}。
以及其他一些专用的调试器,如称为全能调试器的\blscmd{ODB}等。
\blscmd{kgdb}调试Linux内核需要两台计算机。\blscmd{kdb}调试内核只需一台计算机。
难以用普通调试器调试内核,是因为内核非以普通进程存在等。

\paragraph{使用断言}
使用适合项目的自定义ASSERT断言。可以通过-DNDEBUG编译选项关闭assert()宏。

\paragraph{使用调试器}
如\blscmd{GDB}。

\paragraph{跟踪系统调用}
使用\blscmd{strace}和\blscmd{ltrace}跟踪系统调用和库调用。

\paragraph{其他手段}
\blscmd{/proc/modules}中存储有当前所有被加载的系统模块。调试系统内核用\blscmd{printk}函数。

C/C++程序与内存相关的调试时最耗时的调试,有许多工具和方法可以帮助调试:

运行时用\blscmd{\$MALLOC\_CHECK\_=2 prog}可以检查内存。

\blscmd{efence}库可以检查两类内存问题:访问malloc()分配空间之外的内存;
访问已被free()的内存。只需在编译时静态或动态链接efence库即可。

\section{GDB Introduction}
\noindent\href{http://www.yolinux.com/TUTORIALS/GDB-Commands.html}{调试STL的GDB脚本}
用此脚本可以直接打印STL容器的内容。\href{http://www.stanford.edu/~afn/gdb\_stl\_utils/}{这里}
也有一个脚本可供打印STL。

在GDB中运行GDB脚本:source ~/.gdbinit

GDB对C++的支持不好(也许要追溯到GCC对C++的支持就不好),因此调试时不得以采用许多Tricks。
比如查看STL容器等,通常GDB会打印出令人费解的调试信息,通常你需要人工过滤掉你不需要的输出。
当然有热心人提供了GDB脚本来支持C++调试,但毕竟不是官方支持,而且这些脚本通常又是针对某一个特定的GDB版本。
GDB的一种Trick是在程序内部编写调试函数,并在GDB中用call func(parameters)的技巧。

GDB对回退调试(即返回到已经运行过的地方,rewind)的支持并不理想,是以bookmark方式实现。
回退调试也被称作反向调试,曾有GDB Path支持(record)。

在GDB中设置变量值,如\blscmd{p \$p=(Peer*) 0x86d9068},则可以像使用程序变量一样使用在GDB中定义的变量,如:
\blscmd{p \$p-\textgreater str()}。

\blscmd{sharedlibrary}加载系统动态库的符号文件。
\blscmd{show auto-solib-add}查看动态库是否自动加载。

GDB会捕获信号,如在socket编程常见的SIGPIPE。用命令\blscmd{handle SIGPIPE}
查看GDB对此信号的处理方式,用\blscmd{handle SIGPIPE nostop}修改处理方式,使之不中断程序。
对非错误信号,如SIGALRM, SIGWINCH, SIGCHILD,GDB默认执行nostop, noprint, pass操作,
对错误信号,GDB默认执行stop, print, pass操作。

\section{Debug Core Dump}
如果出现不可恢复的错误,OS将为程序生成core dump。
GDB调试core dump的命令是\blscmd{gdb prog core.PID}。
进入GDB调试后,通常用命令\blscmd{bt}或\blscmd{where}查看程序奔溃时的堆栈情况。
不过如果程序奔溃时堆栈也被破坏掉了(如越界写操作,可能将保存堆栈信息的内存破坏),
那么用bt或where命令就不能准确的定位问题,而GDB的输出可能是???等不可读信息。

如果程序奔溃时没有生成core.PID文件,则可能是系统限制,
检查\blscmd{ulimit -c}的值,如果输出为0,则表示当前shell禁止生成core文件。
可以修改其值,如\blscmd{ulimit -c 1024},表示可以生成core文件,
且将其大小限制为1024 bytes。

\section{Examining the Program}
\subsection{Examining the Stack}
使用\blscmd{bt},\blscmd{where}命令查看堆栈。
使用\blscmd{frame}查看调用帧。检查bt输出的调用栈,如果想用print命令打印函数参数却失败,
可以尝试用\blscmd{up}和\blscmd{down}移动堆栈,从而查看函数参数。

\subsection{Breakpoint}
\blscmd{display}自动显示断点处的数据,\blscmd{undisplay}取消显示。
\blscmd{jump}可以往回跳转,但不是起到rewind的作用,要回退调试,需要bookmark。

\blscmd{tbreak}可以用在直接跳转到指定行的场合,例如快速跳过while/for循环。
跳过循环体的标准做法可能是\blscmd{until}或\blscmd{u},GDB help的解释是:
Execute until the program reaches a source line greater than the current,
or a specified location within the current frame.

\blscmd{stepi}和\blscmd{nexti}是单步跟踪一条机器指令。

\section{Debug with Remote Target}
要远程调试程序,需在远程机上运行gdbserver,如:\bllcmd{gdbserver host:2345 --attach 27523}
注意,gdbserver参数的顺序,在某些系统上,与\blscmd{Debugging with GDB}手册的描述有所不同。

在本机上启动gdb,之后用\blscmd{file}命令加载符号文件,即本地编译的可执行文件;
再用\blscmd{target}命令连接目标机:\bllcmd{target remote tcp:192.168.16.105:2345}
之后可以像调试本机文件一样调试远程文件。

%%%%%%%%%%%%%%%%%%%%%%%%%%%%%%%%%%%%%%%%%%%%%%%%%%%%%%%%%%%%%%%%%%%%%%%%%%%%%%%%
% make
%%%%%%%%%%%%%%%%%%%%%%%%%%%%%%%%%%%%%%%%%%%%%%%%%%%%%%%%%%%%%%%%%%%%%%%%%%%%%%%%
\chapter{make}

%%%%%%%%%%%%%%%%%%%%%%%%%%%%%%%%%%%%%%%%%%%%%%%%%%%%%%%%%%%%%%%%%%%%%%%%%%%%%%%%
% binutils
%%%%%%%%%%%%%%%%%%%%%%%%%%%%%%%%%%%%%%%%%%%%%%%%%%%%%%%%%%%%%%%%%%%%%%%%%%%%%%%%
\chapter{binutils}
\blscmd{binutils}是对多种目标文件的操作工具集。
binutil主要包括以下工具:

\begin{tabular}{|l|l|}\hline
as          & 汇编器\\\hline
ld          & 链接器\\\hline
gprof       & 性能分析profiler \\\hline
addr2line   & 映射地址到代码行/文件 \\\hline
ar          & create, modify, and extract from archives\\\hline
nm          & list symbols in object files\\\hline
objcopy     & copy object files, possibly making changes\\\hline
objdump     & dump information about object files\\\hline
readelf     & display content of ELF files\\\hline
size        & list total and section sizes\\\hline
strings     & list printable strings\\\hline
strip       & remove symbols from an object file\\\hline
\end{tabular}

\blscmd{addr2line}映射地址到代码行/文件。
\bllcmd{\$ addr2line -e ./fget -s 08086890}
输出\blscmd{main.cpp:10}。
其中addr2line默认文件名为a.out,可用\blscmd{-e}指定输入文件。
\blscmd{-s}是不输出路径名。

\blscmd{strip}可以删除目标文件里的unnecessary信息。
试验\blscmd{strip}把\blscmd{fget}从原始的14MB变为652KB。

%%%%%%%%%%%%%%%%%%%%%%%%%%%%%%%%%%%%%%%%%%%%%%%%%%%%%%%%%%%%%%%%%%%%%%%%%%%%%%%%
% Program tools
%%%%%%%%%%%%%%%%%%%%%%%%%%%%%%%%%%%%%%%%%%%%%%%%%%%%%%%%%%%%%%%%%%%%%%%%%%%%%%%%
\chapter{Program Tools and Compile System}

%%%%%%%%%%%%%%%%%%%%%%%%%%%%%%%%%%%%%%%%%%%%%%%%%%%%%%%%%%%%%%%%%%%%%%%%%%%%%%%%
% valgrind
%%%%%%%%%%%%%%%%%%%%%%%%%%%%%%%%%%%%%%%%%%%%%%%%%%%%%%%%%%%%%%%%%%%%%%%%%%%%%%%%
\section{Valgrind}
Valgrind之名来自北欧神话中奥丁\footnote{奥丁是北欧神话主神,创造了人类,
掌管死亡、战斗、诗歌、魔法及智慧等。是众神之父。}神殿Valhalla的圣神入口。

Valgrind对全局数组,或栈上数组的访问越界无能为力。Valgrind FAQ上的解释是,
在目前memcheck的工作方式下,没有合理的方法实现访问越界检查。

\blitem{--db-attach=yes}{当Valgrind发现错误时,挂载调试器。}
\blitem{--leak-check=yes}{当使用memcheck时,检查内存泄漏。}

\subsection{Callgrind}
Callgrind用来监察函数的调用,可以分析函数的调用频率。
与memcheck不同的是,callgrind并不向标准输出写数据,而是在程序结束时,
将报告写入一个文件。如果想实时的监控callgrind的输出,可以使用
~callgrind\_annotate~和~callgrind\_control~这两个命令行工具。

\blscmd{callgrind\_annotate}读取callgrind的输出,并输出更易理解的内容。
\blscmd{callgrind\_control}可以实时观测callgrind的报告。

命令:
\begin{verbatim}
$ valgrind --tool=callgrind ./release/fget 
$ callgrind_control -b
\end{verbatim}
第二个命令callgrind\_control是在另一个终端开启的,因为valgrind占用了标准输出。
callgrind\_control的常用参数是:-b: 输出调用堆栈(像GDB的堆栈输出,但更简洁),
-s:输出程序状态,如PID,线程数等。


%%%%%%%%%%%%%%%%%%%%%%%%%%%%%%%%%%%%%%%%%%%%%%%%%%%%%%%%%%%%%%%%%%%%%%%%%%%%%%%%
% objdump 
%%%%%%%%%%%%%%%%%%%%%%%%%%%%%%%%%%%%%%%%%%%%%%%%%%%%%%%%%%%%%%%%%%%%%%%%%%%%%%%%
\section{objdump}
object可以反汇编二进制文件,如\blscmd{objdump -d ./fget}。
\paragraph{-d}反汇编二进制文件。
\paragraph{-C, --demangle}可以指定demangle的风格。
\paragraph{-l}输出携带行号和文件名信息。

%%%%%%%%%%%%%%%%%%%%%%%%%%%%%%%%%%%%%%%%%%%%%%%%%%%%%%%%%%%%%%%%%%%%%%%%%%%%%%%%
% nm 
%%%%%%%%%%%%%%%%%%%%%%%%%%%%%%%%%%%%%%%%%%%%%%%%%%%%%%%%%%%%%%%%%%%%%%%%%%%%%%%%
\section{nm}
\blitem{-C/--demangle[=style]}{Decode (demangle) low-level names into user-level names.}
\blitem{-l/--line-numbers}{打印符号对应的文件名和行号。}

%%%%%%%%%%%%%%%%%%%%%%%%%%%%%%%%%%%%%%%%%%%%%%%%%%%%%%%%%%%%%%%%%%%%%%%%%%%%%%%%
% Network tools, #network
%%%%%%%%%%%%%%%%%%%%%%%%%%%%%%%%%%%%%%%%%%%%%%%%%%%%%%%%%%%%%%%%%%%%%%%%%%%%%%%%
\chapter{Network Tools and Configure}
本章的网络工具包括netstat, route, ifconfig, traceroute, tcpdump,
iptables, ethtool, wget, curl, 
vnstat(实时网络流量统计), nmap(端口扫描侦测)等。

为使新系统连上网络,需配置:
\begin{enumerate}
\item 配置名字服务器。修改/etc/resolv.conf文件,添加DNS服务器地址。
\item 配置IP。用命令ifconfig eth0配置IP地址和子网掩码。
\item 配置路由。用命令route修改路由表。
\end{enumerate}

配置静态IP和路由,在Redhat/CentOS/Fedora上需要修改\blscmd{/etc/sysconfig/network-scripts/}路径下的
\blscmd{ifcfg-eth0}(配置IP)和\blscmd{route-eth0}(配置路由)文件。

%%%%%%%%%%%%%%%%%%%%%%%%%%%%%%%%%%%%%%%%%%%%%%%%%%%%%%%%%%%%%%%%%%%%%%%%%%%%%%%%
% #configure files
%%%%%%%%%%%%%%%%%%%%%%%%%%%%%%%%%%%%%%%%%%%%%%%%%%%%%%%%%%%%%%%%%%%%%%%%%%%%%%%%
\section{Configure Files}

%%%%%%%%%%%%%%%%%%%%%%%%%%%%%%%%%%%%%%%%%%%%%%%%%%%%%%%%%%%%%%%%%%%%%%%%%%%%%%%%
% #/etc/hosts
%%%%%%%%%%%%%%%%%%%%%%%%%%%%%%%%%%%%%%%%%%%%%%%%%%%%%%%%%%%%%%%%%%%%%%%%%%%%%%%%
\blscmd{/etc/hosts}配置主机名和IP地址的映射关系。添加如下一行:
\bllcmd{192.168.16.71 ls.funshion.com}
则调用如\blscmd{ping ls.funshion.com}命令时就会避开DNS查询,而直接连接指定的IP。
在Windows下,该配置文件名为:\blscmd{C:/WINDOWS/system32/drivers/etc/hosts}。

%%%%%%%%%%%%%%%%%%%%%%%%%%%%%%%%%%%%%%%%%%%%%%%%%%%%%%%%%%%%%%%%%%%%%%%%%%%%%%%%
% #ethtool
%%%%%%%%%%%%%%%%%%%%%%%%%%%%%%%%%%%%%%%%%%%%%%%%%%%%%%%%%%%%%%%%%%%%%%%%%%%%%%%%
\section{ethtool}
用\blscmd{ethtool}可以查看网络接口的硬件信息,并作修改。
例如使用ethtool修改网卡速度,如:
\bllcmd{ethtool eth0 -s speed 100  // 100Mbps}
网络速度受网卡硬件限制,同时也受网络接入速度限制,如接入的路由限制为100Mbps,
而设置1000Mbps则会出错。

网卡可能出于半双工状态,设置网卡的工作模式为自动协商:
\bllcmd{ethtool -s eth0 autoneg on}

%%%%%%%%%%%%%%%%%%%%%%%%%%%%%%%%%%%%%%%%%%%%%%%%%%%%%%%%%%%%%%%%%%%%%%%%%%%%%%%%
% #nmap
%%%%%%%%%%%%%%%%%%%%%%%%%%%%%%%%%%%%%%%%%%%%%%%%%%%%%%%%%%%%%%%%%%%%%%%%%%%%%%%%
\section{nmap}
\blscmd{nmap}是为Network Mapper之意,用作端口扫描。
手册上的一个例子是\blscmd{nmap -A -T4 www.google.com playground},可以看到
目标开放的端口,服务名词,使用的服务程序等。尽管都是nmap通过猜测所得,
但依然很有参考价值,比如扫描西科校友网,发现ftp(21/tcp)使用Serv-U ftpd 6.1,
http(80/tcp)使用IIS 6.0,且没有robots.txt。扫描西科(太多开放端口了),
http使用Apache 2.2.8,关闭了https(443/tcp)等。

\blscmd{-O}可以让nmap猜测对端的操作系统。

nmap猜测的对端系统启动日期uptime不太准。

%%%%%%%%%%%%%%%%%%%%%%%%%%%%%%%%%%%%%%%%%%%%%%%%%%%%%%%%%%%%%%%%%%%%%%%%%%%%%%%%
% #vnstat
%%%%%%%%%%%%%%%%%%%%%%%%%%%%%%%%%%%%%%%%%%%%%%%%%%%%%%%%%%%%%%%%%%%%%%%%%%%%%%%%
\section{vnstat}
据说Gnome中有一个网络速度查看软件\blscmd{netspeed},可以查看网络传输速度。

\blscmd{vnstat}可以实施监控网络速度,简单的用法如:\blscmd{vnstat -l -i eth0}
其中参数-l表示live,实时地更新速度。用Ctrl-C终止vnstat后,有一个数据统计输出。

vnstat源码包没有configure,只需直接make即可。make后的可执行文件在src/目录中。

%%%%%%%%%%%%%%%%%%%%%%%%%%%%%%%%%%%%%%%%%%%%%%%%%%%%%%%%%%%%%%%%%%%%%%%%%%%%%%%%
% #traceroute
%%%%%%%%%%%%%%%%%%%%%%%%%%%%%%%%%%%%%%%%%%%%%%%%%%%%%%%%%%%%%%%%%%%%%%%%%%%%%%%%
\section{traceroute}
\blscmd{traceroute}用于检测到达目的主机的中间路由。
通过traceroute可以查看到某个路由的时耗,调整系统性能。

%%%%%%%%%%%%%%%%%%%%%%%%%%%%%%%%%%%%%%%%%%%%%%%%%%%%%%%%%%%%%%%%%%%%%%%%%%%%%%%%
% #netstat 
%%%%%%%%%%%%%%%%%%%%%%%%%%%%%%%%%%%%%%%%%%%%%%%%%%%%%%%%%%%%%%%%%%%%%%%%%%%%%%%%
\section{netstat}

\begin{table}
\centering
\caption{netstat command list}
\begin{tabularx}{\textwidth}{lX}
-r          & 显示路由表。和\blscmd{route}命令作用相同。\\
-s          & 显示统计数据。\\
-n          & 不作名字解析。\\
-i          & 显示所有的网络接口。\\
-p          & 打印进程信息。\\
-c          & continue打印。\\
-o          & 显示timer。\\
-I          & 指定网络接口。可指定的接口包括\blscmd{eth0}, \blscmd{eth1}, \blscmd{lo}
            等,可以先用\blscmd{netstat -i}查看所有接口。
            在-I和接口间没有空格,参数需写作\blscmd{-Ieth0}。\\
指定协议    & 可指定的协议包括\blscmd{-t/--tcp}, \blscmd{-u/--udp},
                \blscmd{-S/--sctp}, \blscmd{-w/--raw}, \blscmd{-x/--unix}等。\\
\end{tabularx}
\end{table}

通过\blscmd{netstat -s}查看统计信息,可帮助改进程序性能。如观察TCP统计,可以获得
接收和发送的reset报文个数,可查看

通过\blscmd{netstat -i}或\blscmd{netstat -Ieth0}可查看网络接口的流量信息。

%%%%%%%%%%%%%%%%%%%%%%%%%%%%%%%%%%%%%%%%%%%%%%%%%%%%%%%%%%%%%%%%%%%%%%%%%%%%%%%%
% #wget and #curl 
%%%%%%%%%%%%%%%%%%%%%%%%%%%%%%%%%%%%%%%%%%%%%%%%%%%%%%%%%%%%%%%%%%%%%%%%%%%%%%%%
\section{wget and curl}
wget常用参数
\blitem{--limit-rate=amount}{下载限速。可以使用单位k, m作为后缀。
因为wget用sleep实现限速用,因此下载小文件时限速可能不能很好工作。}
\blitem{--user-agent}{让wget冒充其他客户程序。}

curl常用参数。\blscmd{--local-port PORT}指定本地端口。

%%%%%%%%%%%%%%%%%%%%%%%%%%%%%%%%%%%%%%%%%%%%%%%%%%%%%%%%%%%%%%%%%%%%%%%%%%%%%%%%
% tcpdump
%%%%%%%%%%%%%%%%%%%%%%%%%%%%%%%%%%%%%%%%%%%%%%%%%%%%%%%%%%%%%%%%%%%%%%%%%%%%%%%%
\section{tcpdump}
\noindent\href{http://www.tcpdump.org/}{Tcpdump Homepage} 这里还提供了一个名为libpcap的库,
随tcpdump一起发布,用于顶层抓包程序的编写。 \\
\noindent\href{http://www.tcpdump.org/tcpdump\_man.html}{Online man page} \\

\subsection{Command Paramters}
\paragraph{-D}列出所有tcpdump可以监听的物理接口。可以用\blscmd{-i}选择这个输出的接口,可以是数字或字符串。
如:\blscmd{tcpdump -i 1}或\blscmd{tcpdump -i eth0}。
如果系统没有\blscmd{ifconfig -a}命令,则\blscmd{tcpdump -D}就可派上用场。

\paragraph{-F}从\blscmd{-F}指定的文件读入filter expression,忽略命令行输入的表达式。

\paragraph{-i}指定tcpdump抓包的目标接口。如server和client运行在同一台计算机上,
用tcpdump抓包的命令是\blscmd{tcpdump -i lo},这里指定物理端口为lo本地环回端口。

\paragraph{-l}将输出行缓冲。一个应用如\blscmd{tcpdump -l \textbar tee dat},
或\blscmd{tcpdump -l \textgreater dat \& tail -f dat},输出到stdout和dat文件。

\paragraph{-n/-nn}\blscmd{-n}对IP地址不作DNS查询,\blscmd{-nn}对端口号不作名字转换。

\subsection{Expression}

\paragraph{host}指定监听的IP地址。\blscmd{src host}和\blscmd{dst host}分别制定源IP和目标IP地址。
\paragraph{tcp}指定监听TCP报文。
\paragraph{port}指定监听端口号,如\blscmd{tcpdump tcp port 6601}。

%%%%%%%%%%%%%%%%%%%%%%%%%%%%%%%%%%%%%%%%%%%%%%%%%%%%%%%%%%%%%%%%%%%%%%%%%%%%%%%%
% #iptables, #firewall
%%%%%%%%%%%%%%%%%%%%%%%%%%%%%%%%%%%%%%%%%%%%%%%%%%%%%%%%%%%%%%%%%%%%%%%%%%%%%%%%
\chapter{iptables}
iptables默认的表名叫filter,如果不用\blscmd{-t talbe-name}指定表,则规则都实施在filter表上。
除了filter表,iptables的内建表还有\blscmd{nat}和\blscmd{mangle}表。

Linux防火墙通常是由rc启动脚本中的一系列iptables命令来实现的。
iptables的常见形式为:\\
\blscmd{iptables -F chain-name},\\
\blscmd{iptables -P chain-name target},\\
\blscmd{iptables -A chain-name -i interface -j target}

\begin{tabular}{ll}
-A, --append        & 附加策略。\\
-I, --insert        & 插入策略。与-A不同是可指定策略位置。\\
-N, --new-chain     & 新建一条chain。\\

-f, --fragment      & 对于第二个或之后的framgmented报文起作用。\\

-m, --match [!]     & 指定匹配细则。\\

-F, --flush         & 把之前的所有规则从链中清除。\\
-X, --delete-chain  & 删除非内置的规则链。\\
-Z, --zero          & 清空所有计数(packages and bytes)。\\
-v, --verbose       & 详细输出。包括计数信息。\\

-j, --jump          & 跳转到指定处理规则。\\

-P                  & 给链设置一条默认的策略。\\
-p                  & 指定协议名。\\
-s                  & 源IP地址。\\
-d                  & 目标IP地址。\\
--sport             & 源端口。\\
--dport             & 目标端口。\\
-t                  & 指定表名。默认为filter表。\\
!                   & 否定一条子句。\\
\end{tabular}

target可以是:\blscmd{ACCEPT}, \blscmd{DROP}, \blscmd{REJECT}, \blscmd{LOG}, \blscmd{MIRROR},
\blscmd{QUEUE}, \blscmd{REDIRECT}, \blscmd{RETURN}, \blscmd{ULOG}。DROP是默默地丢弃,
REJECT则会返回一个ICMP错误消息,LOG提供日志,ULOG提供更加详细的日志。

\blscmd{-I, --insert chain [rulenum] rule-specification}和\blscmd{-A}类似,都是加入一条rule。
不同在-A是附加在chain末尾,而Insert(如果不指定rulenum, 即默认情况)是加入在rule之首。

\blscmd{-m, --match}可以跟很多参数。常见的如:
\blscmd{-m state --state STATE}, STATE可以为NEW, ESTABLISHED, RELATED等。
\blscmd{-m limit --limit RATE}, RATE单位可以是/second, /minute, /hour, /day。
\blscmd{-m limit --limit-burst N}, 一次最大匹配报文个数。    % todo 
\blscmd{-m conntrack --cstate STATE}, 类似--state, 但更详细。

\blscmd{[!] --syn}是只匹配设置SYN bit的TCP报文,ACK, RST, FIN位必须为0。

\blscmd{\$ iptables -L -n}列出所有的规则。

\blscmd{\$ iptables -I INPUT -p tcp --dport 6699 -j ACCEPT}

\blscmd{\$ iptables -A INPUT -p tcp ! --syn -m state --state NEW -j DROP}丢弃所有非SYN的新连接。

%%%%%%%%%%%%%%%%%%%%%%%%%%%%%%%%%%%%%%%%%%%%%%%%%%%%%%%%%%%%%%%%%%%%%%%%%%%%%%%%
% #monitor
%%%%%%%%%%%%%%%%%%%%%%%%%%%%%%%%%%%%%%%%%%%%%%%%%%%%%%%%%%%%%%%%%%%%%%%%%%%%%%%%
\chapter{Monitor Tools}
对于CPU,内存,磁盘I/O,网络I/O的信息,有多种名为\blscmd{*stat}的工具。某些是系统自带的,
某些则是需要自行安装的。以下略为区别:

\begin{tabular}{|l|l|l|}
netstat   & 自带    & 查看网络信息          \\\hline
vmstat    & 自带    & 查看虚拟内存信息      \\\hline
vnstat    & 安装    & 查看网络实时流量      \\\hline
iostat    & 安装    & 来自sysstat,查看磁盘实时信息     \\\hline
mpstat    & 安装    & 来自sysstat,查看CPU相关信息      \\\hline
pidstat   & 安装    & 来自sysstat,查看进程相关信息     \\\hline
sar       & 安装    & 来自sysstat,查看系统统计信息     \\\hline
\end{tabular}


\section{System Information}
\blscmd{lspci -vv}可以查看系统硬件的详细信息,不过一般不容易看懂,
因此需要其他工具帮助分析判断系统的硬件组成和其他系统信息。

\blscmd{locale}可以查看本地化信息。可以通过\blscmd{export LANG=zh\_CN.UTF-8}修改本地化信息。
设置本地化为英语可用\blscmd{en\_US}。

\blscmd{lsb\_release}可以查看发行版本(distribution)的信息,其中LSB是Linux Standard Base之意。
也可以通过\blscmd{cat /etc/issue}查看较为简单的发行版信息。
\blscmd{uname}可以查看一般性的系统信息。\blscmd{cat /proc/version}和uname命令输出大致相同。

%%%%%%%%%%%%%%%%%%%%%%%%%%%%%%%%%%%%%%%%%%%%%%%%%%%%%%%%%%%%%%%%%%%%%%%%%%%%%%%%
% #disk
%%%%%%%%%%%%%%%%%%%%%%%%%%%%%%%%%%%%%%%%%%%%%%%%%%%%%%%%%%%%%%%%%%%%%%%%%%%%%%%%
\section{Disk}
\href{http://www.acnc.com/04_02_01.html}{这里}介绍了Unix环境中的disk io benchmark工具。\\
\href{http://www.iozone.org/}{Iozone's Homepage}。对磁盘进行压力测试,并生成输出可视化。

在Linux系统,\blscmd{hdparm}命令可以查看ATA/IDE硬盘的属性。
SCSI和SATA硬盘的标识符都是sdX,可以用以下命令查看硬盘的类型:

\bllcmd{\$ grep -i ``[sh]d[a-z]'' /var/log/dmesg \textbar grep -i scsi}
就是查看dmesg的方式。

\blscmd{/proc/diskstats}提供了磁盘信息。可以在
\blscmd{/sys/block}中查看块设备的更加全面的信息。

\subsection{Sysstat}
\blscmd{iostat}是\blscmd{sysstat}中的一个组件,用来查看磁盘实时性能。

命令示例:\blscmd{iostate -d sda sdb -m 10},每10秒刷新一次屏幕,查看sda, sdb的读写信息,并以MB为单位。

\subsection{Iozone}
命令实例:\blscmd{iozone -RazMb diskio.xls}。

\begin{table}
\centering
\begin{tabularx}{\textwidth}{lX}
-R      & 生成Excel文件。\\
-b      & 指定Excel文件名。\\
-a      & full automatic mode。文件大小值域16k-512M,读写块值域4k-16k。\\
-M      & 使用uname()获取内核信息。\\
-s      & 指定文件大小。\\
-r      & 指定块大小。\\
-g      & 最大测试文件大小。\\
\end{tabularx}
\end{table}

Iozone的输出以Kbytes/s为单位。但测试的输出结果很奇怪,其值远大于用其他性能观察工具所视之值。待研究。

Iozone输出几种类型的测试信息。值得注意的是:\blscmd{Stride Read Report},手册介绍举例说,如每读4K数据,
就偏移200K,再读取4K数据,如此反复迭代。\blscmd{Fwrite}, \blscmd{Fread}表示用fwrite(), fread()系统调用。

%%%%%%%%%%%%%%%%%%%%%%%%%%%%%%%%%%%%%%%%%%%%%%%%%%%%%%%%%%%%%%%%%%%%%%%%%%%%%%%%
% #filesystem, #fs
%%%%%%%%%%%%%%%%%%%%%%%%%%%%%%%%%%%%%%%%%%%%%%%%%%%%%%%%%%%%%%%%%%%%%%%%%%%%%%%%
\section{FileSystem}
\blscmd{df -h}显示文件系统使用情况。
输出项中的/dev/shm表示virtual memory file system。

\blscmd{du}显示文件目录的占用空间大小。 
\blscmd{du -hs}显示当前目录(包括子目录)的大小。

%%%%%%%%%%%%%%%%%%%%%%%%%%%%%%%%%%%%%%%%%%%%%%%%%%%%%%%%%%%%%%%%%%%%%%%%%%%%%%%%
% #memory
%%%%%%%%%%%%%%%%%%%%%%%%%%%%%%%%%%%%%%%%%%%%%%%%%%%%%%%%%%%%%%%%%%%%%%%%%%%%%%%%
\section{Memory}
用\blscmd{free}、\blscmd{cat /proc/meminfo}查看内存。

free命令行参数:\\
\begin{tabular}{ll}
-b -k -m    & 显示内存大小的单位。\\
-s delay    & 显示延迟。\\
\end{tabular}

%%%%%%%%%%%%%%%%%%%%%%%%%%%%%%%%%%%%%%%%%%%%%%%%%%%%%%%%%%%%%%%%%%%%%%%%%%%%%%%%
% #cpu
%%%%%%%%%%%%%%%%%%%%%%%%%%%%%%%%%%%%%%%%%%%%%%%%%%%%%%%%%%%%%%%%%%%%%%%%%%%%%%%%
\section{CPU}
\blscmd{uptime}只有一行输出,用于查看系统的启动时间,当前用户数,以及系统负载。
系统负载有三个输出,包括最近1分钟,5分钟,15分钟内的系统负载。
如果系统负载的值超过3,则可能处于较大压力。
uptime的输出和\blscmd{w}输出的第一行相同。

\blscmd{uptime}和\blscmd{w}的man手册页说从\blscmd{/var/run/utmp}获取登录用户信息。
如果\blscmd{cat /var/run/utmp}则输出乱码以致干扰终端显示,慎用。


%%%%%%%%%%%%%%%%%%%%%%%%%%%%%%%%%%%%%%%%%%%%%%%%%%%%%%%%%%%%%%%%%%%%%%%%%%%%%%%%
% procinfo 
%%%%%%%%%%%%%%%%%%%%%%%%%%%%%%%%%%%%%%%%%%%%%%%%%%%%%%%%%%%%%%%%%%%%%%%%%%%%%%%%
\subsection{procinfo}
\blscmd{procinfo}主要输出cpu和内存信息。
在一些Linux发行版上不是自带的。procinfo主要解析\blscmd{/proc}文件系统,
与\blscmd{free},\blscmd{uptime}等输出类似。它的命令行参数如下:
\blscmd{-f} 全屏显示。
\blscmd{-nN} 间隔N秒输出一次。但不是如top等命令的刷屏,而是直接往下输出。
\blscmd{-m} 输出模块和设备,而非cpu和内存信息。



%%%%%%%%%%%%%%%%%%%%%%%%%%%%%%%%%%%%%%%%%%%%%%%%%%%%%%%%%%%%%%%%%%%%%%%%%%%%%%%%
% #rpm
%%%%%%%%%%%%%%%%%%%%%%%%%%%%%%%%%%%%%%%%%%%%%%%%%%%%%%%%%%%%%%%%%%%%%%%%%%%%%%%%


%%%%%%%%%%%%%%%%%%%%%%%%%%%%%%%%%%%%%%%%%%%%%%%%%%%%%%%%%%%%%%%%%%%%%%%%%%%%%%%%
% other tools
%%%%%%%%%%%%%%%%%%%%%%%%%%%%%%%%%%%%%%%%%%%%%%%%%%%%%%%%%%%%%%%%%%%%%%%%%%%%%%%%
\chapter{Other Tools}

%%%%%%%%%%%%%%%%%%%%%%%%%%%%%%%%%%%%%%%%%%%%%%%%%%%%%%%%%%%%%%%%%%%%%%%%%%%%%%%%
% perforce/p4
%%%%%%%%%%%%%%%%%%%%%%%%%%%%%%%%%%%%%%%%%%%%%%%%%%%%%%%%%%%%%%%%%%%%%%%%%%%%%%%%
\section{Perforce/P4}
一个用户可以有多个Client。Linux用命令\blscmd{p4 client cilentname}即可创建一个新的Client。
默认打开一个名为``Client Specification''的文件,修改以下字段:

\begin{tabular}{|l|l|}\hline
Client      & client名  \\\hline
Owner       & 一个owner可以有多个client \\\hline
Root        & 本地路径,如/home/berlin,按默认不用理会  \\\hline
View        & perforce和本地的映射关系  \\\hline
\end{tabular}

View的一个实例如\blscmd{//Project/... //berlin/workspace/...},\ldots表示所有文件,是一个通配符。
以上表示perforce的//Project/映射为/home/berlin/workspace,前缀/home/berlin由Root指定。

%%%%%%%%%%%%%%%%%%%%%%%%%%%%%%%%%%%%%%%%%%%%%%%%%%%%%%%%%%%%%%%%%%%%%%%%%%%%%%%%
% #berkeleydb
%%%%%%%%%%%%%%%%%%%%%%%%%%%%%%%%%%%%%%%%%%%%%%%%%%%%%%%%%%%%%%%%%%%%%%%%%%%%%%%%
\section{BerkeleyDB}
进入build\_unix,执行\blscmd{../dist/configure},不能在dist目录执行configure。
默认make install将库安置于/usr/local/BerkeleyDB.4.7/lib。

%%%%%%%%%%%%%%%%%%%%%%%%%%%%%%%%%%%%%%%%%%%%%%%%%%%%%%%%%%%%%%%%%%%%%%%%%%%%%%%%
% #cron
%%%%%%%%%%%%%%%%%%%%%%%%%%%%%%%%%%%%%%%%%%%%%%%%%%%%%%%%%%%%%%%%%%%%%%%%%%%%%%%%
\section{cron}
\blscmd{cron}自动执行周期性任务。在RH上cron无故被叫作了crond。
cron每分钟醒来一次,会执行crontab文件中的命令,如果crontab文件被修改,
则重新加载。因为掉电或者修改系统时钟导致的任务没有执行,cron不会补过执行。

crontab文件存在于以下3个地方:
\begin{enumerate}
\item \blscmd{/var/spool/cron/}这里存放一般用户的crontab。当用户执行\blscmd{crontab -e}
时,即可编辑其所属之crontab文件,并最后存储于/var/spool/cron/username中。但一般用户不能直接
访问/var中的文件。
\item \blscmd{/etc/crontab},这个文件由系统管理员维护,手工编辑。
\item \blscmd{/etc/cron.d/}这里通常存放软件包需要的cron操作。
\end{enumerate}

除了以上路径,Linux中另有一些如\blscmd{/etc/cron.weekly}, \blscmd{/etc/cron.daily},
\blscmd{/etc/cron.monthly}的目录,里面存放shell脚本,用以cron调度执行。

\blscmd{/etc/cron.deny}是拒绝指定用户执行cron。

%%%%%%%%%%%%%%%%%%%%%%%%%%%%%%%%%%%%%%%%%%%%%%%%%%%%%%%%%%%%%%%%%%%%%%%%%%%%%%%%
% #linux
%%%%%%%%%%%%%%%%%%%%%%%%%%%%%%%%%%%%%%%%%%%%%%%%%%%%%%%%%%%%%%%%%%%%%%%%%%%%%%%%
\chapter{Linux Distrubtion}
CentOS \href{http://www.centos.org/docs/5/}{CentOS 5.x文档库}。

\begin{tabular}{ll}
Fedora  & 是一种浅顶软呢帽,是RedHat上的那顶帽子。\\
Ubuntu  & 是一个南非民族的观念,意为“人道待人”。\\
Debian  & 是Ian Murdock在1998年命名的,来自他女友Debra和他名字的混合。\\
\end{tabular}

以下是在使用各种发行版中的问题记录,基本是一些小问题,但也花费了不少时间。

\bldate{2009/03/31} 在安装Debian系统时,如果只选择“桌面系统”和“基本系统”,你可能什么事也干不了,
因为可以没有gcc, gdb, make等一系列开发工具。所以还得用\blscmd{apt-get}把这些软件都安装。
而Debian安装时的可选粒度也不如fedora, CentOS的好,因此比较麻烦一些。

\bldate{2009/03/30} 安装Debian 5.00后,用SSH登录总是失败。检查Debian网络配置,一切正常。
可以连接上互联网。最后发现默认居然没有安装sshd服务。因此不得已,\blscmd{apt-get install ssh}安装sshd,
再\blscmd{/etc/init.d/ssh start}启动sshd。如此这般,就可以ssh登录了。
