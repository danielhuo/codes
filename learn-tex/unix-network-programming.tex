\part{UNIX Network Programming}

UNIX网络编程知识的主要来源是W. Richard Stevens的\emph{UNIX Networkiing Programming}。
主要是API的使用,网络编程的常用模式,各种协议的网络应用等。

这部分关注的一个领域是高性能服务器的设计、开发。包括各种I/O模型的学习,如nonblocking socket
相关的select, poll, kqueue, epoll, complete port等。包括一些优化技术技巧,如cache,thread
等的使用。

\chapter{TCP and Socket API}
\subsection{SYN}
Windows系统中,只允最大存在10个TCP半开连接(half-open tcp connections)。
因为由此限制,在开启P2P软件时,通常P2P软件会占满此半开连接的限制,
所以用户打开web页面等网络应用时会很慢。不过如果将限制打开(很多P2P软件都由此功能),
会消耗更多系统资源,也有说导致路由器梗死,防火墙瘫痪等。因此有建议半开连接数不要超过50个。
\blcomment{打开限制,使P2P软件快速拉起流量?}
修改\blscmd{C:/WINDOWS/system32/drivers/tcpip.sys}可以改变这个限制。
\href{http://half-open.com/}{half-open.com}提供了工具查看和修改这个值。

\href{http://en.wikipedia.org/wiki/SYN_flood}{SYN flood}是一种DoS攻击方式。
正常TCP连接建立经过三个阶段,client-(SYN)-server, server-(SYN/ACK)-client,
client-(ACK)-server,由此TCP连接完成。但恶意客户端可以通过两种方式对server施以SYN攻击。
其一是client不响应ACK包,其二是client填入一个虚假的IP地址,
因此server发送SYN/ACK后等待一个不存在的IP响应ACK。

\subsection{RST}
有三种情况产生RST报文。
\begin{enumerate}
\item connect到一个没有进程监听的端口。
\item 异常终止一个连接。这样做的好处在于,丢弃任何待发送数据,并立即发生复位报文;
RST的接收方可以区分另一方是异常关闭连接。Sockect API通过设置``linger on close"选项(SOL\_LINGER)
提供这种异常关闭的能力,这将导致close时发生RST报文。
\item 检查半打开连接。
\end{enumerate}

\section{Socket Options}
\subsection{SO\_KEEPALIVE}
在TCP层实现侦测无活动连接,TCP自动发送keep-alive probe给对端,对端可能出现三种响应。
其一是接收到ACK,表明对端还存在,其二是接收到RST,表示对端crashed或reboot,此时有ECONNRESET错误,
其三是对端未有任何响应,此时有更多的侦测包发送(see UNP3/e, p200),如果最后仍无响应,则有ETIMEDOUT错误。
但是如果某个keep-alive probe的响应是ICMP错误,则表明host unreachable,将有EHOSTUNREACH错误。

对于侦测时间,通常是2小时。这个时间是pre-kernel相关,而非pre-socket相关的,因此修改此值,
将影响所有的socket。这个选项的目的是,发现对端crashed或unreachable(掉线,断电等)。

此选项多用在服务器程序中,如应对half-open connection,keep-alive选项将发现half-open连接并中断它们。

\section{Nonblocking I/O}
与非阻塞IO相关的网络API包括:select, poll等传统I/O轮询函数,以及如
epoll(Linux), kqueue(FreeBSD), /dev/poll(Solaris), IOCP(Windows)等新的更为高效的函数。
\href{http://monkey.org/~provos/libevent/}{libevent}这个跨平台的事件处理库就在这些平台使用了对应的轮询函数。

\subsection{Nonblocking Socket}
\begin{enumerate}
\item Input操作,对应recv系列函数,如果是非阻塞sockect,而读取不到数据,则返回EWOULDBLOCK。
\item Output操作,对应send系列函数,如果是非阻塞,且发送缓冲已满,则返回EWOULDBLOCK。
\item accept操作,如果是非阻塞且没有新连接,返回EWOULDBLOCK。
\item connect操作,如果connect不能立即完成,则返回EINPROGRESS。
\end{enumerate}

有时select告诉我们文件描述符已可读,但recv()返回-1,errno为EWOULDBLOCK,此为正常情况,忽略。
在Linux下查看man select,其中BUGS中就提到这个问题,并举例说,当有数据到达时,select通知可读,
但此时若数据的checksum失败,则数据被抛弃,read()就将失败。而EAGAIN, EWOULDBLOCK的说明是:
the socket is marked non-blocking and the requested operation would block.
SystemV使用EAGAIN,而Berkeley使用EWOULDBLOCK,但现在大多系统都将两者设为等值。

注意,以上recv()返回-1,errno为EWOULDBLOCK。如果slect()返回可读,但recv()返回0,则说明对端关闭了socket,
此时本地socket也应该被close。

如果select()返回-1,则说明有TCP连接坏了(可能之前就接受到了SIGPIPE),要具体判断是那个连接坏了,可用以下方法。
1、对每个文件描述符调用select(),返回-1则说明此条连接已坏。
2、对每个文件描述符调用getsockopt(),取SOL\_ERROR选项的值,如果其值大于0,则说明此条连接已坏。

关于非阻塞socket的connect(),应当是比较复杂的。如果connect()返回-1且errno为EINPROGRESS,
则需要调用select()判断sockfd。且在select()返回文件可读或可写后,应用getsockopt()判断是否有错误发生。
UNP上说,非阻塞socket的connect()操作的移植性非常低。因为不同系统上的判定方法不尽相同。
以下是摘自UNP 3/e上的代码片段:

\subsection{select}
\begin{lstlisting}[language=C++]
#include <sys/select.h>
#include <sys/time.h>

int select(int maxfd, fd_set* readset, fd_set* writeset, 
           fd_set* exceptset, const struct timeval* timeout);

struct timeval{
    long tv_sec;    // seconds
    long tv_usec;   // microseconds
};

void FD_ZERO(fd_set* fdset);
void FD_SET(int fd, fd_set* fdset);
void FD_CLR(int fd, fd_set* fdset);
int  FD_ISSET(int fd, fd_set* fdset);
\end{lstlisting}

\blscmd{FD\_SETSIZE}定义于\cppheader{sys/types.h}。
在Linux上,此值一般为1024,如果修改此值,需要重新编译内核。
在Windows系统上,这个宏的值默认为64,可以直接修改。

select() return -1,表明有socket坏掉了。
select() 的最后一个参数,如果传入NULL,表明永久等待,直到有事件发生;如果传入0,表明立即返回;如果传入具体时间,
则是等待超时。

\subsection{epoll}
\begin{lstlisting}[language=C++]
#include <sys/epoll.h>

int epoll_create(int size);
int epoll_ctl(int epfd, int op, int fd,
              struct epoll_event* event);
int epoll_wait(int epfd, struct epoll_event* events,
               int maxevents, int timeout);

EPOLL_CTL_ADD   // op: add
EPOLL_CTL_MOD   // op: modify listen event
EPOLL_DEL       // op: delete

typedef union epoll_data{
    void*       ptr;
    int         fd;
    __unit32_t  u32;
    __uint64_t  u64;
}epoll_data_t;

struct epoll_event{
    __unit32_t      events;
    epoll_data_t    data;
};
\end{lstlisting}

以下是来自CU的讨论:
\blcopy{
LT(level triggered)是缺省的工作方式,并且同时支持block和no-block socket.在这种做法中,
内核告诉你一个文件描述符是否就绪了,然后你可以对这个就绪的fd进行IO操作。
如果你不作任何操作,内核还是会继续通知你的,所以,这种模式编程出错误可能性要小一点。
传统的select/poll都是这种模型的代表。}

\blcopy{ET(edge-triggered)是高速工作方式,只支持no-block socket。在这种模式下,
当描述符从未就绪变为就绪时,内核通过epoll告诉你。
然后它会假设你知道文件描述符已经就绪,并且不会再为那个文件描述符发送更多的就绪通知,
直到你做了某些操作导致那个文件描述符不再为就绪状态了(比如,你在发送,接收或者接收请求,
或者发送接收的数据少于一定量时导致了一个EWOULDBLOCK 错误)。
但是请注意,如果一直不对这个fd作IO操作(从而导致它再次变成未就绪),内核不会发送更多的通知(only once),
不过在TCP协议中,ET模式的加速效用仍需要更多的benchmark确认。}

\blcopy{在许多测试中我们会看到如果没有大量的idle-connection或者dead-connection,
epoll的效率并不会比 select/poll高很多,但是当我们遇到大量的idle-connection(例如WAN环境中存在大量的慢速连接),
就会发现epoll的效率大大高于select/poll。}
