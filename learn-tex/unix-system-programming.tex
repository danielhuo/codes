\part{UNIX System Programming}

UNIX系统编程知识的主要来源是W. Richard Stevens的\emph{Advanced Programming in the UNIX Environment}。
另外的来源是GNU的手册,包括libc手册,libstdc++的手册等。

%%%%%%%%%%%%%%%%%%%%%%%%%%%%%%%%%%%%%%%%%%%%%%%%%%%%%%%%%%%%%%%%%%%%%%%%%%%%%%%%
% GNU C/C++ Programming
%%%%%%%%%%%%%%%%%%%%%%%%%%%%%%%%%%%%%%%%%%%%%%%%%%%%%%%%%%%%%%%%%%%%%%%%%%%%%%%%
\chapter{GUN C/C++ Programming}
\section{libstdc++}
\noindent\href{http://gcc.gnu.org/onlinedocs/libstdc++/manual/spine.html}{libstdc++ Manual} 官方手册。
其中有关于g++实现的许多细节问题,关于兼容性,ABI等的说明。
\noindent\href{http://gcc.gnu.org/onlinedocs/libstdc++/manual/configure.html}{Comiple g++} 官方指导。
对编译选项的说明。
\noindent\href{http://gcc.gnu.org/onlinedocs/libstdc++/manual/abi.html}{ABI} 官方关于ABI的说明。

\subsection{C++ ABI}
C++程序通常依赖于某些特定的运行时支持,如异常抛出和捕获。
当C++程序被编译为目标文件时,对象成员的对齐方式,mangling name,虚函数的实现等,都大相径庭,
这些细节定义于编译器的ABI Specification中(Application Binary Interface)。

%%%%%%%%%%%%%%%%%%%%%%%%%%%%%%%%%%%%%%%%%%%%%%%%%%%%%%%%%%%%%%%%%%%%%%%%%%%%%%%%
% freqenct api
%%%%%%%%%%%%%%%%%%%%%%%%%%%%%%%%%%%%%%%%%%%%%%%%%%%%%%%%%%%%%%%%%%%%%%%%%%%%%%%%
\chapter{Frequent API}
\section{Date and Time}
\subsection{Sleep}
milliseconds - 毫秒,1/1000 秒。microsenconds - 微秒, 1/1000 毫秒。
让UNIX程序休眠的系统调用有:sleep(), usleep(), nanosleep()。
其中sleep()的单位是秒,usleep()的单位是微秒,nanosleep的单位是纳秒。
UNIX上似乎没有专门以毫秒为单位的sleep函数。
UNP上还介绍了用select(), poll()实现sleep函数的方法,可以精确到微秒。
在man epoll上看到,epoll\_wait()的休眠单位是毫秒。

%%%%%%%%%%%%%%%%%%%%%%%%%%%%%%%%%%%%%%%%%%%%%%%%%%%%%%%%%%%%%%%%%%%%%%%%%%%%%%%%
% singal and error handle
%%%%%%%%%%%%%%%%%%%%%%%%%%%%%%%%%%%%%%%%%%%%%%%%%%%%%%%%%%%%%%%%%%%%%%%%%%%%%%%%
\section{Signal and Error Handle}
\blscmd{man 7 signal}查看信号说明,\blscmd{man 3 errno}查看错误说明。

\blscmd{SIGKILL}, \blscmd{SIGSTOP}不能被捕获,阻塞,或者忽略。

signal的信号值和对应解释:
\blitem{SIGABRT, 6, Core}  {Abort singal from abort(3).}
\blitem{SIGSEGV, 11, Core} {Invalid memory reference.}

\blitem{SIGKILL, 9, Term}   {Kill signal}
\blitem{SIGINT, 15, Term}   {Interrupt from keyboard, i.e. Ctrl-C。}
\blitem{SIGALRM, 14, Term}  {Timer singal from alarm(2).}

errno的错误码和对应解释:
\blitem{9}   {Bad file descriptor.}
\blitem{11}  {Resource temporarily unavailable}
\blitem{104} {Connection reset by peer.}
\blitem{113} {No route to host. 如果报这个错误,可能是对端的iptables设置,禁止了tcp连接端口。}
\blitem{115} {Operation now in progress. 如果是这个错误,可能是设置非阻塞socket,在connect()时返回此错误码。}
