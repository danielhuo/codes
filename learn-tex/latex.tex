\chapter{\LaTeX{} Notes}
\index{\LaTeX}

\section{Introduction}
\noindent\href{http://www.ctan.org}{CTAN} the Comprehensive TeX Archive Network\\
\href{http://www.ctex.org}{CTEX} 中文Tex\\
\href{http://www.ctan.org/tex-archive/info/lshort/}{lshort} Ctan上lshort各个语言的版本,包括lshort文档的源码。
这本小册子是一个德国人写的,英文名叫:The Not So Short Introduction to \LaTeXe,副标题是``112分钟学会\LaTeXe''。\\
\href{http://en.wikibooks.org/wiki/LaTeX/Packages} {wiki上的\LaTeX宏包列表。} \\
\href{http://blog.sina.com.cn/s/blog\_51e68f8d0100avil.html}{\LaTeX{} listings宏包备忘。}

\TeX{} 被称作一种typographic语言,即一种排版印刷语言。

如果往\TeX系统中添加了新文件,需要更新\TeX系统。
te\TeX和fpTeX的命令是:``texhash''。
Mik\TeX的命令是:``initexmf -\mbox{}-update-fndb''。

\section{\LaTeX{} Command}
\noindent $\backslash$stretch\{十进制小数\} 是一个弹性长度,其自然值是0pt,其伸展值等于十进制小数乘以$\backslash$fill。\\
\noindent $\backslash$texttt\{文本\} 等价于\{$\backslash$ttfamily 文本\},它使用当前系列与形状,
但是打印机(typewriter)族的字体来打印文字。\\
\noindent $\backslash$vspace\{高度\} 产生具有指定高度的竖直空白,当它位于页首或页尾时被忽略。\\
\noindent $\backslash$hspace$*$\{高度\} 产生具有指定高度的竖直空白,当它位于页首或页尾时也不被忽略。\\


\subsection{\LaTeX{} File Type}

\begin{tabular}{|l|lp{\textwidth}|}
tex     & \LaTeX源文件,用latex命令编译。   \\
sty     & \LaTeX宏包文件,可以使用$\backslash$usepackage命令将宏包文件引入到\LaTeX中。\\
dtx     & 文档化\TeX文件。是\LaTeX宏包文件的主要发布格式。\\
ins     & 对应*.dtx文件的安装文件。网上下载\LaTeX宏包,一般会包含一个*.dtx文件和一个ins文件。
            使用\LaTeX处理*.ins文件可以解开*.dtx文件。\\
cls     & 定义文档外观形式的类文件,可以通过$\backslash$documentclass命令选取。\\
fd      & 字体描述文件,告诉\LaTeX有关新字体的信息。\\
dvi     & 设备无关文件,*.tex文件编译的主要输出文件。\\
        & 可以使用DVI预览器预览,或使用dvips/dvipdf等程序输出PS或PDF文档。\\
toc     & 存储所有的章节标题,再次编译时将读取该文件生成目录。\\
lof     & 类似*.toc,用于生成图形目录。\\
lot     & 类似*.toc,用于生成表格目录。\\
aux     & 用于再次编译时传递辅助信息,如交叉引用的信息。\\
idx     & 文档中如果包含索引,\LaTeX可使用该文件存储所有的索引词条。此文件需用makeindex处理,\\
        & 详见位于\pageref{latex-tools-makeindex}页的第\ref{latex-tools-makeindex}节。\\
\end{tabular}

% end of Latex Introduction

\section{Basic Kownledge}

\subsection{\LaTeX{} Source}
\LaTeX的源代码包括:
\begin{itemize}
\item 需要排版的文本。
\item \LaTeX命令。
\end{itemize}

\subsubsection{特殊字符}
特殊字符包括:
\framebox{\# \$ \% \^{} \& \_ \{ \} \ {} } \par

反斜杠不能通过$\backslash$$\backslash$得到,可以使用命令\$$\backslash$backslash\$生成反斜杠。

显示空格符号通过命令\fbox{$\backslash$textvisiblespace}生成,如\textvisiblespace。

带圈的符号通过命令\fbox{$\backslash$textcircled\{letter\}}生成,如:
\textcircled{\scriptsize 1}
\textcircled{\scriptsize a}
\textcircled{\scriptsize c}
\textcircled{c}。

引号不能直接使用“。左引号由两个`(重音)产生,右引号由两个'(直立引号产生)。一个`和一个'产生一个单引号。

\subsubsection{空白距离}
空格和制表符等空白字符在\LaTeX中被视作相同的space,多个连续的空白被视作一个空白。
两行文本间的空白标志上下段,多个空白行等同一个空白行。

\subsubsection{\LaTeX命令}
\LaTeX命令有两种格式:
\begin{itemize}
\item 以反斜杠$\backslash$开始,命令只由字母组成。命令后的空格、数字或任何非字母的字符都标志命令的结束。
\item 由反斜杠$\backslash$和非字母的字符组成。
\end{itemize}
\LaTeX忽略命令之后的空格,因此如果需要在命令后得到空格,可以在命令后加\{\}和一个空格,
或加上一个特殊的空格命令。\{\}将阻止\LaTeX吃掉命令后的空格。
对比"\LaTeX without space"和"\LaTeX{} with space"。

命令的参数放在\{\}中,命令的可选参数放在[]中。

\subsubsection{注释}
注释放在\%后,等同C++中的//。符号\%也可以用来断开不能含有空白字符或换行符的较长输入内容。
如果注释的内容特别长,可以使用verbatim宏包提供的comment环境。comment环境中的注释不显示在文本中。

% end of Latex Source

\subsection{Environment}
\index{\LaTeX!Environment}

verbatim\index{\LaTeX!Environment!verbatim}可以将文本直接打印。
verbatim宏包可以更好的达到这个效果。\ref{latex-macro-package-verbatim}。

% end of "Latex Environment"

\subsection{Box}
\makebox[\textwidth]{central}\par
\makebox[\textwidth][s]{spread}\par
\makebox[\textwidth][r]{right}\par
\framebox[\textwidth]{这叫中规中矩}\par
\framebox[0.8\width][r]{从左边溢出了}\par
\framebox[1cm][l]{从右边溢出了}\par

命令\fbox{$\backslash$mbox\{text\}}保证把几个单词放在同一行上。
命令$\backslash$fbox和$\backslash$mbox类似,它还能围绕内容画一个框,如每个命令周边的框就是fbox的功劳。

% end of "Latex Box"

\section{Macro Package}
\index{\LaTeX!macro package}
常用的宏包:\\
\begin{tabular}{|l|l|}
\hline
宏包            & 说明 \\ \hline
makeidx         & 提供排版索引的命令。随\LaTeX提供。\\ \hline
verbatim        & 直接打印文本。\\ \hline
fancyvrb        & 美化的verbatim。\\ \hline
indentfirst     & 首行缩进。特别对中文有用。 \\ \hline
listings        & 用于在\LaTeX中引用源码。\\ \hline
xcolor          & 与listings搭配使用,源码语法高亮。 \\ \hline
beamer          & 创建演示文稿。\\\hline
\end{tabular}

\subsubsection{verbatim宏包}\index{\LaTeX!macro package!verbatim}\label{latex-macro-package-verbatim}
verbatim环境可以直接包含打印文本,而verbatim宏包则可以直接引入ASCII文本,命令
``$\backslash$verbatiminput\{filename\}''即可。
fancyvrb宏包是美化后的verbatim,它的输出和listings很相似。

\subsection{listings}\index{\LaTeX!macro package!listings}
\begin{quote}
众里寻它千百度。
\end{quote}

listings宏包用于\LaTeX中的源代码引用,效果如下:
\begin{lstlisting}
int main(int argc, char* argv[])
{
    printf("hello, world!\n");
    return 0;
}
\end{lstlisting}

listings可以和xcolor搭配使用,让源代码语法高亮。
listings的一个示例如下:
\begin{Verbatim}[frame=single, rulecolor=\color{green}]
\begin{lstlisting}[
    language={[ANSI]C},
    numbers=left,
    numberstyle=\tiny,
    keywordstyle=\color{blue!70},
    commentstyle=\color{red!50!green!50!blue!50},
    rulesepcolor=\color{red!20!green!20!blue!20},
    escapeinside='',
    xleftmargin=2em,
    xrightmargin=2em,
    frame=shadowbox,
    aboveskip=1em]
int main(int argc, char* argv[])
{
    /* comment */
    printf("hello, '中国'!\n");
}
\end{lstlisting}
\end{Verbatim}

以上,numbers表示在源码前加以行号,可以指定行号的位置,左或右;
numberstyle表示行号风格;之后是色彩的设置;因为litings不支持中文,
所以如果在源码中包含中文,需要“转义”,用escapeinside指定转义符,
以上用'单引号作为转义符;xleftmargin和xrightmargin指定了左右边距;
aboveskip指定了上边距;frame指定了源码边框,以上为shadowbox。

可以把listings的设置放在lstset\{\}中,
这样就不用每次使用listings时都设置各项参数。

\subsubsection{语言支持和语法高亮}
\paragraph{语言支持}
listings支持常见语言,如C(ANSI, Objective, Sharp), 
    C++(ANSI, GNU, ISO, Visual), Python, \TeX(\LaTeX), 
    make, XML, SQL,
    PHP, Perl, Ruby, Lisp, Java等。

\paragraph{语法高亮}
可以自由配置高亮单词,在lstset\{\}中添加如:
\begin{lstlisting}
emph={boost},emphstyle=\color{red},
emph={[2]vector,list,iterator},emphstyle={[2]\color{blue}}
\end{lstlisting}
这样的语句即可。

\paragraph{定制边框}
可以设置代码的边框,在$\backslash$begin\{lstlisting\}[]中添加frame设置。
frame的可选值为:none, leftline, topline, bottomline, lines, single, shadowbox。
lines表示在上下划线,single表示在四周划线,shadowbox表示一个阴影框。
frame的另一个表示方式是取\{trblTRBL\}的任意子集,其中大写字母表示双划线。

\subsection{beamer}\index{\LaTeX!macro package!beamer}
Beamer是Till Tantau在2003年创建的,用于演示文稿。帮助手册参考:
\verb|texmf\doc\latex\beamer\beameruserguide.pdf|。

% end of Macro Package

\newpage
\section{Large Document}
\subsection{Create Index}
\index{\LaTeX!index}
\begin{enumerate}
\item 引入宏包:"$\backslash$usepackage\{makeidx\}"。
\item 在导言中使用命令:"$\backslash$makeindex"激活索引命令。
\item 要索引的内容通过命令:"$\backslash$index\{key\}"指定,key是索引项的关键词。
\item 当\LaTeX处理源文件时,每个$\backslash$命令都会将适当的索引项和当前页面写入一个专门的文件*.idx中,因此需要使用外部命令
"makeindex\label{latex-tools-makeindex} fname.idx"来生成索引。
\item 在源文件的适当位置(一般位于文档最后),将索引文件引入源文件,使用命令"$\backslash$printindex"。
\end{enumerate}

\newpage
\section{Chinese}
\subsection{Chinese Bookmark}
在tex文档序言部分引入:
\begin{lstlisting}
\usepackage[pdftex,CJKbookmarks]{hyperref}
\end{lstlisting}

王垠提供了一段代码,也可以达到以上效果:
\begin{lstlisting}[language=TEX]
\usepackage[pdftex]{hyperref}
\pdfstringdefDisableCommands{
\let\CJK@XX\relax
\let\CJK@XXX\relax
\let\CJK@XXXp\relax
\let\CJK@XXXX\relax
\let\CJK@XXXXp\relax
}
\end{lstlisting}

生成中文书签的步骤:
\begin{lstlisting}
pdflatex cs.notes.tex
gbk2uni cs.notes.out
pdflatex cs.notes.tex
\end{lstlisting}

CTEX中带了gbk2uni这个小工具。

% end of Latex
