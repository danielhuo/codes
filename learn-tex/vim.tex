\chapter{Vim}
\index{Vim}
\noindent\href{http://vimcdoc.sourceforge.net/}{vimcdoc},是一个Vim帮助的中文文档。
\noindent\href{http://easwy.com/blog/}{Easwy's Blog},许多的Vim技巧与插件介绍。

%%%%%%%%%%%%%%%%%%%%%%%%%%%%%%%%%%%%%%%%%%%%%%%%%%%%%%%%%%%%%%%%%%%%%%%%%%%%%%%%
% Tricks
%%%%%%%%%%%%%%%%%%%%%%%%%%%%%%%%%%%%%%%%%%%%%%%%%%%%%%%%%%%%%%%%%%%%%%%%%%%%%%%%
\section{Tricks}
\blscmd{:r !date} 在当前行插入时间。

\blscmd{:vimgrep /pattern/[j][g] files}
在多个文件搜索。j将搜索结果输出到quickfix窗口。用\blscmd{:clist}, \blscmd{:cnext},\blscmd{:cprevious}查看。
如果files是一个正则表达式匹配某种类型的文件,使用如\blscmd{**/*.cpp}将匹配当前目录下的所有cpp文件,且是递归搜索。

\blscmd{:gegisters}查看所有的寄存器内容。所有vim寄存器以\blscmd{"}开头,如\blscmd{"x}。
常用者如选中文本,\blscmd{"x x}剪切,切换文件,再执行\blscmd{"x p}即可。

要Linux用\blscmd{Shift-K}可以查看man手册,但如果如printf,shift-K就将你带到了man(1) printf,不是你想要的C printf。
可以在Vim使用命令\blscmd{3 shift-K}即可。

转当前文件为HTML页面:\blscmd{:TOhtml}。

捕获Shell命令输出,定向到新Buffer:\blscmd{:split +enew \textbar r !ls}。
以上\textbar不是管道操作,而是命令分隔符。\blscmd{split \{+cmd\}}表示打开一个新窗口,
并在窗口中执行\blscmd{\{cmd\}}这个命令。:h中解释\blscmd{+\{cmd\}}是说打开一个窗口并执行cmd命令。

打开DOS格式的文本,通常可见行尾的\^{}M控制字符,实则为$\backslash$r。
可以用命令\blscmd{:\%s/$\backslash$r//g}清除之。

\blscmd{:cs reset}重置所有的cscope文件。
\blscmd{:se ignorecase}和\blscmd{:se noignorecase}搜索时大小写敏感设置。
\blscmd{:ls}和\blscmd{:buffers}查看缓冲区列表。
\blscmd{gv}选中上次用visual模式选择的文本块。

\blscmd{:argdo \%s/a/b/ge $|$ update}对多个文件执行替换。\blscmd{e}避免找不到匹配时输出错误信息。
与\blscmd{argdo}类似的命令有\blscmd{windo},\blscmd{bufdo}。不过执行bufdo需要小心,最好先用\blscmd{:ls}
或\blscmd{:buffers}查看缓冲区列表o


%%%%%%%%%%%%%%%%%%%%%%%%%%%%%%%%%%%%%%%%%%%%%%%%%%%%%%%%%%%%%%%%%%%%%%%%%%%%%%%%
% Tricks
%%%%%%%%%%%%%%%%%%%%%%%%%%%%%%%%%%%%%%%%%%%%%%%%%%%%%%%%%%%%%%%%%%%%%%%%%%%%%%%%
\section{Plugin}
\index{Vim!Plugin}

\subsection{Mini Plugin}host-to-host
\blitem{matrix.vim}{黑客帝国屏幕。命令:Matrix启动。如果打开标签,则:Matrix会报错说不能打开窗口。}
\blitem{ColorSchemeMenuMaker.vim}{用于快速浏览色彩机制,用命令:ColorSchemeExplorer启动。}

\blitem{mark.vim}{多种色彩,同时高亮多个word。默认命令$\backslash$m即可高亮,或取消高亮;
    $\backslash$r手工输入正则表达式,然后高亮匹配;$\backslash$n清楚光标所在高亮或所有高亮。
    高亮非常方便,不过在高亮间跳转十分麻烦。\href{http://www.vim.org/scripts/script.php?script_id=1238}{这里}
    有具体的说明。
}

\subsection{A}
\index{Vim!Plugin!A}
A插件的作用是在头文件和源文件间跳转。
在Vim中使用:A命令即可。文件的关联方式和搜索路径都可以在\~{}/.vimrc中设置。
设置文件关联方式:
\bllcmd{let g:alternateExtension\_CPP = "inc,h,H,HPP,hpp"}
设置搜索路径:
\bllcmd{let g:alternateSearchPath = "sfr:../include,sfr:../src"}
搜索路径的设置很重要,sfr:后跟的是相对“当前文件”的搜索路径。
如存在:./src/utility.cpp和./include/utility.h,则在编辑utility.h时使用:A,
想要跳转到./src/utility.cpp,则搜索路径是:../src/。

\section{Tricks}
\subsection{Search and Replace}
将一串数字,如:1234567890,替换为另一串数字,如:
0x12, 0x34, 0x56, 0x78, 0x90, 
使用命令:
\begin{lstlisting}
:s/\(\d\{2}\)/0x\1, /g
\end{lstlisting}

在选定行的每一行后面加一个反斜杠。
如在C/C++定义宏时可用:
\begin{lstlisting}
:s/$/\\/g
\end{lstlisting}
消除每行后的反斜杠则是:
\begin{lstlisting}
:s/\\$//g
\end{lstlisting}

\section{Vim Script} \label{language-vim-script}
