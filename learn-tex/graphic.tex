\part{Graphic Tools}
\index{Graphic Tools}

\chapter{Metapost}
\noindent\href{http://ect.bell-labs.com/who/hobby/index.shtml}{John D.Hobby} Metapost作者的主页。\\
\noindent\href{http://docs.huihoo.com/homepage/shredderyin/metapost.html}{王垠写的Metapost介绍。} 其中有大量的Metapost实例。\\

Metapost是一种用于绘图的语言,源自Knuth的Metafont语言。
在CTEX提供的软件包中附带有Metapost工具:mp/mpost。较新的系统都使用mpost命令。
Metapost文件通常以.mp为后缀,编译命令为\fbox{mpost fname.mp}
生成的图形文件为\fbox{fname.1}或以某个数字为后缀。这种图形文件就是eps文件。
用CTEX软件包中的GSview就可以查看esp, ps图形文件,也可以用GSview将它们转换为其他许多格式的图像文件,如jpeg等。

一个简单的Metapost源文件是:
\begin{lstlisting}
beginfig(1)
draw (20,20)--(0,0);
endfig;
\end{lstlisting}
就画了一条直线。Metapost的源文件,每行以``;''结尾;注释与\LaTeX注释相同,使用\%作注释。

\section{Grapviz-Dot}
\index{Graphic Tools!Dot}
\noindent\href{http://www.graphviz.org/}{Graphviz} Graph Visualization Software\\

Dot是Graphviz套件中的一个工具,用于绘图。Dot擅长绘制如数据结构、继承关系之类的复杂图形。
我所知的dot的一些应用,如Doxygen,KCachegrind等工具都依赖dot生成调用关系图。

Dot的命令非常简单:\fbox{dot -Tpng graph.dot -o graph.png}。
-T指明了输出图形的类型,Dot可以输出的图形类型包括:gif, png, svg, ps等。

定义图的代码是:
\begin{verbatim}
digraph G{
    main -> func;   /* comment */ // also comment
}
\end{verbatim}
每行代码都需要以``;''结束;用C/C++注释符号作注释,即/**/或//作注释。

在图中定义子图的代码是:
\begin{verbatim}
digraph G{
    subgraph cluster0{
        main;
    }

    subgraph cluster1{
        func;
    }

    main->func;
}
\end{verbatim}
子图必须以subgraph为关键字,且子图的名词必须以cluster打头。
不能将箭头指向子图,只能指向子图中的元素节点。
