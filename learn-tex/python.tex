\part{Python Programming Lanauage}
\chapter{Language Introduction} \label{language-python}
\noindent\href{http://www.woodpecker.org.cn/diveintopython/}{Dive into Python在线} Python啄木鸟在线。

\section{From C/C++}
Python中Bool关键字有\blscmd{True}, \blscmd{False}。

Python中没有switch语句,建议是使用\blscmd{dictionary}。

Python的for循环从根本上不同于C/C++的for语句,而类似foreach语句。

Python中只有一种浮点数,即float,表示机器一级的双精度浮点数。为了不增加语言的复杂度,Python里没有double这种数据类型。

Python中进行数值运算的模块是\blscmd{cmath},大概对应C/C++的\cppheader{math.h}。
cmath.pi, cmath.e定义了两个常量。而像正余弦计算,对数、自然对数、开方、指数等运算,均在这个模块提供。

Python中复数(Complex Number)是内建类型,通过\blscmd{z=complex(1.0,2.2)},或\blscmd{z=(1.0+2.2j)}可以定义复数。
通过\blscmd{z.real}和\blscmd{z.imag}访问复数的实部和虚部。

类似C++标准库的map,Python中有内建\blscmd{dict}。

Python中数值与字符串相互转换的函数是:\blscmd{int()},\blscmd{float()}、\blscmd{str()}。

Python中,\blscmd{type()}即可查看对象类型。

\section{Statement}
\subsection{Loop}

\paragraph{for-loop} for的基本形式是:\blscmd{for i in sequence}。

\section{Build-in}
\subsection{Build-in function}
\blitem{len(x)}{统计x的元素个数,x通常是序列(string, tuple, list)或字典等。}

\subsection{array}
array是内置数组,它的常用成员函数是:
\blitem{count(x)}{计算x的出现次数。}
\blitem{append(x)}{push back x.}
array没有如size()的成员函数,可以用内建函数\blscmd{len()}计算数组元素个数。
