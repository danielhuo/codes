\usepackage{CJK}                % 使用中文
\usepackage{makeidx}            % 生成索引
\usepackage{verbatim}           % 引入ASCII文本
\usepackage{indentfirst}        % 使段首也缩进
\usepackage[pdftex]{color,graphicx}     % 在文档中插入图形

\usepackage{tabularx}

\usepackage{listings}   % 引用程序
\usepackage{xcolor}     % 程序加以色彩

\usepackage{colortbl}   % color table, supoort \rowcolor , etc.

\usepackage{fancyvrb}   % fancy verbatim

\usepackage{fancyhdr}
\pagestyle{fancy}

% 以下是王垠提供的中文bookmark方案
% http://docs.huihoo.com/homepage/shredderyin/tex_frame.html
% 可以在目录中正常显示中文,但Acrobat左侧书签是乱码
% 欲使书签也能正常显示中文,用gbk2uni工具转换
\usepackage[pdftex,CJKbookmarks]{hyperref}   % 在PDF文档中使用超链接,下面有关于超链接的详细设置
%\pdfstringdefDisableCommands{
%\let\CJK@XX\relax
%\let\CJK@XXX\relax
%\let\CJK@XXXp\relax
%\let\CJK@XXXX\relax
%\let\CJK@XXXXp\relax
%}

%\usepackage[CJKbookmarks]{hyperref}

%\AtBeginDvi{\special{pdf:tounicode UTF8-UCS2}}

% ------------------------------------------------------------------------------

%\hypersetup{bookmarks=true,unicode=true,pdffitwindow=false,colorlinks,urlcolor=cyan,linkcolor=blue}
\hypersetup{bookmarks=true,unicode=true,pdffitwindow=false,colorlinks,urlcolor=blue,linkcolor=blue}

% ------------------------------------------------------------------------------

\lstset{
    %numbers=left,
    %numberstyle=\tiny,
    keywordstyle=\color{blue!70},
    commentstyle=\color{red!50!green!50!blue!50},
    rulesepcolor=\color{red!20!green!20!blue!20},
    escapeinside='',
    xleftmargin=0em,
    xrightmargin=0em,
    aboveskip=0.5em,
    %frame=single
    frame=lines,
    %emph={boost,vector,list,bitset,const\_iterator,find\_if, bind},emphstyle=\color{blue},
    %emph={[2]\_1},emphstyle={[2]\color{red}}
}

\renewcommand{\baselinestretch}{1.1}    % 设置行间距,设置为默认行间距的1.5倍

%================================================================================ 
% New Command
%================================================================================ 

\newcommand{\cppheader}[1]{\textless#1\textgreater}
\newcommand{\blkai}[1]{\CJKfamily{kai}#1} % 楷体
%\newcommand{\blscmd}[1]{\textcolor{blue!70}{\,#1\,}}    % short command
\newcommand{\blscmd}[1]{\emph{\bfseries{\,#1\,}}}    % short command
\newcommand{\blcomment}[1]{\blkai{\textcolor{blue!50}{#1}}\CJKfamily{song}}              % berlin's comment
\newcommand{\blcopy}[1]{\blkai{\textcolor{red!50!green!50!blue!50}{#1}}\CJKfamily{song}}              % berlin's copy
\newcommand{\bllinespace}{\par\addvspace{1ex plus 0.8ex minus 0.2ex}}

\newcommand{\bldate}[1]{\emph{\date{#1}}}

% item list, e.g.:
% -d/--debug
%     Print debug mode.
%     For development only.
\newcommand{\blitem}[2]{%
    \par\addvspace{1.5ex plus 0.8ex minus 0.2ex}%
    \noindent\textbf{#1}\hfill\\%
    %\par\addvspace{0.5ex plus 0.1ex minus 0.1ex}%
    \hphantom{MM}\parbox{\textwidth}{#2}%
    \par\addvspace{1.5ex plus 0.8ex minus 0.2ex}%
}

% long command
\newcommand{\bllcmd}[1]{%
    \par\addvspace{1.0ex plus 0.2ex minus 0.2ex}%
    \begin{tabular*}{\textwidth}{l@{\extracolsep\fill}}%
    \rowcolor{gray!20}%
    #1%
    \end{tabular*}%
    \par\addvspace{1.0ex plus 0.2ex minus 0.2ex}%
}

\newenvironment{blcommand}%
    {\nopagebreak\par\small\addvspace{3.2ex plus 0.8ex minus 0.2ex}%
     \vskip -\parskip
     \noindent%
     \begin{tabular}{|l|}\hline\rule{0pt}{1em}\ignorespaces}%
    {\\\hline\end{tabular}\par\nopagebreak\addvspace{3.2ex plus 0.8ex
        minus 0.2ex}%
     \vskip -\parskip}
