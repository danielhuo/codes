\part{C/C++ Programming Language} 
\index{C/C++} \label{language-c-cpp}

\chapter{Memory}
包括内存的分配管理,GC相关技术,内存越界,内存泄漏等技术与工具。
\section{Buffer Overflow Protection}\label{cpp-buffer-overflow-protection}
\noindent\href{http://en.wikipedia.org/wiki/Buffer_overflow_protection}{wiki上Buffer Overflow保护的介绍。}

内存溢出保护的两个知名实现是StackGuard和Stack-smashing Protection(SSP, or ProPolice)。

\chapter{C/C++ Miscellanea}
\noindent C语言的最佳参考是~K\&R,C++语言的最佳参考是TC++PL \cite{tcpl}。 \par
\noindent\href{http://www.cplusplus.com/}{C++ Resources Network},其中最常用的一个部分是
\href{http://www.cplusplus.com/reference/}{C++标准库的参考}。\par
\noindent\href{http://gcc.gnu.org/projects/cxx0x.html}{GCC对C++0x的支持列表}。 \par
\noindent\href{https://developer.mozilla.org/En/C\_\_\_Portability\_Guide}{Mozilla关于编写可移植C++代码的指导}。\par
\noindent\href{http://gcc.gnu.org/onlinedocs/cpp/Predefined-Macros.html}{GNU C/C++ Predefined Macros}

\chapter{GNU C/C++ Programming}
在Fedora 10, GCC 4.3.2的实现环境中,map的一个insert操作,需要3次拷贝构造,3次析构。

\section{Language and Library}

\blscmd{g++ (GCC) 3.4.6 20060404 (Red Hat 3.4.6-3)}不支持std::bitset的to\_string()函数。
编译时有如下输出:\bllcmd{error: no matching function for call to `std::bitset$<$8u$>$::to\_string()'}

C++中动态分配数组是未经初始化,勿忘初始化数组。

C/C++的预定义宏包括:\_\_FILE\_\_, \_\_LINE\_\_,
另外C99引入了\_\_func\_\_,这个宏是小写。在GNU中则早已具备一个名为\_\_FUNCTION\_\_的宏。
GNU的预定义宏扩展还包括:\_\_VERSION\_\_(GCC版本号)等。

\cppheader{string.h}中,除了memmove外,其他函数都没有定义重叠对象间的复制行为。strerror()也在\cppheader{string.h}头文件中。

\chapter{Boost Library}
\noindent\href{http://groups.google.com/group/boost\_doc\_translation}{中文Boost的GoogleGroup},主要致力于
翻译Boost文档。更新速度较快,目前有1.38的文档。

Boost官方网站导航\\
\begin{tabular}{ll}
\href{http://www.boost.org/users/history/}{历史列表}  & Boost库的所有历史版本在此。\\
\end{tabular}

\section{Boost Compile and Install}
\paragraph{Compile in Linux}
在Linux下编译Boost。

\paragraph{Static Link boost}
静态链接Boost库时,需注意Boost库的依赖,如:filesystem依赖system。
当静态链接失败时,可以用\blscmd{ldd}查看对应的动态库,检查是否有其他依赖,
若有,将依赖的静态库一并链接即可。

另一个需要注意是,静态链接boost,需要把生产的目标文件放前面,boost静态库放后面。
否则链接时gcc报出一大堆错误。

再一个需要注意是,如果静态链接boost::thread,那么一定要在编译时加上-lpthread参数。
否则运行时会有boost::thread的断言错误。

经gcc4.3.2编译的boost1.36,其子库program\_options在gcc3.4.6上静态链接失败。
因此需要在gcc3.4上重新编译。因link报错是说输入输出流中有函数未定义,因此怀疑是stdc++的缺陷。

\section{boost::thread}
Boost::thread的介绍\href{http://www.boost.org/doc/libs/1\_38\_0/doc/html/thread/changes.html}{文档}写道:
自1.34发行以来,Boost.Thread发生了翻天覆地的变化,这个库的几乎每行代码都发生了改变。

\section{boost::dynamic\_bitset}
boost::dynamic\_bitset相比std::bitset的好处在于,
dynamic\_bitset无需在编译时知道bitset的大小。
但dynamic\_bitset仍然是一个不完备的容器,例如它没有begin()/end()等函数返回的迭代器,
因此许多标准算法就不能用于dynamic\_bitset。

有几种方法可以将dynamic\_bitset转换为字符串:
\begin{itemize}
\item 用iostream/stringstream等输入输出流。
\item 用to\_string(dynamic\_bitset, string)函数,由boost::dynamic\_bitset库提供。
\item 用for循环,将dynmic\_bitset的每一位输出,用operator[]输出[0, size()]的每一位。
此种输出,与输出流的输出的bit流顺序正好相反。
\end{itemize}

\section{boost::string\_algo}
C++标准库的字符串是个巨库,但仍然没有如:icase\_compare(),
即忽略大小写的字符串比较等函数。遗憾的是,在Boost里也没发现这样一个函数。
Boost中倒是可以将字符串大小写转换,避免用C库的to\_uppper()和to\_lower()进行单个字符转换。

\begin{lstlisting}
#include <boost/algorithm/string.hpp>
string msg("Hello World\n");
void boost::to_upper(msg);
string boost::to_upper_copy(msg);
\end{lstlisting}
boost::to\_upper将修改传入的字符串,并返回void。如果不改变字符串且返回转换字符串,
则使用对应的\_copy函数。

\section{BOOST\_FOREACH}
BOOST\_FOREACH应该算是一个小巧实用的库,支持STL容器,内嵌数组等。
BOOST\_FOREACH库最早出现在boost 1.34.0。

\begin{lstlisting}
#include <boost/foreach.hpp>
list<Peer*> plist;
BOOST_FOREACH(Peer* peer, plist){
    peer->send_have();
}
\end{lstlisting}
在遇见小循环时,BOOST\_FOREACH似乎要比boost::bind和boost::lambda简单明了。

\section{boost::format}
Boost介绍format称它是printf-like的一个库。
boost::format是对printf和cout的一个很好补充。printf是类型不安全的,而且不能扩展,
而cout控制输出格式等方便又过于麻烦,boost::format则兼备两者之优,并独有所擅长。

\begin{lstlisting}
#include <boost/format.hpp>
boost::format("%s %3.2f%% %8d") %"info" %100.00 %1024;
\end{lstlisting}
以上即为熟悉的C风格。但boost::format还支持其他许多种的输出风格。

\chapter{log4cxx}
\noindent\href{http://logging.apache.org/log4cxx/index.html}{log4cxx Homepage} Apache项目组的log4cxx介绍。 \\
\noindent\href{http://www.dreamcubes.com/blog/?itemid=59}{log4cxx xml configure demo} 一个非官方的xml配置例子。
其中的xml配置文件已复制到了本地,因为它们太有用了。 \\
\noindent\href{http://expat.sourceforge.net/}{Expat库} log4cxx依赖这个XML解析库。\\

Apache log4cxx是一个C++的日志框架,是从log4j(j: Java)移植而来。
~log4cxx~使用APR(Apache Portable Runtime)。

如果调试人员不能接触代码,或是一个多线程大型的程序,使用日志是跟踪程序的好方法。
日志可以提供精确的context。日志也可以看作是很好的审核(auditing)工具。
日志的缺点则是会使程序变慢,而且太多的日志也会导致scrolling blindness。
而log4cxx则很好的解决了这些问题。可以通过修改log4cxx的配置文件,在不修改源码的基础上,控制日志行为。

我使用的log4cxx版本是:0.10.0,编译程序的链接库是:liblog4cxx.so.10。

\section{Compile log4cxx in Linux}
log4cxx的依赖库包括:apr, aprutil, expat库。


在Linux下编译log4cxx,发现编译时竟然有头文件未包含的错误。
例如使用memmove等函数,但没有include \cppheader{cstring},类似的低级错误,
而且至少有三四个文件有类似错误吧。

因为log4cxx依赖APR,因此下载apr库一起编译,\href{http://apr.apache.org/download.cgi}{这里}下载apr。

在Linux下静态链接log4cxx,编译通过的命令如下:
\begin{lstlisting}
$ g++ test.cpp liblog4cxx.a libapr-1.a libaprutil-1.a libapriconv-1.a libexpat.a -lpthread
\end{lstlisting}
编译后的执行文件约11MB。可见灰常复杂。而且,最惊讶是,以上编译通过竟然是瞎猫碰见了死耗子,
因为改变这几个静态库的顺序,就编译出错了。

除了以上库,还需要编译expat库,apr-util-1.3.4/xml/expat里有这个库。

\section{Use log4cxx}
log4cxx的三个重要概念是:Logger, Appender, Layout。
\begin{itemize}
\item Logger表示日志的层次结构。
\item Appender表示信息如何附加,如附加到文件或标准输出。
当日志有特殊要求,例如日志文档的大小限制,每日定时备份日志,定时将日志通过网络发送等,
就需要对Appender进行设置。
\item Layout表示日志输出的格式布局,如时间格式等。
\end{itemize}

在log4cxx的配置文件,或者程序源代码中,都可以设置以上的参数。
log4cxx的配置文件有几种格式,如XML和Java格式的配置文档。
而log4cxx的编程,也有许多细节需要注意。

\section{log4cxx xml configure}
至今还没有找到官方的关于log4cxx xml介绍的文章。目前的配置都是东摘西抄的:)
以下是一个简单的配置文件:

\noindent\begin{lstlisting}[language=XML, basewidth={0.45em, 0.25em}, fontadjust]
<?xml version="1.0" encoding="UTF-8" ?>
<log4j:configuration xmlns:log4j="http://jakarta.apache.org/log4j/">
    <appender name="testlog4cxx" class="org.apache.log4j.FileAppender">
        <param name="File" value="/home/bailing/fget/fget.log" />
        <param name="Append" value="true" />
        <layout class="org.apache.log4j.PatternLayout">
            <param name="ConversionPattern" value="%d{yyyy-MM-dd HH:mm:ss} (%F:%L) - %m%n"/>
        </layout>
    </appender>
    <root>
        <priority value="trace" />
        <appender-ref ref="testlog4cxx" />
    </root>
</log4j:configuration>
\end{lstlisting}

以上可见appender是配置文件的主体。其中指定了输出为文件,并且为附加方式。
在appender中有layout的设置,通常layout只有输出格式的设置。
root即是Logger,设置了日志的级别为trace,即跟踪方式,并将root的appender指定到上面的设置。

appender的class有多种,常用的有:

\begin{tabular}{|l|l|}
\hline
appender class                          & 说明 \\\hline
org.apache.log4j.FileAppender           & 文件方式输出。    \\\hline
org.apache.log4j.ConsoleAppender        & 终端输出。        \\\hline
org.apache.log4j.RollingFileAppender    & 每日备份日志。    \\
                                        & 设置文件大小,备份日志。  \\\hline
org.apache.log4j.net.SMTPAppender       & 邮件发送日志。    \\\hline
org.apache.log4j.odbc.ODBCAppender      & 将日志存入数据库。    \\\hline
org.apache.log4j.net.XMLSocketAppender  & 用XMLSocketAppender发送,用XMLSocketReceiver接收。 \\\hline
\end{tabular}

通常只用到输出文件和标准输出。

\section{log4cxx Programming}
一个最简单的log4cxx程序:
\begin{lstlisting}[language=C++]
#include <iostream>
#include <log4cxx/logger.h>
#include <log4cxx/xml/domconfigurator.h>

int main()
{
    setlocale(LC_ALL, "");
    log4cxx::xml::DOMConfigurator::configure("conf.xml");
    log4cxx::LoggerPtr logger = log4cxx::Logger::getRootLogger();
    LOG4CXX_DEBUG(logger, "hello, world");
    return 0;
}
\end{lstlisting}

编译需要动态链接log4cxx库,如:
\begin{lstlisting}
$ g++ main.cpp -L /usr/local/lib -llog4cxx
\end{lstlisting}

log4cxx预定义了几个宏,用于不同级别的日志输出:

\begin{lstlisting}[language=C++]
LOG4CXX_TRACE(log4cxx::LoggerPtr, msg);
LOG4CXX_DEBUG(log4cxx::LoggerPtr, msg);
LOG4CXX_INFO(log4cxx::LoggerPtr, msg);
LOG4CXX_ERROR(log4cxx::LoggerPtr, msg);
LOG4CXX_FATAL(log4cxx::LoggerPtr, msg);
\end{lstlisting}

其中msg可以视作stringstream,如下一般使用:
\begin{lstlisting}[language=C++]
LOG4CXX_ERROR(logger, "error : " << errno << strerror(errno));
\end{lstlisting}

