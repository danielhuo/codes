% berlin
% history:
% 2009/02/03    study LaTeX, add "LaTeX Note" to cs.notes

% ------------------------------------------------------------------------------

\documentclass[a4paper,11pt,oneside,openany]{book}

% ------------------------------------------------------------------------------

%-------------------------------------------------------------------------------- 
\documentclass[a4paper,11pt,oneside,openany]{book}

\usepackage{CJK}                % 使用中文
\usepackage{makeidx}            % 生成索引
\usepackage{verbatim}           % 引入ASCII文本
\usepackage{indentfirst}        % 使段首也缩进
\usepackage[pdftex]{color,graphicx}     % 在文档中插入图形

\usepackage{tabularx}

\usepackage{listings}   % 引用程序
\usepackage{xcolor}     % 程序加以色彩

\usepackage{colortbl}   % color table, supoort \rowcolor , etc.

\usepackage{fancyvrb}   % fancy verbatim

\usepackage{fancyhdr}
\pagestyle{fancy}

% 以下是王垠提供的中文bookmark方案
% http://docs.huihoo.com/homepage/shredderyin/tex_frame.html
% 可以在目录中正常显示中文,但Acrobat左侧书签是乱码
% 欲使书签也能正常显示中文,用gbk2uni工具转换
\usepackage[pdftex,CJKbookmarks]{hyperref}   % 在PDF文档中使用超链接,下面有关于超链接的详细设置
%\pdfstringdefDisableCommands{
%\let\CJK@XX\relax
%\let\CJK@XXX\relax
%\let\CJK@XXXp\relax
%\let\CJK@XXXX\relax
%\let\CJK@XXXXp\relax
%}

%\usepackage[CJKbookmarks]{hyperref}

%\AtBeginDvi{\special{pdf:tounicode UTF8-UCS2}}

% ------------------------------------------------------------------------------

%\hypersetup{bookmarks=true,unicode=true,pdffitwindow=false,colorlinks,urlcolor=cyan,linkcolor=blue}
\hypersetup{bookmarks=true,unicode=true,pdffitwindow=false,colorlinks,urlcolor=blue,linkcolor=blue}

% ------------------------------------------------------------------------------

\lstset{
    %numbers=left,
    %numberstyle=\tiny,
    keywordstyle=\color{blue!70},
    commentstyle=\color{red!50!green!50!blue!50},
    rulesepcolor=\color{red!20!green!20!blue!20},
    escapeinside='',
    xleftmargin=0em,
    xrightmargin=0em,
    aboveskip=0.5em,
    %frame=single
    frame=lines,
    %emph={boost,vector,list,bitset,const\_iterator,find\_if, bind},emphstyle=\color{blue},
    %emph={[2]\_1},emphstyle={[2]\color{red}}
}

\renewcommand{\baselinestretch}{1.1}    % 设置行间距,设置为默认行间距的1.5倍

%================================================================================ 
% New Command
%================================================================================ 

\newcommand{\cppheader}[1]{\textless#1\textgreater}
\newcommand{\blkai}[1]{\CJKfamily{kai}#1} % 楷体
%\newcommand{\blscmd}[1]{\textcolor{blue!70}{\,#1\,}}    % short command
\newcommand{\blscmd}[1]{\emph{\bfseries{\,#1\,}}}    % short command
\newcommand{\blcomment}[1]{\blkai{\textcolor{blue!50}{#1}}\CJKfamily{song}}              % berlin's comment
\newcommand{\blcopy}[1]{\blkai{\textcolor{red!50!green!50!blue!50}{#1}}\CJKfamily{song}}              % berlin's copy
\newcommand{\bllinespace}{\par\addvspace{1ex plus 0.8ex minus 0.2ex}}

\newcommand{\bldate}[1]{\emph{\date{#1}}}

% item list, e.g.:
% -d/--debug
%     Print debug mode.
%     For development only.
\newcommand{\blitem}[2]{%
    \par\addvspace{1.5ex plus 0.8ex minus 0.2ex}%
    \noindent\textbf{#1}\hfill\\%
    %\par\addvspace{0.5ex plus 0.1ex minus 0.1ex}%
    \hphantom{MM}\parbox{\textwidth}{#2}%
    \par\addvspace{1.5ex plus 0.8ex minus 0.2ex}%
}

% long command
\newcommand{\bllcmd}[1]{%
    \par\addvspace{1.0ex plus 0.2ex minus 0.2ex}%
    \begin{tabular*}{\textwidth}{l@{\extracolsep\fill}}%
    \rowcolor{gray!20}%
    #1%
    \end{tabular*}%
    \par\addvspace{1.0ex plus 0.2ex minus 0.2ex}%
}

\newenvironment{blcommand}%
    {\nopagebreak\par\small\addvspace{3.2ex plus 0.8ex minus 0.2ex}%
     \vskip -\parskip
     \noindent%
     \begin{tabular}{|l|}\hline\rule{0pt}{1em}\ignorespaces}%
    {\\\hline\end{tabular}\par\nopagebreak\addvspace{3.2ex plus 0.8ex
        minus 0.2ex}%
     \vskip -\parskip}


\begin{CJK*}{GBK}{song}
\title{\huge \bfseries Bldoc, Berlin's Document}
\end{CJK*}

\makeindex

\begin{document}
\begin{CJK*}{GBK}{song}

\maketitle

\clearpage      % 如果是双面打印,即openrigth,则使用\cleardoublepage
\addcontentsline{toc}{chapter}{Contents}    % 将目录本身加入到目录中

\tableofcontents

% 避免影响目录
%\setlength{\parindent}{0pt}                        % 将段首缩进设置为0
\setlength{\parskip}{1ex plus 0.5ex minus 0.2ex}    % 修改段落间距

%-------------------------------------------------------------------------------- 

%\part{Miscellanea Notes}

%%%%%%%%%%%%%%%%%%%%%%%%%%%%%%%%%%%%%%%%%%%%%%%%%%%%%%%%%%%%%%%%%%%%%%%%%%%%%%%%
% bldoc introduction
%%%%%%%%%%%%%%%%%%%%%%%%%%%%%%%%%%%%%%%%%%%%%%%%%%%%%%%%%%%%%%%%%%%%%%%%%%%%%%%%
\chapter{Introduction}
Bldoc主要是学习工作中的随记,大多来自HOWTO、man手册、各种软件自带的帮助文档,技术书籍,
以及网上看到的CS相关的技术文章或逸闻趣事,另外还有极少的个人感悟。
不论程序设计还是文档写作,一个诀窍是Copy and Plaster。
然而,俗话说吃亏就是占便宜,反过来讲,便宜不好占;因此本文绝大多数的内容,虽非原创,但不复制粘贴。
这样就可以避免使文档内容呈爆炸式的增长,也可以在键入每个字的时候,加深印象。
好记性不如烂笔头,书不厌读,识不厌记。

文档的内容主要是:
\begin{itemize}
\item CS随记,包括CS逸闻轶事逸言,CS读书笔记等。
\item C/C++程序设计,包括语言、库、调试技巧等。
\item UNIX程序设计,包括编译系统、系统编程、网络编程等。
\item UNIX环境,包括Bash、系统命令等。
\item UNIX网络,如iptable/netstate/tcpdump等各种网络工具的使用。
\item 网络协议,如TCP/IP/HTTP/FTP/BT等。
\item 数据结构和算法。
\item 硬件相关的笔记。
\item Python程序设计。
\item Vim编辑器。
\item 使用\LaTeX或者DocBook排版。
\end{itemize}

整个文档是在学习\LaTeX{}的过程中完成的。也许永不完成。因为只要还在这个行业,就不可能停止记录。
文档的质量良莠不齐,但我也并不追求完美,这份文档的初衷只是个人的一个学习笔记。

%%%%%%%%%%%%%%%%%%%%%%%%%%%%%%%%%%%%%%%%%%%%%%%%%%%%%%%%%%%%%%%%%%%%%%%%%%%%%%%%
% reading notes
%%%%%%%%%%%%%%%%%%%%%%%%%%%%%%%%%%%%%%%%%%%%%%%%%%%%%%%%%%%%%%%%%%%%%%%%%%%%%%%%
\chapter{Reading Notes}
\section{Master}
\blcomment{本节记录计算机领域的传奇式人物,他们的事迹,言论等。}

%%%%%%%%%%%%%%%%%%%%%%%%%%%%%%%%%%%%%%%%%%%%%%%%%%%%%%%%%%%%%%%%%%%%%%%%%%%%%%%%
% doouglas mcilroy
%%%%%%%%%%%%%%%%%%%%%%%%%%%%%%%%%%%%%%%%%%%%%%%%%%%%%%%%%%%%%%%%%%%%%%%%%%%%%%%%
\subsection{Douglas McIlroy}
\href{http://www.cs.dartmouth.edu/~doug/}{Doug McIlroy's Homepage}。
\href{http://www.cs.dartmouth.edu/~doug/biography}{Doug McIlroy's Biography}。
McIlroy发明了管道。是echo, spell, diff, sort, join, graph, speak, tr的作者。

Those types are note ``abstract'', they are as real as int and float.

但是,Linux给出``因为没安装对应的软件,所以打不开文件''这种Mac式诊断之时,
就是Linux不再是UNIX之日。

[McIlroy78]中揭露的UNIX哲学:
\begin{itemize}
\item 让每个程序就做好一件事。如果有新任务,就重新开始,不要往原程序中加入新功能而搞得复杂。
\item 假定每个程序的输出都会成为另一个程序的输入,哪怕那个程序还是未知的。
输出中不要有无关的信息干扰。避免使用严格的分栏格式和二进制输入。不要坚持使用交互式输入。
\item 尽可能早地将设计和编译的软件投入试用,哪怕是操作系统也不例外,理想情况下,
应该是在几星期内。对拙劣的代码别犹豫,扔掉重写。
\item 优先使用工具而不是拙劣的帮助来减轻编程任务的负担。工欲善其事,必先利其器。
\end{itemize}

\href{http://www.cs.dartmouth.edu/~sinclair/doug/?doug=mcilroy}{这里}有许多关于McIlroy
的笑话,摘录数条于下:
\begin{itemize}
\item Doug McIlroy doesn't make system calls. System calls call Doug McIlroy.
\item Doug McIlroy doesn't use malloc to allocate memory. He uses his bare hands.
\item Doug McIlroy doesn't debug. He stares at tty0 until it fixes the problem.
\item Alan Turing always wanted to win a McIlroy Award, but didn't qualify. No one has.
\item In 1984, the Department of Justice broke up AT\&T because they had a monopoly(垄断). On Doug McIlroy.
\item Doug McIlroy supervises the hypervisor.
\end{itemize}

%%%%%%%%%%%%%%%%%%%%%%%%%%%%%%%%%%%%%%%%%%%%%%%%%%%%%%%%%%%%%%%%%%%%%%%%%%%%%%%%
% c++ d&e
%%%%%%%%%%%%%%%%%%%%%%%%%%%%%%%%%%%%%%%%%%%%%%%%%%%%%%%%%%%%%%%%%%%%%%%%%%%%%%%%
\section{C++ D\&E}
\blcomment{The Design and Evolution of C++,昵称C++ D\&E,中文名《C++语言的设计和演化》,
是C++语言的创始人Bjarne Stroustrup的著作。}

尊重人群而不尊重人群中的个体实际上就是什么也不尊重。
C++的许多设计决策根源于我对强迫人按某种特定方式行事的极度厌恶。
在历史上,一些最坏的灾难就起因于理想主义者们试图强迫人们``做某些对他们最好的事情”。
这种理想主义不仅导致了对无辜受害者的伤害,也迷惑和腐化了施展权利的理想主义者们。

我不认为自己有权利把个人观点强加给别人。不同的人们确实会按不同的方式思考,
喜欢不同的方式做事情,对于这些情况的高度容忍和接受是我最愿意做的事情。

经常的情况是,如果一个人可以很容易地转变到``信仰”X,那么进一步转变到``信仰”Y也是很有可能的。
我喜欢怀疑论者而不是``真实的信徒”。我把一点点实在的证据看得比许多理论更有价值,
把实际经验结果看得比许多逻辑论述更重要。

我绝不想通过一种有局限性的程序设计语言定义去推行某种唯一的设计理念。

一个程序设计语言只是这个世界中微乎其微的一个部分,因此也不应该把它看得太重了。
要保持一种平衡的心态,特别重要的是应该维持自己的幽默感。

C++语言在1985年之后的演变,就说明了来自Ada(模板、异常、名字空间),Clu(异常),
以及ML(异常)的思想的影响。

如果地图与地表不符,要相信地表。 -瑞士军队格言

我认为原理这个词在一个真正科学原理非常贫乏的领域中显得过于自命不凡了,
而程序设计语言设计就是这样的一个领域。

你可以在任何语言里写出很坏的程序。

委员会的每个成员都需要学会尊重那些看起来是异己的观点,学会妥协。
这些倒是很符合C++的精神。

Samlltalk鼓励人们把继承看成是唯一的,或者至少是最基本的程序组织方式,
并鼓励人们把类组织到只有一个根的层次结构中。在C++里,类就是类型,并不是组织程序的唯一方式。
我也极端怀疑一种论断,说是需要强迫人们去采用面向对象的风格写程序。
除非你能把握住如何表现隐藏在数据抽象和面向对象的程序设计后面的原理,
否则你能做的不过是错误的使用支持这些概念的语言特征。

有多少天赋,也打不败细节的纠缠。 -古语

魔鬼隐藏在细节之中。    -古语

\section{The long tail}
作者:Chris Anderson

广播电视有一个了不起的地方:它可以用无可匹敌的效率将一个节目送到数百万人面前。
但是,相反的事情它却做不到--将数百万节目送到同一个人面前。而这一点正是互联网的强项。

21世纪经济学的秘密就藏在企业的服务器中。

19世纪意大利经济学家帕累托提出的80/20原则。

在传统零售经济学已经举步维艰的地方,网络零售经济学仍然能够高歌猛进。

最大的财富孕育自最小的销售。

这些“货架空间无穷无”的企业已经领悟了数学集合论的一个原理:
一个极大极大的数(长尾中的产品)乘以一个相对较小的数(每一种长尾产品的销量),仍然等于一个极大极大的数。而且,这个极大的数只会变的越来越大。

电磁波有一种无与伦比的威力:它可以毫无成本的向各个方向传播。

“饮水机效应”指的是办公室里围绕某个大众文化事件的热烈讨论。

下水道的最高排放量通常是在Super Bowl的中场休息时测量到的。

把魅力四射的年轻男人卖给年轻的女人。

一些根本性的东西已经在2000年改变了。

每一个热门都拥有数量虽少但却更加执着的拥?

年轻人不会等待某个神圣的数据来告诉他们什么东西是最重要的,他们想控制他们的媒体而不是被媒体控制。

他们给一件不可预见的事带来了一点点可预见性。

货架空间的分配就是一个零和游戏:一种产品取代另一种产品。

我们正在从一个大规模市场退回到利基市场,只不过,定义不同市场的不再是地理位置,而是我们的兴趣爱好。

在经济学中,搜索成本是指任何妨碍你寻找目标的东西。

其他消费者的行动往往是最有用的信号,因为他们的动机与我们最为统一。

现在,专业-业余写作也可以成就伟业。

面对茫茫太空,你唯有在绝对正确的时间去看绝对正确的方位才能观察到那些最有趣的新现象,
比如小行星或星体演化。这不是一台望远镜有多大,多贵的问题,而是在某一个特定时刻能有多少双眼睛盯着太空的问题。

如果有足够多的眼睛,所有bug都不在话下。

复印机率先揭穿了“媒体的利用永远属于拥有媒体的人”这句谎言。

业余者(amateur)这个词本身就来自拉丁语的爱人(amator)一词,动词是去爱(amare)。

正在成长的一代人目睹他们的同龄人制作出这样动人的创造性杰作,必然会被深深地触动。
某些才华横溢的人和某些了不起的设备共同造就了这些令我们神魂颠倒的艺术经典。
但是,一旦你了解了幕后的玄机,你就会意识到你也能成为这样的“天才”。

Jimmy Wales,最富有的期权交易商。维基百科全书没有标榜权威二字。
维基的真正非凡之处在于有机的治疗自己,自然选择那些必要的特征以避开本生态系统内的食肉动物和病原体的侵袭。

长尾中的事情有许多都不是以商业利益为目标的。在需求的头部和尾部,创造的动机截然不同。需求曲线开始于头部的传统货币经济,终结于尾部的非货币经济。

lulu.com - 一个新型的DIY出版商。

因为说完整个词太费时间,等三个字说完,你也就不再年轻了。

朋克摇滚的精神就是:没错,你有你的吉他,但你不一定要做正确的事!你可以做错!你是一个好音乐家一点意义也没有,唯一最要的是-你有话要说。

%%%%%%%%%%%%%%%%%%%%%%%%%%%%%%%%%%%%%%%%%%%%%%%%%%%%%%%%%%%%%%%%%%%%%%%%%%%%%%%%
% c++ coding standards
%%%%%%%%%%%%%%%%%%%%%%%%%%%%%%%%%%%%%%%%%%%%%%%%%%%%%%%%%%%%%%%%%%%%%%%%%%%%%%%%
\section{C++ Coding Standards}
C++ Coding Standards, 101 Rules, Guidelines, and Best Practices.\\
作者:Herb Sutter, Andrei Alexandrescu。\\
译者:刘基诚


小类只体现一个概念,承担一个责任。巨类,削弱封装性。

编译时隔离。

除非需要继承,否则不要忍受其弊端。

有些人不想生孩子。勿将独立类用作基类。

利器在手,勿再徒手为之。

策略应该上推,而实现应该下放。

健壮的设计就是能将修改限于局部的设计。新的需求不应该引起对已有代码的重新改写。
设计如果含有混合了实现细节的接口,就很可能会出现复杂的依赖网。

passthrough function, 通道函数,常见如get/set函数。
instrumentation: 度量性。

DIP: Dependency Inversion Priciple.

Law of Second Chances: 需要保证正确的最重要的东西是接口,其他所有东西以后都可以修改,
如果接口弄错了,可能再也不允许修改了。

Liskov: Substitution Priciple. 公用继承的目的是实现可替换性。

虽然派生类通常会增加更多的状态(即数据成员),但它们所建模的是其基类的子集而非超集。
派生类是更一般的基础概念的一个特例。

一个函数无法很好的履行两种职责。公用虚拟函数具有两个职责:指定了接口,指定了实现细节。
NVI: Nonvirtual Interface。

并非所有变化都是进步。

隐藏数据却又暴露句柄的做法是一种自欺欺人,就像你锁上了自己家的门,却把钥匙留在了锁里。

C++将私有成员指定为不可访问的,但没有指定为不可见的。
可访问性:是否能够被调用或者使用某种东西。可见性:是否能看到他从而依赖他的定义。
类的私有成员在成员函数和友元之外是不可访问的,但对整个世界,即所有看到类定义的代码而言,
都是可见的。使用Pimpl。但是,只有在弄清了增加间接层次确实有好处之后,才能添加复杂性,Pimpl也是一样。

\section{Effective C++ 3/e}
作者:Scott Meyers \\
译者:侯捷

一位女性若非怀孕,就是没有怀孕。不可能说她部分怀孕。同样道理,一个软件系统要不就具备异常安全性,
要不就全然否决,没有所谓的局部异常安全系统。

pimpl idiom. copy-and-swap.

RCSP, Reference-Counting Smart Pointer, 但RCSP无法打破循环引用。\ shared\_ptr 就是RCSP。

shared\_ptr有一个特别好的性质:它会自动使用它的每个指针专属的deleter,因而消除另一个潜在的客户错误,
即所谓的cross-DLL problem。问题发生于对象在某一DLL中被new创建,却在另一个DLL内被delete。
许多平台上,这类跨DLL之new/delete成对应用会导致运行期错误。




%%%%%%%%%%%%%%%%%%%%%%%%%%%%%%%%%%%%%%%%%%%%%%%%%%%%%%%%%%%%%%%%%%%%%%%%%%%%%%%%
% working notes
%%%%%%%%%%%%%%%%%%%%%%%%%%%%%%%%%%%%%%%%%%%%%%%%%%%%%%%%%%%%%%%%%%%%%%%%%%%%%%%%
\chapter{Working Notes}
\section{Working In Funshion}
\bldate{2009/07/09} 阅读HS源码时看到的hash算法,有来自UNIX system V的ELFhash,
来自阎宏飞、谢正茂在天网搜索引擎中使用的hash算法,另有一种是谢正茂的hash算法。
下载了个一致性hash(consistent hash)的实现,据说是memcache client的一个实现,
由ketama实现。

\bldate{2009/07/08} BS地址service-bs.funshion.com

\bldate{2009/07/08} fget公网运行的一个问题。fget单独运行正常,被MS调用则出错。
因为MS fork()/exec()之前,已有大量打开的文件描述(数千个),fget继承父进程打开的文件描述符,
因此socket()创建的sockfd的值>1024,用非阻塞connect()后调用select()判断sockfd是否可写,
而select()系统限制为1024,因此select这里即出错。解决方案是在fget启动时,关闭文件描述符。

\bldate{2009/06/19} web查看homeserver信息:admin.funshion.com

\bldate{2009/06/19} 测试了socket选项SO\_REUSEADDR。在Fedora9上测试,在一个进程内部,
不能对同一个端口有两次bind()调用,也不能有两个进程同时绑定到同一个端口。
在CentOS 5.2上测试,一个进程内部,两个socket可以bind到同一个端口,
但两个进程不能同时bind到同一个端口。
man 7 socket上关于SO\_REUSEADDR的说明中有提及不能bind到同一个端口。
在Windows上,一个进程内部可以多次bind同一端口,也可两个进程bind同一端口,只是结果是未知的。

\bldate{2009/06/17} 代码审查时发现的问题。send失败,可休眠1秒再次尝试。
accept如果需要限制被动连接数,可先accept,再close。

\bldate{2009/06/17} 调试发现的问题。对本的机器做了对等端口映射(192.168.16.252:9501映射公网端口9501)。
fget接受被动连接。发现有被动连接进来的peer发送bitfield的长度和本地任务不匹配。
但只有被动连接的peer有此问题,主动连接则无。缘故是:在tracker上announce后,
客户端可能已存储fget信息,当fget切换另一任务下载时,客户端可能连接fget并交互旧任务的bitfield。
解决方法是在handshake时,判断任务hash,如果不是当前任务hash,则中断连接。

\bldate{2009/06/15} 调试是发现,两个函数调用之间参数被改变,如foo(par) \{ boo(par); \},
函数声明foo, boo的参数都为unsigned int,且par的类型也为unsigned int,但调试时仍发现,
在foo(par)时par值为VAL1(10485760),在boo(par)时par值为VAL2(4096)。但重新编译后,再次运行,无此问题。
非常诡异。

\bldate{2009/06/03} 此前以往八月,皆属无稽之谈。

\bldate{2009/05/31} 连接公网peer,连接速度很慢,不容易连接上。因此如果连接成功,则不要轻易断开。
另一则是利用被动连接。

\bldate{2009/04/27} 最近希望看一些高端服务器设计方面的论文或者书籍。不过资料似乎不多。
从C10K等看起,但貌似资料有些旧了,都是好些年前。看TAoUP,上面对线程编程评价似乎很低,
在High-Performance-Server-Design上也看到关于线程的代价评论,在另一篇Why-Threads-Are-Bad-Idea里也有
把线程与event编程模型的对比。继续深入学习。

\bldate{2009/04/23} fget项目中做了一个修改。将utility.h修改为fget\_utility.h,然后单独建立一个utility目录,
其中又有一个utility.h文件,并在fget\_utility.h中包含utility.h。经过一些简单的试验,
如gcc -E a.cpp输出没有包含utility.h中声明的函数,在gcc -E b.cpp的输出却包含utlity.h中声明的函数。
比较奇怪的问题。查明结果是在把原始的utility.h改为fget\_utility.h时,忘了修改头文件的宏定义,即\#ifndef \_UTILITY\_H。
再次说明一个问题,切勿冒冒然修改程序,想清楚前后关联影响,再修改。

\bldate{2009/04/23} 当机器有多个网口时,一般eth0配置为外网地址,eth1配置为内网地址。当bind时,
如果不指定地址,用INADDR\_ANY,则由内核选择一个网口。如果bind到指定网口,可能路由时不能抵达目标地址(未验证)。

\bldate{2009/04/07} select()返回-1,表示有连接坏掉了,此时也可能受到了SIGPIEPE信号。此时sockfd $>$ 0还成立,
因此要把这个坏掉的连接抓出来,有几种方法。1,用select依次调用每个sockfd,如果返回-1,则此连接已坏。
2,调用一些socket函数,如果返回错误,则此连接坏掉。注意如果调用send(), recv()等函数,可能另有问题,
如,sockfd是非阻塞的,但连接已坏,此时可能被send, recv阻塞。这种情况下,可以尝试调用getsockopt等函数判断。
另外,select()返回-1,如果errno为EINTR,则此错误可以忽略。

\bldate{2009/03/31} 测试fget时,如果连接公网Tracker,则fget也需要有公网ip和port,否则fget之间不能互联。
对于非阻塞socket的connect()调用,在返回EINPROGRESS后,select()的等待时间可为4秒。

\bldate{2009/03/26} 今天得知运营是用PRTG的工具监视流量的。当然是UI是web页面。

\bldate{2009/03/12} 系统调用的错误处理。在send/write后,应该仔细检查返回值是否等于发送缓冲的长度。
如果缓冲区已满,则应保留未发送的数据,等待下一次send。

\bldate{2009/03/10} 公网调试MediaServer时发现的问题。kill了mediaserver,但很快它自己又运行了。
注意检查deamon。如果有deamon,先killdeamon。另外,用valgrind检查MediaServer内存时,发现没有MediaServer
吐不出流量,很奇怪。

\bldate{2009/03/10} 检查函数返回值时的低级错误。在网络编程时,如果是非阻塞socket,
则send/recv返回-1,且errno是EWOULDBLOCK(或者EAGAIN),则应该忽略这个错误。
但注意与EWOULDBLOCK比较的对象是errno,而非send/recv函数的返回值(-1)。

\bldate{2009/02/26} 网络速度的快慢。但发现自己写的网络程序速度慢,要主要检查两个方面,
一是send速度,一是request速度。如BT程序,如果Client请求发的频率很低(请求慢),
则Server发送的数据自然就少,反应到Monitor上,就表象为速度慢。
而且网络程序,如果发现问题,需要检查连接两端,先确保一端没有问题,再调试另一端。
所以发现“速度慢”之类的问题,不要忽略了请求速度的控制。

\bldate{2009/02/26} TCP发送窗口的问题。调试时发现一条TCP连接速度较快,
但用抓包器观察,连接双方的TCP窗口却很小,特别是TCP三次握手期间,第1和第2次握手交换的窗口还较为合理,
但第3次握手的窗口就非常低。这个问题纠缠了数天,后来查明原因是:
TCP最初设计窗口大小为16bit的值,即最大为65535字节;后来对TCP协议的扩展中,增加了一个
window scale,是16bit,可用其中的[0,14],wscale在SYN报文(SYN报文在3次握手的前两次中出现,
第3次是ACK报文)中声明,之后发的窗口,都要经过这个wscale的计算。如wsacle为7,window为56,
则实际的window大小为:$56 * 2^{7}$,即7168。

\bldate{2009/02/24} 调试程序最怕遇见的时随机出现的问题。memcpy时常导致core dump,
但查不出究竟。在memcpy前做了非常详细的参数检查,许多的assert。后来发现这样做其实有些无济于事,
因为assert在这里不能找出问题根源(或者应该用异常?)。数小时的努力后,
放弃调试,开始源码阅读(write by myself -\_-!!),数分钟后发现了问题所在。
精简的代码逻辑如下:
\noindent\begin{lstlisting}[language=C++]
pointer* p = arr_have_long_name.find();
for(int i = 0; i < arr_have_long_name.size(); ++i){
    // do some check, but unfortunate, p is changed
    p = arr_have_long_name[i];
    check(p);
}
return p;
\end{lstlisting}
教训是,最经济的排错方法是仔细阅读源码,而非利用各种工具或者不适用的技巧。

\bldate{2009/02/10} 用GDB调试的一个问题。编译时加-g参数,编译生成a.out。
在serv-a上编译,在serv-b上调试。在GDB中用edit查看源文件,发现源文件不同;
用系统命令md5sum计算serv-a/a.out和serv-b/a.out,发现两个的MD5值却完全相同。
导致此问题的原因是:-g参数并不将源代码嵌入可执行程序,而只是在可执行程序中
嵌入调试信息,与源代码关联起来。因此在serv-b上看到程序源码,完全出于巧合,
即假设在serv-a的/home/bailing/编译./hello.cpp,而在serv-b的/home/bailing/目录下,
恰好存在一个名为hello.cpp的文件,因此被GDB发现;如果没有hello.cpp,
则GDB会抱怨说找不到源码。

%%%%%%%%%%%%%%%%%%%%%%%%%%%%%%%%%%%%%%%%%%%%%%%%%%%%%%%%%%%%%%%%%%%%%%%%%%%%%%%%
% other notes
%%%%%%%%%%%%%%%%%%%%%%%%%%%%%%%%%%%%%%%%%%%%%%%%%%%%%%%%%%%%%%%%%%%%%%%%%%%%%%%%
\chapter{Other Notes}
\noindent\href{http://lamp.linux.gov.cn/jinbuguo\_florilegium.html}{金步国的主页} 
严谨的翻译和原创作品,实用。\\
\noindent\href{http://www.ringkee.com/}{肥肥世家}
丰富的学习笔记。\\
\noindent\href{http://docs.huihoo.com/homepage/shredderyin/}{王垠的个人主页}
\LaTeX{} 等介绍。\\
\noindent\href{blog.csdn.net/haoel}{陈皓} CSDN博客。Makefile、GDB调试等文章。\\
\noindent\href{http://blog.youxu.info/}{4G Spaces} 计算机八卦等。\\
\noindent\href{http://blog.csdn.net/g9yuayon/}{袁泳 \textbar{} 负暄琐话} \\
\noindent\href{http://blog.csdn.net/pongba/}{刘未鹏 \textbar{} C++的罗浮宫} \\
\noindent\href{http://mindhacks.cn/}{刘未鹏 \textbar{} Mind Hacks} \\

\bldate{2009/02/24} 搜索Linux下的网络监测工具很是费劲,最后发现的一个较好的搜索词为:
Linux network traffic monitor tool。

listings\footnote{\LaTeX{} 的一个宏包,用它排版计算机程序。}的手册介绍了一个让\TeX{} crash的技巧。
在\LaTeX{} 中引入以下源码:
\begin{lstlisting}
\lstdefinestyle{crash}{style=crash}
\lstset{style=crash}
\end{lstlisting}
然后手册的作者说:Only bad boys use such recursive calls, 
but only good girls use this package.

逸闻趣事,从《计算机网络4/e》8.3.2节上看到的:
第一个公开密钥算法是背包算法(MerKle and Hellman,1978),
算法的发明者Ralph Merkle对自己的算法非常自信,因此他悬赏100美金给破解算法的人。
Adi Shamir(即RSA中的S)迅速的破解了算法,领取了奖金。
但Merkle并不气馁,又加强算法,并悬赏1000美金给破解算法的人。
这次Ronald Rivest(即RSA中的R)也迅速地破解了该算法,并领取了奖金。
Merkle不敢再为下一个版本悬赏10,000美金了,所以“A”(Leonard Adleman)很是不幸,无法领取奖金了。
背包算法不再被认为是安全的,也没在实践中使用。

\href{http://developers.solidot.org/article.pl?sid=09/03/24/0859257}{这里}看到的关于Vi和Emacs的趣闻。
说是海盗用Emacs,忍者用Vi。总所周知Emacs拥有强大无比的定制和扩展能力,而海盗也无时不在定制化他们的趁手工具,
Emacs确实比Vi慢,但海盗并不在意,因为他们经常喝的醉醺醺的。
英文原文在\href{http://philosecurity.org/2009/03/23/pirates-and-ninjas-emacs-or-vi}{这里}。

John W. Backus: You need the willingness to fail all the time\ldots{} You have to generate
many ideas and then you have to work very hard only to discover that they don't work.
And you keep doing that over and over until you find one that does work.

Edmund Burke: All that is needed for the triumph of misguided cause is that good people do nothing.
[谬误想要获得胜利,只需好人袖手旁观。]

BBN had a big contract to implement TCP/IP, but their stuff didn't work, and Joy's grad student
stuff worked. So they had this big meeting and this grad student in a T-shirt shows up, and they said,
"How did you do this?" And Bill said, ``It's very simple, you read the protocol and write the code."

Don says that he chose the term in hopes of making the originators of the term ``structured programming"
feel as guilty when they write illiterate programs as he is made to feel when he writes unstructured programs.

Who do you think is the best coder, and why?

Leonardo Da Vinci是``从Vinci来的Leonardo"之意。

John Carmack, Castle Wolfstein, doom, doomII, Quake的作者。其简历说自己的专长是``Exhaust 3-D technology"。

C.A.R.Hoare: Premature optiomization is the root of all evil.[提前优化是万恶之源。]

只有向后看才能理解生活,但是要生活好,则必须向前看。 -SREN AABYE KIERKEGAARD 日记(1843)

%%%%%%%%%%%%%%%%%%%%%%%%%%%%%%%%%%%%%%%%%%%%%%%%%%%%%%%%%%%%%%%%%%%%%%%%%%%%%%%%
% temp notes
%%%%%%%%%%%%%%%%%%%%%%%%%%%%%%%%%%%%%%%%%%%%%%%%%%%%%%%%%%%%%%%%%%%%%%%%%%%%%%%%
\chapter{Temp Notes}
\blcomment{
本章为网络浏览时随手之笔记,大多为外文资料随译之笔,并未修饰言辞。
且随意放置,未加管束。其中所记,多未加实践实验,权当开阔眼界。
待时日既久,材料既殷,再分门别类。}

Linux内核源码在如\blscmd{/usr/src/kernels/2.6.18-92.el5-x86\_64}的目录,但似乎其中没有源文件。

DCCP, Datagram Congestion Control Protocol,数据报拥塞控制协议。
\href{http://www.linuxfoundation.org/en/Net:DCCP}{这里}有比较全的介绍。
dccp在Linux内核的\blscmd{net/dccp}这个目录。DCCP是一个传输层协议,
在RFC 4340-4342中有其说明。

CDN, Content Delivery(or Distribution) Network,
有两种类型的content,下载和流。下载类的如页面访问和视频。流如在线视频。
应该是服务器均衡负载的一种实现网络。wiki上介绍说,策略性的放置服务器,
可以获取吞吐量大于backbone主干线的能力。CND减少了洲际(interconnects)的,
public peers, private peers和bacbones的负担,降低了delivery的开销。
避免了负载于backbone或peer link的压力,而将之重定向且分摊于edge servers。

TCP因丢包和时延而受损,而CDN置服务器于edge networks,并使用户易于访问。
某种CDN,如高速web页面的cache。当靠近server的client访问web时,此server查询cache
中是否存在此页面,若有直接返回用户;若无,则从其他server获取,并cache之。

CRT(C Runtime Library)。MSDN上看到的,VC编译时的C运行时库,有多种选项,其说明如下:\\
\begin{tabular}{|l|l|l|l|}\hline
编译选项    & 关联的DLL     & C运行时库         & 说明  \\\hline
/MT         & 无。          & libcmt.lib        & 多线程,静态链接 \\\hline
/MD         & msvcr80.dll   & msvcrt.lib        & 多线程,动态链接 \\\hline
/MTd        & 无。          & libcmtd.lib       & 多线程,静态链接,调试版本 \\\hline
/MDd        & msvcr80d.dll  & msvcrtd.lib       & 多线程,动态链接,调试版本 \\\hline
\end{tabular}

之前有单线程的C运行时库(libc.lib, libcd.lib,用/ML, /MLd开启),但现在已不支持。
对于C++运行时库,把以上libc都变为libcp即是。编译选项相同。

wiki上看到说lihttpd可以承受每秒1000次的访问。在Lighttpd的主页上见其1.5版使用aio,
据称效率有极大的提高。

用\blscmd{F-Secure SSH Client Trial}在Windows和Linux间通过SSH传输文件。

用\blscmd{iftop}可以查看网络流量,比\blscmd{vnstat}更为细致的输出。不过似乎有些字体问题。

\blscmd{COW, Copy-On-Write}技术用在\blscmd{fork()}上,当父进程调用fork()系统调用后,
并不直接将父进程的页面复制到子进程,而是两者共享地址空间。只有当其中一方对页面执行写操作时,
才复制其副本给修改者。

\blscmd{vfork}创建新进程后,父进程挂起,指导子进程执行完成退出,或执行\blscmd{execve()}系列函数。

\blscmd{UPnP}(Universal Plug and Play),简单的理解可以是:自动端口映射。这样内网的端口,
即可在通过NAT时做自动端口映射,即对公网开放了这个端口。要使用UPnP,需要Modem,OS和软件的支持。
在Windows中开启UPnP需要以下设置:
\begin{itemize}
\item 控制面板,添加删除程序,添加/删除Windows组件,网络服务,UPnP用户界面。
\item Windows防火墙,例外选项中选中UPnP框架。以上两步,可以通过:网上邻居,
显示联网的UPnP设备的图标,一次完成。
\item 控制面板,管理工具,服务,启动:SSDP Discovery Service和Universal Plug and Play Device Host
两项服务。
\end{itemize}

\href{http://www.usenix.org/about/flame.html}{USENIX frame award}。

斯德哥尔摩综合症又称为人质情绪、人质综合症,是指犯罪的被害者对于犯罪者产生情感,
甚至反过来帮助犯罪者的一种情绪。

浏览DHT时看到的外部链接,一个名为memcached的系统,用在加速需要与数据库交互动态页面访问场合,
被youtube, twitter等网站使用。\href{http://www.danga.com/memcached/}{这里}。

\href{http://www.xmlbar.com/}{稞麦网}提供下载视频网站的工具。如下载优酷视频。

\href{http://www.devtopics.com/101-great-computer-programming-quotes/}{101条计算机妙语}。
\href{http://news.csdn.net/a/20090522/211469.html}{中文版}。

ip转发,查看:
\bllcmd{cat /proc/sys/net/ipv4/ip\_forward}
也可修改\blscmd{/etc/sys}中的\blscmd{net.ipv4/ip\_forward}值。

\href{http://www.planet-lab.org/}{planet-lab}计算机网络实验。

安装\blscmd{wordpress},用作本地博客系统。依赖PHP, MySQL。
单独安装PHP, MySQL, Apache组件较为繁琐,可用\href{http://www.apachefriends.org/en/xampp.html}{xampp}软件。
此软件携带以上程序。安装xamp后,设定Apache监听端口,编辑\blscmd{/apahche/conf/httpd.conf},
如果默认80端口被占用,可改用8080等其他端口。再次启动Apache。用浏览器访问
\blscmd{http://localhost:8080/phpmyadmin},选择左侧之mysql,继而可查看user。默认有root用户,且密码为空。

MySQL的默认名为“localhost”,在wordpress的解压目录中,修改wp-config-sample.php,填入数据库信息。
其中已有注释提示,修改数据库名为wordpress,填入用户名(root),密码(),MySQL的名称(localhost)。
修改wp-config-sample.php为wp-config.php,并将整个wordpress目录放在\blscmd{/xampp/htdocs/}下。
或者如果单独安装apache,则放在\blscmd{/apache/htdocs}目录。
注意,要现在MySQL中建立wordpress数据库,不用见表等操作,但需要先建立此库。

\href{http://gigamonkeys.com/book/}{Practical Common Lisp}。



if browser were woman.



%\part{C/C++ Programming Language} 
\index{C/C++} \label{language-c-cpp}

\chapter{Memory}
包括内存的分配管理,GC相关技术,内存越界,内存泄漏等技术与工具。
\section{Buffer Overflow Protection}\label{cpp-buffer-overflow-protection}
\noindent\href{http://en.wikipedia.org/wiki/Buffer_overflow_protection}{wiki上Buffer Overflow保护的介绍。}

内存溢出保护的两个知名实现是StackGuard和Stack-smashing Protection(SSP, or ProPolice)。

\chapter{C/C++ Miscellanea}
\noindent C语言的最佳参考是~K\&R,C++语言的最佳参考是TC++PL \cite{tcpl}。 \par
\noindent\href{http://www.cplusplus.com/}{C++ Resources Network},其中最常用的一个部分是
\href{http://www.cplusplus.com/reference/}{C++标准库的参考}。\par
\noindent\href{http://gcc.gnu.org/projects/cxx0x.html}{GCC对C++0x的支持列表}。 \par
\noindent\href{https://developer.mozilla.org/En/C\_\_\_Portability\_Guide}{Mozilla关于编写可移植C++代码的指导}。\par
\noindent\href{http://gcc.gnu.org/onlinedocs/cpp/Predefined-Macros.html}{GNU C/C++ Predefined Macros}

\chapter{GNU C/C++ Programming}
在Fedora 10, GCC 4.3.2的实现环境中,map的一个insert操作,需要3次拷贝构造,3次析构。

\section{Language and Library}

\blscmd{g++ (GCC) 3.4.6 20060404 (Red Hat 3.4.6-3)}不支持std::bitset的to\_string()函数。
编译时有如下输出:\bllcmd{error: no matching function for call to `std::bitset$<$8u$>$::to\_string()'}

C++中动态分配数组是未经初始化,勿忘初始化数组。

C/C++的预定义宏包括:\_\_FILE\_\_, \_\_LINE\_\_,
另外C99引入了\_\_func\_\_,这个宏是小写。在GNU中则早已具备一个名为\_\_FUNCTION\_\_的宏。
GNU的预定义宏扩展还包括:\_\_VERSION\_\_(GCC版本号)等。

\cppheader{string.h}中,除了memmove外,其他函数都没有定义重叠对象间的复制行为。strerror()也在\cppheader{string.h}头文件中。

\chapter{Boost Library}
\noindent\href{http://groups.google.com/group/boost\_doc\_translation}{中文Boost的GoogleGroup},主要致力于
翻译Boost文档。更新速度较快,目前有1.38的文档。

Boost官方网站导航\\
\begin{tabular}{ll}
\href{http://www.boost.org/users/history/}{历史列表}  & Boost库的所有历史版本在此。\\
\end{tabular}

\section{Boost Compile and Install}
\paragraph{Compile in Linux}
在Linux下编译Boost。

\paragraph{Static Link boost}
静态链接Boost库时,需注意Boost库的依赖,如:filesystem依赖system。
当静态链接失败时,可以用\blscmd{ldd}查看对应的动态库,检查是否有其他依赖,
若有,将依赖的静态库一并链接即可。

另一个需要注意是,静态链接boost,需要把生产的目标文件放前面,boost静态库放后面。
否则链接时gcc报出一大堆错误。

再一个需要注意是,如果静态链接boost::thread,那么一定要在编译时加上-lpthread参数。
否则运行时会有boost::thread的断言错误。

经gcc4.3.2编译的boost1.36,其子库program\_options在gcc3.4.6上静态链接失败。
因此需要在gcc3.4上重新编译。因link报错是说输入输出流中有函数未定义,因此怀疑是stdc++的缺陷。

\section{boost::thread}
Boost::thread的介绍\href{http://www.boost.org/doc/libs/1\_38\_0/doc/html/thread/changes.html}{文档}写道:
自1.34发行以来,Boost.Thread发生了翻天覆地的变化,这个库的几乎每行代码都发生了改变。

\section{boost::dynamic\_bitset}
boost::dynamic\_bitset相比std::bitset的好处在于,
dynamic\_bitset无需在编译时知道bitset的大小。
但dynamic\_bitset仍然是一个不完备的容器,例如它没有begin()/end()等函数返回的迭代器,
因此许多标准算法就不能用于dynamic\_bitset。

有几种方法可以将dynamic\_bitset转换为字符串:
\begin{itemize}
\item 用iostream/stringstream等输入输出流。
\item 用to\_string(dynamic\_bitset, string)函数,由boost::dynamic\_bitset库提供。
\item 用for循环,将dynmic\_bitset的每一位输出,用operator[]输出[0, size()]的每一位。
此种输出,与输出流的输出的bit流顺序正好相反。
\end{itemize}

\section{boost::string\_algo}
C++标准库的字符串是个巨库,但仍然没有如:icase\_compare(),
即忽略大小写的字符串比较等函数。遗憾的是,在Boost里也没发现这样一个函数。
Boost中倒是可以将字符串大小写转换,避免用C库的to\_uppper()和to\_lower()进行单个字符转换。

\begin{lstlisting}
#include <boost/algorithm/string.hpp>
string msg("Hello World\n");
void boost::to_upper(msg);
string boost::to_upper_copy(msg);
\end{lstlisting}
boost::to\_upper将修改传入的字符串,并返回void。如果不改变字符串且返回转换字符串,
则使用对应的\_copy函数。

\section{BOOST\_FOREACH}
BOOST\_FOREACH应该算是一个小巧实用的库,支持STL容器,内嵌数组等。
BOOST\_FOREACH库最早出现在boost 1.34.0。

\begin{lstlisting}
#include <boost/foreach.hpp>
list<Peer*> plist;
BOOST_FOREACH(Peer* peer, plist){
    peer->send_have();
}
\end{lstlisting}
在遇见小循环时,BOOST\_FOREACH似乎要比boost::bind和boost::lambda简单明了。

\section{boost::format}
Boost介绍format称它是printf-like的一个库。
boost::format是对printf和cout的一个很好补充。printf是类型不安全的,而且不能扩展,
而cout控制输出格式等方便又过于麻烦,boost::format则兼备两者之优,并独有所擅长。

\begin{lstlisting}
#include <boost/format.hpp>
boost::format("%s %3.2f%% %8d") %"info" %100.00 %1024;
\end{lstlisting}
以上即为熟悉的C风格。但boost::format还支持其他许多种的输出风格。

\chapter{log4cxx}
\noindent\href{http://logging.apache.org/log4cxx/index.html}{log4cxx Homepage} Apache项目组的log4cxx介绍。 \\
\noindent\href{http://www.dreamcubes.com/blog/?itemid=59}{log4cxx xml configure demo} 一个非官方的xml配置例子。
其中的xml配置文件已复制到了本地,因为它们太有用了。 \\
\noindent\href{http://expat.sourceforge.net/}{Expat库} log4cxx依赖这个XML解析库。\\

Apache log4cxx是一个C++的日志框架,是从log4j(j: Java)移植而来。
~log4cxx~使用APR(Apache Portable Runtime)。

如果调试人员不能接触代码,或是一个多线程大型的程序,使用日志是跟踪程序的好方法。
日志可以提供精确的context。日志也可以看作是很好的审核(auditing)工具。
日志的缺点则是会使程序变慢,而且太多的日志也会导致scrolling blindness。
而log4cxx则很好的解决了这些问题。可以通过修改log4cxx的配置文件,在不修改源码的基础上,控制日志行为。

我使用的log4cxx版本是:0.10.0,编译程序的链接库是:liblog4cxx.so.10。

\section{Compile log4cxx in Linux}
log4cxx的依赖库包括:apr, aprutil, expat库。


在Linux下编译log4cxx,发现编译时竟然有头文件未包含的错误。
例如使用memmove等函数,但没有include \cppheader{cstring},类似的低级错误,
而且至少有三四个文件有类似错误吧。

因为log4cxx依赖APR,因此下载apr库一起编译,\href{http://apr.apache.org/download.cgi}{这里}下载apr。

在Linux下静态链接log4cxx,编译通过的命令如下:
\begin{lstlisting}
$ g++ test.cpp liblog4cxx.a libapr-1.a libaprutil-1.a libapriconv-1.a libexpat.a -lpthread
\end{lstlisting}
编译后的执行文件约11MB。可见灰常复杂。而且,最惊讶是,以上编译通过竟然是瞎猫碰见了死耗子,
因为改变这几个静态库的顺序,就编译出错了。

除了以上库,还需要编译expat库,apr-util-1.3.4/xml/expat里有这个库。

\section{Use log4cxx}
log4cxx的三个重要概念是:Logger, Appender, Layout。
\begin{itemize}
\item Logger表示日志的层次结构。
\item Appender表示信息如何附加,如附加到文件或标准输出。
当日志有特殊要求,例如日志文档的大小限制,每日定时备份日志,定时将日志通过网络发送等,
就需要对Appender进行设置。
\item Layout表示日志输出的格式布局,如时间格式等。
\end{itemize}

在log4cxx的配置文件,或者程序源代码中,都可以设置以上的参数。
log4cxx的配置文件有几种格式,如XML和Java格式的配置文档。
而log4cxx的编程,也有许多细节需要注意。

\section{log4cxx xml configure}
至今还没有找到官方的关于log4cxx xml介绍的文章。目前的配置都是东摘西抄的:)
以下是一个简单的配置文件:

\noindent\begin{lstlisting}[language=XML, basewidth={0.45em, 0.25em}, fontadjust]
<?xml version="1.0" encoding="UTF-8" ?>
<log4j:configuration xmlns:log4j="http://jakarta.apache.org/log4j/">
    <appender name="testlog4cxx" class="org.apache.log4j.FileAppender">
        <param name="File" value="/home/bailing/fget/fget.log" />
        <param name="Append" value="true" />
        <layout class="org.apache.log4j.PatternLayout">
            <param name="ConversionPattern" value="%d{yyyy-MM-dd HH:mm:ss} (%F:%L) - %m%n"/>
        </layout>
    </appender>
    <root>
        <priority value="trace" />
        <appender-ref ref="testlog4cxx" />
    </root>
</log4j:configuration>
\end{lstlisting}

以上可见appender是配置文件的主体。其中指定了输出为文件,并且为附加方式。
在appender中有layout的设置,通常layout只有输出格式的设置。
root即是Logger,设置了日志的级别为trace,即跟踪方式,并将root的appender指定到上面的设置。

appender的class有多种,常用的有:

\begin{tabular}{|l|l|}
\hline
appender class                          & 说明 \\\hline
org.apache.log4j.FileAppender           & 文件方式输出。    \\\hline
org.apache.log4j.ConsoleAppender        & 终端输出。        \\\hline
org.apache.log4j.RollingFileAppender    & 每日备份日志。    \\
                                        & 设置文件大小,备份日志。  \\\hline
org.apache.log4j.net.SMTPAppender       & 邮件发送日志。    \\\hline
org.apache.log4j.odbc.ODBCAppender      & 将日志存入数据库。    \\\hline
org.apache.log4j.net.XMLSocketAppender  & 用XMLSocketAppender发送,用XMLSocketReceiver接收。 \\\hline
\end{tabular}

通常只用到输出文件和标准输出。

\section{log4cxx Programming}
一个最简单的log4cxx程序:
\begin{lstlisting}[language=C++]
#include <iostream>
#include <log4cxx/logger.h>
#include <log4cxx/xml/domconfigurator.h>

int main()
{
    setlocale(LC_ALL, "");
    log4cxx::xml::DOMConfigurator::configure("conf.xml");
    log4cxx::LoggerPtr logger = log4cxx::Logger::getRootLogger();
    LOG4CXX_DEBUG(logger, "hello, world");
    return 0;
}
\end{lstlisting}

编译需要动态链接log4cxx库,如:
\begin{lstlisting}
$ g++ main.cpp -L /usr/local/lib -llog4cxx
\end{lstlisting}

log4cxx预定义了几个宏,用于不同级别的日志输出:

\begin{lstlisting}[language=C++]
LOG4CXX_TRACE(log4cxx::LoggerPtr, msg);
LOG4CXX_DEBUG(log4cxx::LoggerPtr, msg);
LOG4CXX_INFO(log4cxx::LoggerPtr, msg);
LOG4CXX_ERROR(log4cxx::LoggerPtr, msg);
LOG4CXX_FATAL(log4cxx::LoggerPtr, msg);
\end{lstlisting}

其中msg可以视作stringstream,如下一般使用:
\begin{lstlisting}[language=C++]
LOG4CXX_ERROR(logger, "error : " << errno << strerror(errno));
\end{lstlisting}


%\part{Python Programming Lanauage}
\chapter{Language Introduction} \label{language-python}
\noindent\href{http://www.woodpecker.org.cn/diveintopython/}{Dive into Python在线} Python啄木鸟在线。

\section{From C/C++}
Python中Bool关键字有\blscmd{True}, \blscmd{False}。

Python中没有switch语句,建议是使用\blscmd{dictionary}。

Python的for循环从根本上不同于C/C++的for语句,而类似foreach语句。

Python中只有一种浮点数,即float,表示机器一级的双精度浮点数。为了不增加语言的复杂度,Python里没有double这种数据类型。

Python中进行数值运算的模块是\blscmd{cmath},大概对应C/C++的\cppheader{math.h}。
cmath.pi, cmath.e定义了两个常量。而像正余弦计算,对数、自然对数、开方、指数等运算,均在这个模块提供。

Python中复数(Complex Number)是内建类型,通过\blscmd{z=complex(1.0,2.2)},或\blscmd{z=(1.0+2.2j)}可以定义复数。
通过\blscmd{z.real}和\blscmd{z.imag}访问复数的实部和虚部。

类似C++标准库的map,Python中有内建\blscmd{dict}。

Python中数值与字符串相互转换的函数是:\blscmd{int()},\blscmd{float()}、\blscmd{str()}。

Python中,\blscmd{type()}即可查看对象类型。

\section{Statement}
\subsection{Loop}

\paragraph{for-loop} for的基本形式是:\blscmd{for i in sequence}。

\section{Build-in}
\subsection{Build-in function}
\blitem{len(x)}{统计x的元素个数,x通常是序列(string, tuple, list)或字典等。}

\subsection{array}
array是内置数组,它的常用成员函数是:
\blitem{count(x)}{计算x的出现次数。}
\blitem{append(x)}{push back x.}
array没有如size()的成员函数,可以用内建函数\blscmd{len()}计算数组元素个数。

%\part{Programming Language}

本部分主要一些小型语言的记录。如shell,awk, sed,metapost, dot等。
常使用的通用编程语言,如C/C++,Python单独成章。
一些具体应用领域的语言,如GCC汇编,Vim脚本等,分别归属于这些工具的笔记章节。
不过,这里仍是所有记录的一个入口,以下给出所有语言的索引。

C/C++的笔记见第\ref{language-c-cpp}部分。Python的笔记见第\ref{language-python}章。
GCC汇编的笔记见第\ref{language-gcc-assemble}节。Vim脚本的笔记见第\ref{language-vim-script}节。

对于编程语言,在学习过程中所写的代码,大都以外部文件的形式给出,并不附加于本文档。

%%%%%%%%%%%%%%%%%%%%%%%%%%%%%%%%%%%%%%%%%%%%%%%%%%%%%%%%%%%%%%%%%%%%%%%%%%%%%%%%
% bash
%%%%%%%%%%%%%%%%%%%%%%%%%%%%%%%%%%%%%%%%%%%%%%%%%%%%%%%%%%%%%%%%%%%%%%%%%%%%%%%%
\chapter{bash}

特殊变量\\
\begin{tabular}{ll}
\$*     &   展开为参数位置(成一个list)。\$@和\$*功能一样。\\
\$0     &   展开为shell或shell脚本名。\\
\$\$    &   展开为shell的PID。\\
\$?     &   展开为最近执行的程序的退出状态。\\
\$\#    &   展开为参数个数。\\
\end{tabular}

%%%%%%%%%%%%%%%%%%%%%%%%%%%%%%%%%%%%%%%%%%%%%%%%%%%%%%%%%%%%%%%%%%%%%%%%%%%%%%%%
% awk
%%%%%%%%%%%%%%%%%%%%%%%%%%%%%%%%%%%%%%%%%%%%%%%%%%%%%%%%%%%%%%%%%%%%%%%%%%%%%%%%
\chapter{awk}
awk有与C/C++类似的命令行参数处理,分别是ARGC, ARGV。

\blscmd{NF}表示field的个数。
\blscmd{FS}表示field的分隔符。

\blscmd{getline}是一个函数。但与普通函数不同的是,getline的语法类似一个语句,不能写成
getline()。getline从输入中读入一行,可以从文件、管道中读取。它的一个常见用法是:
\bllcmd{while( (getline $<$ "fname") $>$ 0 ) \{\} }
以上$<$表示重定向,而$>$则表示逻辑判断,与0的比较;其实是判断getline的返回值。

%%%%%%%%%%%%%%%%%%%%%%%%%%%%%%%%%%%%%%%%%%%%%%%%%%%%%%%%%%%%%%%%%%%%%%%%%%%%%%%%
% sed 
%%%%%%%%%%%%%%%%%%%%%%%%%%%%%%%%%%%%%%%%%%%%%%%%%%%%%%%%%%%%%%%%%%%%%%%%%%%%%%%%
\chapter{sed}

%\input{unix.tex}
\part{UNIX Tools}

UNIX Tools主要记录以下一些类别的UNIX工具:
\begin{itemize}
\item 常用工具。各种各样的常用工具,以及一些使用技巧。
\item 编译系统的主要工具。包括gcc, gdb, make。另有GNU出品的coreutil,包括strip等。
\item 编译系统的其他工具。如valgrind, gprof, objdump、nm等。
\item 网络工具。如traceroute, tcpdump, wget, curl,iptables等。
\item 其他工具。如perforce/p4源码管理客户端等。
\end{itemize}

UNIX变种的各种平台都大量使用GNU软件,\href{http://www.gnu.org/manual/manual.html}{这里}是GNU
软件的在线手册列表,包括gcc, gdb, libc, stdc++等。

%%%%%%%%%%%%%%%%%%%%%%%%%%%%%%%%%%%%%%%%%%%%%%%%%%%%%%%%%%%%%%%%%%%%%%%%%%%%%%%%
% UNIX environment tools
%%%%%%%%%%%%%%%%%%%%%%%%%%%%%%%%%%%%%%%%%%%%%%%%%%%%%%%%%%%%%%%%%%%%%%%%%%%%%%%%
\chapter{UNIX Enivronment Tools}

%%%%%%%%%%%%%%%%%%%%%%%%%%%%%%%%%%%%%%%%%%%%%%%%%%%%%%%%%%%%%%%%%%%%%%%%%%%%%%%%
% #environment basic
%%%%%%%%%%%%%%%%%%%%%%%%%%%%%%%%%%%%%%%%%%%%%%%%%%%%%%%%%%%%%%%%%%%%%%%%%%%%%%%%
\section{Basic}
\blscmd{adduser}添加用户,如:\blscmd{useradd bailing}。
\blscmd{passwd}为用户设置密码,如:\blscmd{passwd bailing}。
\blscmd{userdel}删除用户,如\blscmd{userdel bailing}。

修改\blscmd{/etc/sudoers}文件,为一般用户赋予需要root执行权限的命令,添加如下一行:
\bllcmd{bailing ALL=/usr/bin/p4}
这样一般用户bailing就具备了执行p4命令的权限,使用\blscmd{sudo p4}即可。
注意即使root用户也不能修改/etc/sudoers文件,使用命令\blscmd{visodu}即可。
如果要完全开放一个命令的执行权限,可以用\blscmd{chmod a+x}即可。

\blscmd{sudo}的操作会被日志记录。

\blscmd{wc}用于统计文件的行数,字符串个数等。
\blscmd{wc -l}表示输出行数。

\blscmd{sort}用于将文件每一行进行排序,默认按字典序排序。
\blscmd{sort -n}表示按数值排序。
\blscmd{wc f1 f2 f3 -l \textbar sort -n}按行数对f\{1-3\}文件排序。

\bllcmd{\$ netstat --tcp -n \textbar grep ``6699.*ES'' \textbar sort -r -n -k 3 \textbar less}
对netstat输出的第三列排序,从大到小。\blscmd{-k}指定目标排序域。

\blscmd{dd}命令可用于转储文件,man手册解释:convert and copy a file.
常用的切割文件的需求可用dd完成。如:
\bllcmd{dd if=fin of=fout skip=500 ibs=1 count=500}
以上\blscmd{ibs}指定了输入文件的切割单位是字节,\blscmd{count}指定了长度,以ibs为单位。

%%%%%%%%%%%%%%%%%%%%%%%%%%%%%%%%%%%%%%%%%%%%%%%%%%%%%%%%%%%%%%%%%%%%%%%%%%%%%%%%
% #environment tricks
%%%%%%%%%%%%%%%%%%%%%%%%%%%%%%%%%%%%%%%%%%%%%%%%%%%%%%%%%%%%%%%%%%%%%%%%%%%%%%%%
\section{Tricks}



使用Debian时,\blscmd{apt-get install ssh}后,发现用SecureCRT,以ssh连接debian,只要一段时间空闲,
就会被自动断开。因此需要设置:\blscmd{/etc/ssh/sshd\_config}文件,把\blscmd{LoginGraceTime}选项注释掉即可。

\blscmd{info core.util}查看系统所有的核心命令。

\blscmd{date}命令设置时间,可以用\blscmd{-d}查看字符串对应的时间是否正确,再用\blscmd{--set}设置时间。
\bllcmd{\$ date -d ``03/18/07 15:04''}, 输出\bllcmd{ 2009年 03月 18日 星期三 15:04:00 CST }

\blscmd{tee}read from standard input and write to standard output and files.
\blscmd{tcpdump}的手册中介绍的一个应用:
\bllcmd{tcpdump -l \textbar tee dat}
将\blscmd{tcpdump}输出行缓冲,并输出到stdout和dat文件。

%%%%%%%%%%%%%%%%%%%%%%%%%%%%%%%%%%%%%%%%%%%%%%%%%%%%%%%%%%%%%%%%%%%%%%%%%%%%%%%%
% #xargs
%%%%%%%%%%%%%%%%%%%%%%%%%%%%%%%%%%%%%%%%%%%%%%%%%%%%%%%%%%%%%%%%%%%%%%%%%%%%%%%%
\blscmd{xargs}build and execute command lines from standard input.
man xargs上的例子很值得学习。如:
\bllcmd{find /tmp -name core -type f -print \textbar xargs /bin/rm -f }
这个命令可能出现的错误是,如果文件名包含换号符(newlines)或空格(spaces),则会出错。因为\blscmd{xargs}以空格或换号作为分隔符。
\bllcmd{find /tmp -name core -type f -print0 \textbar xargs -0 /bin/rm -f }
这个命令则修复了上一个命令的错误。
\bllcmd{find /tmp -depth -name core -type f -delete}
与上一个命令效果相同,但效率更高,因为没有fork和exec的开销,也不用xargs的调用。
\bllcmd{cut -d: -f1 < /etc/passwd \textbar sort \textbar xargs echo}
产生一个紧凑的输出。如果没有xargs echo,则每个名字跟一个换行。

Bash的快捷操作:\\
\begin{tabular}{ll}
Ctrl-r  & 历史搜索模式。\\
Ctrl-a  & 将光标定位到命令开头。\\
Ctrl-e  & 将光标定位到命令结束。\\
Ctrl-u  & 剪切光标之前的内容。\\
Ctrl-k  & 剪切光标之后的内容。\\
Ctrl-y  & 粘贴由Ctrl-u/k剪切的内容。\\
Ctrl-t  & 交换光标之前两个字符的顺序,并将光标后移一位。\\
Ctrl-w  & 删除光标左边的一个word。\\
Ctrl-l  & 清屏。\\
\end{tabular}

在bash比较字符串的两个方法。用if语句:
\bllcmd{\$ if [ 1 = 3 ]; then echo ok; fi}
另一种方法是:
\bllcmd{\$ [ 1 = 3 ];ret=\$?;echo \$ret}

数学表达式:\blscmd{echo \$((1+3))}。

%%%%%%%%%%%%%%%%%%%%%%%%%%%%%%%%%%%%%%%%%%%%%%%%%%%%%%%%%%%%%%%%%%%%%%%%%%%%%%%%
% #ps, #top, #fuser, #lsof
%%%%%%%%%%%%%%%%%%%%%%%%%%%%%%%%%%%%%%%%%%%%%%%%%%%%%%%%%%%%%%%%%%%%%%%%%%%%%%%%
\section{Process Monitor}

%%%%%%%%%%%%%%%%%%%%%%%%%%%%%%%%%%%%%%%%%%%%%%%%%%%%%%%%%%%%%%%%%%%%%%%%%%%%%%%%
% #ps
%%%%%%%%%%%%%%%%%%%%%%%%%%%%%%%%%%%%%%%%%%%%%%%%%%%%%%%%%%%%%%%%%%%%%%%%%%%%%%%%
\subsection{ps}
\blscmd{ps -C fget -o pid=}输出进程fget的pid。可用在如top查看指定进程的运行时状态,
或用gdb挂载运行时进程:\blscmd{gdb fget `ps -C fget -o pid=`}。
以上命令更简单的模式是:\blscmd{gdb fget `pgrep fget`}。
如果有\blscmd{pidof}命令,则更为简单,\blscmd{gdb fget `pidof fget`}。

\blscmd{ps aux}输出进程的详细信息,其中进程状态一列的格式如下:\\
\begin{tabular}{ll} \hline
R   & Runing or runable(on run queue)\\
S   & Interruptible sleep(waiting for an event to complete) \\
Z   & defunct(zombie) process, terminated but not reaped by its parent \\\hline
<   & high-priority\\
N   & low-priority\\
L   & has pages locked into memory\\
s   & is a session leader\\
l   & is multi-threaded\\
+   & is in the foreground process group\\
\end{tabular}

查看mediaserver的进程状态输出是:\blscmd{SLl},则表明该进程在sleep,执行了锁页调用,
且是一个多线程进程。

\blscmd{-L} 显示线程信息,包括NLWP(number of thread)和LWP(Thread ID)。
\blscmd{-f} full-format listing。
\blscmd{-p PID} 输出PID进程的信息。
\blscmd{--no-headers} 不输出说明行。

%%%%%%%%%%%%%%%%%%%%%%%%%%%%%%%%%%%%%%%%%%%%%%%%%%%%%%%%%%%%%%%%%%%%%%%%%%%%%%%%
% #fuser
%%%%%%%%%%%%%%%%%%%%%%%%%%%%%%%%%%%%%%%%%%%%%%%%%%%%%%%%%%%%%%%%%%%%%%%%%%%%%%%%
\subsection{fuser}
\blscmd{fusr}查看使用某个文件/路径的所有进程。
\blscmd{-k}

%%%%%%%%%%%%%%%%%%%%%%%%%%%%%%%%%%%%%%%%%%%%%%%%%%%%%%%%%%%%%%%%%%%%%%%%%%%%%%%%
% #find, #grep
%%%%%%%%%%%%%%%%%%%%%%%%%%%%%%%%%%%%%%%%%%%%%%%%%%%%%%%%%%%%%%%%%%%%%%%%%%%%%%%%
\section{find and grep}
find是在目录中搜索文件。
grep是在文件中搜索字符串。
grep的名字源自ed行编辑器中的“全局正则表达式打印”\blscmd{g/re/p},
这个命令中的g表示在全局范围,/re/表示查找对象(正则表达式),p表示打印。

\blscmd{pgrep}和\blscmd{pkill},man手册中这两个工具是并列的,因此也一并记录。
\blscmd{pgrep}用来查找程序对于的PID,如\blscmd{pgrep fget},查找fget的PID。
\blscmd{pkill}向匹配的程序发送默认的SIGTERM信号。
\bllcmd{pgrep -u root sshd}列出所有属于root的sshd进程(AND)。
\bllcmd{pgrep -u root,daemon}列出所有属于root的进程,或deamon进程(OR)。

\verb|find ./ -name "*.h"  -exec grep setsock "{}" \;|

\blscmd{pgrep fget}则会搜索并输出fget的pid。
\blscmd{top -p `pgrep fget`}则可单只查看fget的信息。

``逻辑与"与``逻辑或"的grep表达式
\bllcmd{cat f \textbar grep ``word1 $\backslash$\textbar word2''}
\bllcmd{cat f \textbar grep ``word1.*word2''}

grep命令行参数:\\
\begin{tabular}{ll}
-i          & 不区分大小写\\
-v          & 反向搜索,不显示匹配项\\
-c          & 打印匹配行数,不显示具体的匹配内容\\
-I          & 不搜索二进制文件\\
\end{tabular}

\bllcmd{\$ netstat --tcp -n \textbar grep 6699 \textbar grep ES -c}

\blscmd{-I}不搜索二进制文件,同等\blscmd{--binary-files=without-match}。

%%%%%%%%%%%%%%%%%%%%%%%%%%%%%%%%%%%%%%%%%%%%%%%%%%%%%%%%%%%%%%%%%%%%%%%%%%%%%%%%
% bc, dc
%%%%%%%%%%%%%%%%%%%%%%%%%%%%%%%%%%%%%%%%%%%%%%%%%%%%%%%%%%%%%%%%%%%%%%%%%%%%%%%%
\section{bc and dc}
任意进制间的转换:
\bllcmd{ \$ echo 'ibase=10;obase=2;207' \textbar bc }
输出\blscmd{11001111}



%%%%%%%%%%%%%%%%%%%%%%%%%%%%%%%%%%%%%%%%%%%%%%%%%%%%%%%%%%%%%%%%%%%%%%%%%%%%%%%%
% #gcc
%%%%%%%%%%%%%%%%%%%%%%%%%%%%%%%%%%%%%%%%%%%%%%%%%%%%%%%%%%%%%%%%%%%%%%%%%%%%%%%%
\chapter{GCC}
\noindent\href{http://gcc.gnu.org/onlinedocs/}{GCC online documentation}GCC所有版本的手册列表。
\noindent\href{http://www.gnu.org/software/hello/manual/libc/}{GNU C Library Manual}其中有关于GNU C库
的细节说明。并且给出了一些编程技巧,如malloc hook,debug traceback(), 锁页mlock()等的说明。其中还有一些关于GNU
平台上特殊的函数等介绍。

\section{Compile Parameter}


\paragraph{-g}
生成调试信息。-glevel指定调试级别,默认为-g2。-g0表示没有任何调试信息,即-g0是-g取反(negates)。
-g3附加更多的调试信息,如宏定义/展开等。查看宏用\blscmd{p macroname},查看宏的展开形式用:
\blscmd{macro expand macroname}。需要查看宏时,还可加gcc选项-gdwarf-2。
    
\blscmd{man gcc}上说-g参数会为GDB生成专用调试信息,
这些调试信息会让GDB更好工作,但如果使用其他调试器,则可能因此而崩溃。
所以可以使用其他命令生成指定的调试信息,如\blscmd{-gstabs+},\blscmd{-gstabs},
\blscmd{-gxcoff},\blscmd{-gvms}等。

\paragraph{-D}
在命令行提供宏定义,如:
\bllcmd{g++ -DVERSION="\"fget 0.1.0\"" -DBUILD\_HOST=\"`uname -r`\" test.cpp}
需要注意引号的使用。双引号防止shell将参数里的空格作为命令行分隔符;
斜杠双引号,即双引号之间的内容,表示宏的值,包括双引号本身,
使用斜杠是防止shell解释引号。

\blitem{-fbounds-checking}{在代码中插入数组边界检查。man gcc说目前只对Java和Fortran管用。}
\blitem{-fstack-protector/-fstack-protector-all}{检查栈上数据。参考\ref{cpp-buffer-overflow-protection}的说明。}

\blitem{-static}{静态链接。man gcc的解释是:prevents linking with the shared libraries.
如:\blscmd{g++ -static hello.cpp}生成的a.out文件约1.3M,而不加此参数生成文件约6K。
也不能用\blscmd{ldd}查看a.out文件了,因为不是动态可执行文件。
}

\blitem{-fno-builtin/-fno-builtin-function}{不识别那些不以\_\_builtin\_作为前缀的内建函数。
GCC通常会为内建函数生成更有效率的代码,例如\blscmd{alloca}(在栈上分配空间)会成为一条指令,
而\blscmd{memcpy}则可能成为inline copy loops。这样做的结果就是生成代码更小更快,但这些函数
也许就不再存在,你不能设置断点或通过链接外部库来修改它们的行为。在使用\cppheader{math.h}
中的某些函数时,就可能遇见这些内建函数。
}

g++可以通过\blscmd{-fabi-version}控制ABI版本。另一些具体细节的控制,如\blscmd{-fpack-struct},
\blscmd{-fno-exception}等。

\blscmd{gcc --help=warning}查看关于warning的说明。

\section{GCC Assemble} \label{language-gcc-assemble}
\%ebp: frame pointer, 一般在所有函数调用第一行都是\blscmd{push \%ebp}。

\%eax: 用于存储函数返回值。GDB调试时查看函数返回值用命令\blscmd{p \$rax}。

\section{Ccache}
gcc编译时,用\blscmd{ccache},在编译时生成cache,以加速再次编译C/C++程序的速度,不过生成的文件也是颇大。
CCACHE\_DIR的默认值是\blscmd{\$HOME/.ccache}。可以直接删除此目录,也可通过\blscmd{ccache -C}删除编译缓存。
\blscmd{-F max-file-size}和\blscmd{-M maxsize}设置最大缓存文件大小,和缓存最大占用空间。
\blscmd{-s}查看缓存统计信息,包括命中次数,cache大小等。

除以上命令,还有许多环境变量控制ccache。

%%%%%%%%%%%%%%%%%%%%%%%%%%%%%%%%%%%%%%%%%%%%%%%%%%%%%%%%%%%%%%%%%%%%%%%%%%%%%%%%
% #gdb 
%%%%%%%%%%%%%%%%%%%%%%%%%%%%%%%%%%%%%%%%%%%%%%%%%%%%%%%%%%%%%%%%%%%%%%%%%%%%%%%%
\chapter{Debuging with GDB}
\section{Thinking in Debug}
常用的调试技巧有:

\paragraph{使用日志系统}
日志需要可以分级控制,可以用某种简单的开关控制,如C/C++的日志宏。
当出现系统级错误时,需要检查系统日志,如\blscmd{/var/log/*}, \blscmd{dmesg}等。

\paragraph{调试内核}
调试内核用的\blscmd{kgdb}, \blscmd{kdb}。
以及其他一些专用的调试器,如称为全能调试器的\blscmd{ODB}等。
\blscmd{kgdb}调试Linux内核需要两台计算机。\blscmd{kdb}调试内核只需一台计算机。
难以用普通调试器调试内核,是因为内核非以普通进程存在等。

\paragraph{使用断言}
使用适合项目的自定义ASSERT断言。可以通过-DNDEBUG编译选项关闭assert()宏。

\paragraph{使用调试器}
如\blscmd{GDB}。

\paragraph{跟踪系统调用}
使用\blscmd{strace}和\blscmd{ltrace}跟踪系统调用和库调用。

\paragraph{其他手段}
\blscmd{/proc/modules}中存储有当前所有被加载的系统模块。调试系统内核用\blscmd{printk}函数。

C/C++程序与内存相关的调试时最耗时的调试,有许多工具和方法可以帮助调试:

运行时用\blscmd{\$MALLOC\_CHECK\_=2 prog}可以检查内存。

\blscmd{efence}库可以检查两类内存问题:访问malloc()分配空间之外的内存;
访问已被free()的内存。只需在编译时静态或动态链接efence库即可。

\section{GDB Introduction}
\noindent\href{http://www.yolinux.com/TUTORIALS/GDB-Commands.html}{调试STL的GDB脚本}
用此脚本可以直接打印STL容器的内容。\href{http://www.stanford.edu/~afn/gdb\_stl\_utils/}{这里}
也有一个脚本可供打印STL。

在GDB中运行GDB脚本:source ~/.gdbinit

GDB对C++的支持不好(也许要追溯到GCC对C++的支持就不好),因此调试时不得以采用许多Tricks。
比如查看STL容器等,通常GDB会打印出令人费解的调试信息,通常你需要人工过滤掉你不需要的输出。
当然有热心人提供了GDB脚本来支持C++调试,但毕竟不是官方支持,而且这些脚本通常又是针对某一个特定的GDB版本。
GDB的一种Trick是在程序内部编写调试函数,并在GDB中用call func(parameters)的技巧。

GDB对回退调试(即返回到已经运行过的地方,rewind)的支持并不理想,是以bookmark方式实现。
回退调试也被称作反向调试,曾有GDB Path支持(record)。

在GDB中设置变量值,如\blscmd{p \$p=(Peer*) 0x86d9068},则可以像使用程序变量一样使用在GDB中定义的变量,如:
\blscmd{p \$p-\textgreater str()}。

\blscmd{sharedlibrary}加载系统动态库的符号文件。
\blscmd{show auto-solib-add}查看动态库是否自动加载。

GDB会捕获信号,如在socket编程常见的SIGPIPE。用命令\blscmd{handle SIGPIPE}
查看GDB对此信号的处理方式,用\blscmd{handle SIGPIPE nostop}修改处理方式,使之不中断程序。
对非错误信号,如SIGALRM, SIGWINCH, SIGCHILD,GDB默认执行nostop, noprint, pass操作,
对错误信号,GDB默认执行stop, print, pass操作。

\section{Debug Core Dump}
如果出现不可恢复的错误,OS将为程序生成core dump。
GDB调试core dump的命令是\blscmd{gdb prog core.PID}。
进入GDB调试后,通常用命令\blscmd{bt}或\blscmd{where}查看程序奔溃时的堆栈情况。
不过如果程序奔溃时堆栈也被破坏掉了(如越界写操作,可能将保存堆栈信息的内存破坏),
那么用bt或where命令就不能准确的定位问题,而GDB的输出可能是???等不可读信息。

如果程序奔溃时没有生成core.PID文件,则可能是系统限制,
检查\blscmd{ulimit -c}的值,如果输出为0,则表示当前shell禁止生成core文件。
可以修改其值,如\blscmd{ulimit -c 1024},表示可以生成core文件,
且将其大小限制为1024 bytes。

\section{Examining the Program}
\subsection{Examining the Stack}
使用\blscmd{bt},\blscmd{where}命令查看堆栈。
使用\blscmd{frame}查看调用帧。检查bt输出的调用栈,如果想用print命令打印函数参数却失败,
可以尝试用\blscmd{up}和\blscmd{down}移动堆栈,从而查看函数参数。

\subsection{Breakpoint}
\blscmd{display}自动显示断点处的数据,\blscmd{undisplay}取消显示。
\blscmd{jump}可以往回跳转,但不是起到rewind的作用,要回退调试,需要bookmark。

\blscmd{tbreak}可以用在直接跳转到指定行的场合,例如快速跳过while/for循环。
跳过循环体的标准做法可能是\blscmd{until}或\blscmd{u},GDB help的解释是:
Execute until the program reaches a source line greater than the current,
or a specified location within the current frame.

\blscmd{stepi}和\blscmd{nexti}是单步跟踪一条机器指令。

\section{Debug with Remote Target}
要远程调试程序,需在远程机上运行gdbserver,如:\bllcmd{gdbserver host:2345 --attach 27523}
注意,gdbserver参数的顺序,在某些系统上,与\blscmd{Debugging with GDB}手册的描述有所不同。

在本机上启动gdb,之后用\blscmd{file}命令加载符号文件,即本地编译的可执行文件;
再用\blscmd{target}命令连接目标机:\bllcmd{target remote tcp:192.168.16.105:2345}
之后可以像调试本机文件一样调试远程文件。

%%%%%%%%%%%%%%%%%%%%%%%%%%%%%%%%%%%%%%%%%%%%%%%%%%%%%%%%%%%%%%%%%%%%%%%%%%%%%%%%
% make
%%%%%%%%%%%%%%%%%%%%%%%%%%%%%%%%%%%%%%%%%%%%%%%%%%%%%%%%%%%%%%%%%%%%%%%%%%%%%%%%
\chapter{make}

%%%%%%%%%%%%%%%%%%%%%%%%%%%%%%%%%%%%%%%%%%%%%%%%%%%%%%%%%%%%%%%%%%%%%%%%%%%%%%%%
% binutils
%%%%%%%%%%%%%%%%%%%%%%%%%%%%%%%%%%%%%%%%%%%%%%%%%%%%%%%%%%%%%%%%%%%%%%%%%%%%%%%%
\chapter{binutils}
\blscmd{binutils}是对多种目标文件的操作工具集。
binutil主要包括以下工具:

\begin{tabular}{|l|l|}\hline
as          & 汇编器\\\hline
ld          & 链接器\\\hline
gprof       & 性能分析profiler \\\hline
addr2line   & 映射地址到代码行/文件 \\\hline
ar          & create, modify, and extract from archives\\\hline
nm          & list symbols in object files\\\hline
objcopy     & copy object files, possibly making changes\\\hline
objdump     & dump information about object files\\\hline
readelf     & display content of ELF files\\\hline
size        & list total and section sizes\\\hline
strings     & list printable strings\\\hline
strip       & remove symbols from an object file\\\hline
\end{tabular}

\blscmd{addr2line}映射地址到代码行/文件。
\bllcmd{\$ addr2line -e ./fget -s 08086890}
输出\blscmd{main.cpp:10}。
其中addr2line默认文件名为a.out,可用\blscmd{-e}指定输入文件。
\blscmd{-s}是不输出路径名。

\blscmd{strip}可以删除目标文件里的unnecessary信息。
试验\blscmd{strip}把\blscmd{fget}从原始的14MB变为652KB。

%%%%%%%%%%%%%%%%%%%%%%%%%%%%%%%%%%%%%%%%%%%%%%%%%%%%%%%%%%%%%%%%%%%%%%%%%%%%%%%%
% Program tools
%%%%%%%%%%%%%%%%%%%%%%%%%%%%%%%%%%%%%%%%%%%%%%%%%%%%%%%%%%%%%%%%%%%%%%%%%%%%%%%%
\chapter{Program Tools and Compile System}

%%%%%%%%%%%%%%%%%%%%%%%%%%%%%%%%%%%%%%%%%%%%%%%%%%%%%%%%%%%%%%%%%%%%%%%%%%%%%%%%
% valgrind
%%%%%%%%%%%%%%%%%%%%%%%%%%%%%%%%%%%%%%%%%%%%%%%%%%%%%%%%%%%%%%%%%%%%%%%%%%%%%%%%
\section{Valgrind}
Valgrind之名来自北欧神话中奥丁\footnote{奥丁是北欧神话主神,创造了人类,
掌管死亡、战斗、诗歌、魔法及智慧等。是众神之父。}神殿Valhalla的圣神入口。

Valgrind对全局数组,或栈上数组的访问越界无能为力。Valgrind FAQ上的解释是,
在目前memcheck的工作方式下,没有合理的方法实现访问越界检查。

\blitem{--db-attach=yes}{当Valgrind发现错误时,挂载调试器。}
\blitem{--leak-check=yes}{当使用memcheck时,检查内存泄漏。}

\subsection{Callgrind}
Callgrind用来监察函数的调用,可以分析函数的调用频率。
与memcheck不同的是,callgrind并不向标准输出写数据,而是在程序结束时,
将报告写入一个文件。如果想实时的监控callgrind的输出,可以使用
~callgrind\_annotate~和~callgrind\_control~这两个命令行工具。

\blscmd{callgrind\_annotate}读取callgrind的输出,并输出更易理解的内容。
\blscmd{callgrind\_control}可以实时观测callgrind的报告。

命令:
\begin{verbatim}
$ valgrind --tool=callgrind ./release/fget 
$ callgrind_control -b
\end{verbatim}
第二个命令callgrind\_control是在另一个终端开启的,因为valgrind占用了标准输出。
callgrind\_control的常用参数是:-b: 输出调用堆栈(像GDB的堆栈输出,但更简洁),
-s:输出程序状态,如PID,线程数等。


%%%%%%%%%%%%%%%%%%%%%%%%%%%%%%%%%%%%%%%%%%%%%%%%%%%%%%%%%%%%%%%%%%%%%%%%%%%%%%%%
% objdump 
%%%%%%%%%%%%%%%%%%%%%%%%%%%%%%%%%%%%%%%%%%%%%%%%%%%%%%%%%%%%%%%%%%%%%%%%%%%%%%%%
\section{objdump}
object可以反汇编二进制文件,如\blscmd{objdump -d ./fget}。
\paragraph{-d}反汇编二进制文件。
\paragraph{-C, --demangle}可以指定demangle的风格。
\paragraph{-l}输出携带行号和文件名信息。

%%%%%%%%%%%%%%%%%%%%%%%%%%%%%%%%%%%%%%%%%%%%%%%%%%%%%%%%%%%%%%%%%%%%%%%%%%%%%%%%
% nm 
%%%%%%%%%%%%%%%%%%%%%%%%%%%%%%%%%%%%%%%%%%%%%%%%%%%%%%%%%%%%%%%%%%%%%%%%%%%%%%%%
\section{nm}
\blitem{-C/--demangle[=style]}{Decode (demangle) low-level names into user-level names.}
\blitem{-l/--line-numbers}{打印符号对应的文件名和行号。}

%%%%%%%%%%%%%%%%%%%%%%%%%%%%%%%%%%%%%%%%%%%%%%%%%%%%%%%%%%%%%%%%%%%%%%%%%%%%%%%%
% Network tools, #network
%%%%%%%%%%%%%%%%%%%%%%%%%%%%%%%%%%%%%%%%%%%%%%%%%%%%%%%%%%%%%%%%%%%%%%%%%%%%%%%%
\chapter{Network Tools and Configure}
本章的网络工具包括netstat, route, ifconfig, traceroute, tcpdump,
iptables, ethtool, wget, curl, 
vnstat(实时网络流量统计), nmap(端口扫描侦测)等。

为使新系统连上网络,需配置:
\begin{enumerate}
\item 配置名字服务器。修改/etc/resolv.conf文件,添加DNS服务器地址。
\item 配置IP。用命令ifconfig eth0配置IP地址和子网掩码。
\item 配置路由。用命令route修改路由表。
\end{enumerate}

配置静态IP和路由,在Redhat/CentOS/Fedora上需要修改\blscmd{/etc/sysconfig/network-scripts/}路径下的
\blscmd{ifcfg-eth0}(配置IP)和\blscmd{route-eth0}(配置路由)文件。

%%%%%%%%%%%%%%%%%%%%%%%%%%%%%%%%%%%%%%%%%%%%%%%%%%%%%%%%%%%%%%%%%%%%%%%%%%%%%%%%
% #configure files
%%%%%%%%%%%%%%%%%%%%%%%%%%%%%%%%%%%%%%%%%%%%%%%%%%%%%%%%%%%%%%%%%%%%%%%%%%%%%%%%
\section{Configure Files}

%%%%%%%%%%%%%%%%%%%%%%%%%%%%%%%%%%%%%%%%%%%%%%%%%%%%%%%%%%%%%%%%%%%%%%%%%%%%%%%%
% #/etc/hosts
%%%%%%%%%%%%%%%%%%%%%%%%%%%%%%%%%%%%%%%%%%%%%%%%%%%%%%%%%%%%%%%%%%%%%%%%%%%%%%%%
\blscmd{/etc/hosts}配置主机名和IP地址的映射关系。添加如下一行:
\bllcmd{192.168.16.71 ls.funshion.com}
则调用如\blscmd{ping ls.funshion.com}命令时就会避开DNS查询,而直接连接指定的IP。
在Windows下,该配置文件名为:\blscmd{C:/WINDOWS/system32/drivers/etc/hosts}。

%%%%%%%%%%%%%%%%%%%%%%%%%%%%%%%%%%%%%%%%%%%%%%%%%%%%%%%%%%%%%%%%%%%%%%%%%%%%%%%%
% #ethtool
%%%%%%%%%%%%%%%%%%%%%%%%%%%%%%%%%%%%%%%%%%%%%%%%%%%%%%%%%%%%%%%%%%%%%%%%%%%%%%%%
\section{ethtool}
用\blscmd{ethtool}可以查看网络接口的硬件信息,并作修改。
例如使用ethtool修改网卡速度,如:
\bllcmd{ethtool eth0 -s speed 100  // 100Mbps}
网络速度受网卡硬件限制,同时也受网络接入速度限制,如接入的路由限制为100Mbps,
而设置1000Mbps则会出错。

网卡可能出于半双工状态,设置网卡的工作模式为自动协商:
\bllcmd{ethtool -s eth0 autoneg on}

%%%%%%%%%%%%%%%%%%%%%%%%%%%%%%%%%%%%%%%%%%%%%%%%%%%%%%%%%%%%%%%%%%%%%%%%%%%%%%%%
% #nmap
%%%%%%%%%%%%%%%%%%%%%%%%%%%%%%%%%%%%%%%%%%%%%%%%%%%%%%%%%%%%%%%%%%%%%%%%%%%%%%%%
\section{nmap}
\blscmd{nmap}是为Network Mapper之意,用作端口扫描。
手册上的一个例子是\blscmd{nmap -A -T4 www.google.com playground},可以看到
目标开放的端口,服务名词,使用的服务程序等。尽管都是nmap通过猜测所得,
但依然很有参考价值,比如扫描西科校友网,发现ftp(21/tcp)使用Serv-U ftpd 6.1,
http(80/tcp)使用IIS 6.0,且没有robots.txt。扫描西科(太多开放端口了),
http使用Apache 2.2.8,关闭了https(443/tcp)等。

\blscmd{-O}可以让nmap猜测对端的操作系统。

nmap猜测的对端系统启动日期uptime不太准。

%%%%%%%%%%%%%%%%%%%%%%%%%%%%%%%%%%%%%%%%%%%%%%%%%%%%%%%%%%%%%%%%%%%%%%%%%%%%%%%%
% #vnstat
%%%%%%%%%%%%%%%%%%%%%%%%%%%%%%%%%%%%%%%%%%%%%%%%%%%%%%%%%%%%%%%%%%%%%%%%%%%%%%%%
\section{vnstat}
据说Gnome中有一个网络速度查看软件\blscmd{netspeed},可以查看网络传输速度。

\blscmd{vnstat}可以实施监控网络速度,简单的用法如:\blscmd{vnstat -l -i eth0}
其中参数-l表示live,实时地更新速度。用Ctrl-C终止vnstat后,有一个数据统计输出。

vnstat源码包没有configure,只需直接make即可。make后的可执行文件在src/目录中。

%%%%%%%%%%%%%%%%%%%%%%%%%%%%%%%%%%%%%%%%%%%%%%%%%%%%%%%%%%%%%%%%%%%%%%%%%%%%%%%%
% #traceroute
%%%%%%%%%%%%%%%%%%%%%%%%%%%%%%%%%%%%%%%%%%%%%%%%%%%%%%%%%%%%%%%%%%%%%%%%%%%%%%%%
\section{traceroute}
\blscmd{traceroute}用于检测到达目的主机的中间路由。
通过traceroute可以查看到某个路由的时耗,调整系统性能。

%%%%%%%%%%%%%%%%%%%%%%%%%%%%%%%%%%%%%%%%%%%%%%%%%%%%%%%%%%%%%%%%%%%%%%%%%%%%%%%%
% #netstat 
%%%%%%%%%%%%%%%%%%%%%%%%%%%%%%%%%%%%%%%%%%%%%%%%%%%%%%%%%%%%%%%%%%%%%%%%%%%%%%%%
\section{netstat}

\begin{table}
\centering
\caption{netstat command list}
\begin{tabularx}{\textwidth}{lX}
-r          & 显示路由表。和\blscmd{route}命令作用相同。\\
-s          & 显示统计数据。\\
-n          & 不作名字解析。\\
-i          & 显示所有的网络接口。\\
-p          & 打印进程信息。\\
-c          & continue打印。\\
-o          & 显示timer。\\
-I          & 指定网络接口。可指定的接口包括\blscmd{eth0}, \blscmd{eth1}, \blscmd{lo}
            等,可以先用\blscmd{netstat -i}查看所有接口。
            在-I和接口间没有空格,参数需写作\blscmd{-Ieth0}。\\
指定协议    & 可指定的协议包括\blscmd{-t/--tcp}, \blscmd{-u/--udp},
                \blscmd{-S/--sctp}, \blscmd{-w/--raw}, \blscmd{-x/--unix}等。\\
\end{tabularx}
\end{table}

通过\blscmd{netstat -s}查看统计信息,可帮助改进程序性能。如观察TCP统计,可以获得
接收和发送的reset报文个数,可查看

通过\blscmd{netstat -i}或\blscmd{netstat -Ieth0}可查看网络接口的流量信息。

%%%%%%%%%%%%%%%%%%%%%%%%%%%%%%%%%%%%%%%%%%%%%%%%%%%%%%%%%%%%%%%%%%%%%%%%%%%%%%%%
% #wget and #curl 
%%%%%%%%%%%%%%%%%%%%%%%%%%%%%%%%%%%%%%%%%%%%%%%%%%%%%%%%%%%%%%%%%%%%%%%%%%%%%%%%
\section{wget and curl}
wget常用参数
\blitem{--limit-rate=amount}{下载限速。可以使用单位k, m作为后缀。
因为wget用sleep实现限速用,因此下载小文件时限速可能不能很好工作。}
\blitem{--user-agent}{让wget冒充其他客户程序。}

curl常用参数。\blscmd{--local-port PORT}指定本地端口。

%%%%%%%%%%%%%%%%%%%%%%%%%%%%%%%%%%%%%%%%%%%%%%%%%%%%%%%%%%%%%%%%%%%%%%%%%%%%%%%%
% tcpdump
%%%%%%%%%%%%%%%%%%%%%%%%%%%%%%%%%%%%%%%%%%%%%%%%%%%%%%%%%%%%%%%%%%%%%%%%%%%%%%%%
\section{tcpdump}
\noindent\href{http://www.tcpdump.org/}{Tcpdump Homepage} 这里还提供了一个名为libpcap的库,
随tcpdump一起发布,用于顶层抓包程序的编写。 \\
\noindent\href{http://www.tcpdump.org/tcpdump\_man.html}{Online man page} \\

\subsection{Command Paramters}
\paragraph{-D}列出所有tcpdump可以监听的物理接口。可以用\blscmd{-i}选择这个输出的接口,可以是数字或字符串。
如:\blscmd{tcpdump -i 1}或\blscmd{tcpdump -i eth0}。
如果系统没有\blscmd{ifconfig -a}命令,则\blscmd{tcpdump -D}就可派上用场。

\paragraph{-F}从\blscmd{-F}指定的文件读入filter expression,忽略命令行输入的表达式。

\paragraph{-i}指定tcpdump抓包的目标接口。如server和client运行在同一台计算机上,
用tcpdump抓包的命令是\blscmd{tcpdump -i lo},这里指定物理端口为lo本地环回端口。

\paragraph{-l}将输出行缓冲。一个应用如\blscmd{tcpdump -l \textbar tee dat},
或\blscmd{tcpdump -l \textgreater dat \& tail -f dat},输出到stdout和dat文件。

\paragraph{-n/-nn}\blscmd{-n}对IP地址不作DNS查询,\blscmd{-nn}对端口号不作名字转换。

\subsection{Expression}

\paragraph{host}指定监听的IP地址。\blscmd{src host}和\blscmd{dst host}分别制定源IP和目标IP地址。
\paragraph{tcp}指定监听TCP报文。
\paragraph{port}指定监听端口号,如\blscmd{tcpdump tcp port 6601}。

%%%%%%%%%%%%%%%%%%%%%%%%%%%%%%%%%%%%%%%%%%%%%%%%%%%%%%%%%%%%%%%%%%%%%%%%%%%%%%%%
% #iptables, #firewall
%%%%%%%%%%%%%%%%%%%%%%%%%%%%%%%%%%%%%%%%%%%%%%%%%%%%%%%%%%%%%%%%%%%%%%%%%%%%%%%%
\chapter{iptables}
iptables默认的表名叫filter,如果不用\blscmd{-t talbe-name}指定表,则规则都实施在filter表上。
除了filter表,iptables的内建表还有\blscmd{nat}和\blscmd{mangle}表。

Linux防火墙通常是由rc启动脚本中的一系列iptables命令来实现的。
iptables的常见形式为:\\
\blscmd{iptables -F chain-name},\\
\blscmd{iptables -P chain-name target},\\
\blscmd{iptables -A chain-name -i interface -j target}

\begin{tabular}{ll}
-A, --append        & 附加策略。\\
-I, --insert        & 插入策略。与-A不同是可指定策略位置。\\
-N, --new-chain     & 新建一条chain。\\

-f, --fragment      & 对于第二个或之后的framgmented报文起作用。\\

-m, --match [!]     & 指定匹配细则。\\

-F, --flush         & 把之前的所有规则从链中清除。\\
-X, --delete-chain  & 删除非内置的规则链。\\
-Z, --zero          & 清空所有计数(packages and bytes)。\\
-v, --verbose       & 详细输出。包括计数信息。\\

-j, --jump          & 跳转到指定处理规则。\\

-P                  & 给链设置一条默认的策略。\\
-p                  & 指定协议名。\\
-s                  & 源IP地址。\\
-d                  & 目标IP地址。\\
--sport             & 源端口。\\
--dport             & 目标端口。\\
-t                  & 指定表名。默认为filter表。\\
!                   & 否定一条子句。\\
\end{tabular}

target可以是:\blscmd{ACCEPT}, \blscmd{DROP}, \blscmd{REJECT}, \blscmd{LOG}, \blscmd{MIRROR},
\blscmd{QUEUE}, \blscmd{REDIRECT}, \blscmd{RETURN}, \blscmd{ULOG}。DROP是默默地丢弃,
REJECT则会返回一个ICMP错误消息,LOG提供日志,ULOG提供更加详细的日志。

\blscmd{-I, --insert chain [rulenum] rule-specification}和\blscmd{-A}类似,都是加入一条rule。
不同在-A是附加在chain末尾,而Insert(如果不指定rulenum, 即默认情况)是加入在rule之首。

\blscmd{-m, --match}可以跟很多参数。常见的如:
\blscmd{-m state --state STATE}, STATE可以为NEW, ESTABLISHED, RELATED等。
\blscmd{-m limit --limit RATE}, RATE单位可以是/second, /minute, /hour, /day。
\blscmd{-m limit --limit-burst N}, 一次最大匹配报文个数。    % todo 
\blscmd{-m conntrack --cstate STATE}, 类似--state, 但更详细。

\blscmd{[!] --syn}是只匹配设置SYN bit的TCP报文,ACK, RST, FIN位必须为0。

\blscmd{\$ iptables -L -n}列出所有的规则。

\blscmd{\$ iptables -I INPUT -p tcp --dport 6699 -j ACCEPT}

\blscmd{\$ iptables -A INPUT -p tcp ! --syn -m state --state NEW -j DROP}丢弃所有非SYN的新连接。

%%%%%%%%%%%%%%%%%%%%%%%%%%%%%%%%%%%%%%%%%%%%%%%%%%%%%%%%%%%%%%%%%%%%%%%%%%%%%%%%
% #monitor
%%%%%%%%%%%%%%%%%%%%%%%%%%%%%%%%%%%%%%%%%%%%%%%%%%%%%%%%%%%%%%%%%%%%%%%%%%%%%%%%
\chapter{Monitor Tools}
对于CPU,内存,磁盘I/O,网络I/O的信息,有多种名为\blscmd{*stat}的工具。某些是系统自带的,
某些则是需要自行安装的。以下略为区别:

\begin{tabular}{|l|l|l|}
netstat   & 自带    & 查看网络信息          \\\hline
vmstat    & 自带    & 查看虚拟内存信息      \\\hline
vnstat    & 安装    & 查看网络实时流量      \\\hline
iostat    & 安装    & 来自sysstat,查看磁盘实时信息     \\\hline
mpstat    & 安装    & 来自sysstat,查看CPU相关信息      \\\hline
pidstat   & 安装    & 来自sysstat,查看进程相关信息     \\\hline
sar       & 安装    & 来自sysstat,查看系统统计信息     \\\hline
\end{tabular}


\section{System Information}
\blscmd{lspci -vv}可以查看系统硬件的详细信息,不过一般不容易看懂,
因此需要其他工具帮助分析判断系统的硬件组成和其他系统信息。

\blscmd{locale}可以查看本地化信息。可以通过\blscmd{export LANG=zh\_CN.UTF-8}修改本地化信息。
设置本地化为英语可用\blscmd{en\_US}。

\blscmd{lsb\_release}可以查看发行版本(distribution)的信息,其中LSB是Linux Standard Base之意。
也可以通过\blscmd{cat /etc/issue}查看较为简单的发行版信息。
\blscmd{uname}可以查看一般性的系统信息。\blscmd{cat /proc/version}和uname命令输出大致相同。

%%%%%%%%%%%%%%%%%%%%%%%%%%%%%%%%%%%%%%%%%%%%%%%%%%%%%%%%%%%%%%%%%%%%%%%%%%%%%%%%
% #disk
%%%%%%%%%%%%%%%%%%%%%%%%%%%%%%%%%%%%%%%%%%%%%%%%%%%%%%%%%%%%%%%%%%%%%%%%%%%%%%%%
\section{Disk}
\href{http://www.acnc.com/04_02_01.html}{这里}介绍了Unix环境中的disk io benchmark工具。\\
\href{http://www.iozone.org/}{Iozone's Homepage}。对磁盘进行压力测试,并生成输出可视化。

在Linux系统,\blscmd{hdparm}命令可以查看ATA/IDE硬盘的属性。
SCSI和SATA硬盘的标识符都是sdX,可以用以下命令查看硬盘的类型:

\bllcmd{\$ grep -i ``[sh]d[a-z]'' /var/log/dmesg \textbar grep -i scsi}
就是查看dmesg的方式。

\blscmd{/proc/diskstats}提供了磁盘信息。可以在
\blscmd{/sys/block}中查看块设备的更加全面的信息。

\subsection{Sysstat}
\blscmd{iostat}是\blscmd{sysstat}中的一个组件,用来查看磁盘实时性能。

命令示例:\blscmd{iostate -d sda sdb -m 10},每10秒刷新一次屏幕,查看sda, sdb的读写信息,并以MB为单位。

\subsection{Iozone}
命令实例:\blscmd{iozone -RazMb diskio.xls}。

\begin{table}
\centering
\begin{tabularx}{\textwidth}{lX}
-R      & 生成Excel文件。\\
-b      & 指定Excel文件名。\\
-a      & full automatic mode。文件大小值域16k-512M,读写块值域4k-16k。\\
-M      & 使用uname()获取内核信息。\\
-s      & 指定文件大小。\\
-r      & 指定块大小。\\
-g      & 最大测试文件大小。\\
\end{tabularx}
\end{table}

Iozone的输出以Kbytes/s为单位。但测试的输出结果很奇怪,其值远大于用其他性能观察工具所视之值。待研究。

Iozone输出几种类型的测试信息。值得注意的是:\blscmd{Stride Read Report},手册介绍举例说,如每读4K数据,
就偏移200K,再读取4K数据,如此反复迭代。\blscmd{Fwrite}, \blscmd{Fread}表示用fwrite(), fread()系统调用。

%%%%%%%%%%%%%%%%%%%%%%%%%%%%%%%%%%%%%%%%%%%%%%%%%%%%%%%%%%%%%%%%%%%%%%%%%%%%%%%%
% #filesystem, #fs
%%%%%%%%%%%%%%%%%%%%%%%%%%%%%%%%%%%%%%%%%%%%%%%%%%%%%%%%%%%%%%%%%%%%%%%%%%%%%%%%
\section{FileSystem}
\blscmd{df -h}显示文件系统使用情况。
输出项中的/dev/shm表示virtual memory file system。

\blscmd{du}显示文件目录的占用空间大小。 
\blscmd{du -hs}显示当前目录(包括子目录)的大小。

%%%%%%%%%%%%%%%%%%%%%%%%%%%%%%%%%%%%%%%%%%%%%%%%%%%%%%%%%%%%%%%%%%%%%%%%%%%%%%%%
% #memory
%%%%%%%%%%%%%%%%%%%%%%%%%%%%%%%%%%%%%%%%%%%%%%%%%%%%%%%%%%%%%%%%%%%%%%%%%%%%%%%%
\section{Memory}
用\blscmd{free}、\blscmd{cat /proc/meminfo}查看内存。

free命令行参数:\\
\begin{tabular}{ll}
-b -k -m    & 显示内存大小的单位。\\
-s delay    & 显示延迟。\\
\end{tabular}

%%%%%%%%%%%%%%%%%%%%%%%%%%%%%%%%%%%%%%%%%%%%%%%%%%%%%%%%%%%%%%%%%%%%%%%%%%%%%%%%
% #cpu
%%%%%%%%%%%%%%%%%%%%%%%%%%%%%%%%%%%%%%%%%%%%%%%%%%%%%%%%%%%%%%%%%%%%%%%%%%%%%%%%
\section{CPU}
\blscmd{uptime}只有一行输出,用于查看系统的启动时间,当前用户数,以及系统负载。
系统负载有三个输出,包括最近1分钟,5分钟,15分钟内的系统负载。
如果系统负载的值超过3,则可能处于较大压力。
uptime的输出和\blscmd{w}输出的第一行相同。

\blscmd{uptime}和\blscmd{w}的man手册页说从\blscmd{/var/run/utmp}获取登录用户信息。
如果\blscmd{cat /var/run/utmp}则输出乱码以致干扰终端显示,慎用。


%%%%%%%%%%%%%%%%%%%%%%%%%%%%%%%%%%%%%%%%%%%%%%%%%%%%%%%%%%%%%%%%%%%%%%%%%%%%%%%%
% procinfo 
%%%%%%%%%%%%%%%%%%%%%%%%%%%%%%%%%%%%%%%%%%%%%%%%%%%%%%%%%%%%%%%%%%%%%%%%%%%%%%%%
\subsection{procinfo}
\blscmd{procinfo}主要输出cpu和内存信息。
在一些Linux发行版上不是自带的。procinfo主要解析\blscmd{/proc}文件系统,
与\blscmd{free},\blscmd{uptime}等输出类似。它的命令行参数如下:
\blscmd{-f} 全屏显示。
\blscmd{-nN} 间隔N秒输出一次。但不是如top等命令的刷屏,而是直接往下输出。
\blscmd{-m} 输出模块和设备,而非cpu和内存信息。



%%%%%%%%%%%%%%%%%%%%%%%%%%%%%%%%%%%%%%%%%%%%%%%%%%%%%%%%%%%%%%%%%%%%%%%%%%%%%%%%
% #rpm
%%%%%%%%%%%%%%%%%%%%%%%%%%%%%%%%%%%%%%%%%%%%%%%%%%%%%%%%%%%%%%%%%%%%%%%%%%%%%%%%


%%%%%%%%%%%%%%%%%%%%%%%%%%%%%%%%%%%%%%%%%%%%%%%%%%%%%%%%%%%%%%%%%%%%%%%%%%%%%%%%
% other tools
%%%%%%%%%%%%%%%%%%%%%%%%%%%%%%%%%%%%%%%%%%%%%%%%%%%%%%%%%%%%%%%%%%%%%%%%%%%%%%%%
\chapter{Other Tools}

%%%%%%%%%%%%%%%%%%%%%%%%%%%%%%%%%%%%%%%%%%%%%%%%%%%%%%%%%%%%%%%%%%%%%%%%%%%%%%%%
% perforce/p4
%%%%%%%%%%%%%%%%%%%%%%%%%%%%%%%%%%%%%%%%%%%%%%%%%%%%%%%%%%%%%%%%%%%%%%%%%%%%%%%%
\section{Perforce/P4}
一个用户可以有多个Client。Linux用命令\blscmd{p4 client cilentname}即可创建一个新的Client。
默认打开一个名为``Client Specification''的文件,修改以下字段:

\begin{tabular}{|l|l|}\hline
Client      & client名  \\\hline
Owner       & 一个owner可以有多个client \\\hline
Root        & 本地路径,如/home/berlin,按默认不用理会  \\\hline
View        & perforce和本地的映射关系  \\\hline
\end{tabular}

View的一个实例如\blscmd{//Project/... //berlin/workspace/...},\ldots表示所有文件,是一个通配符。
以上表示perforce的//Project/映射为/home/berlin/workspace,前缀/home/berlin由Root指定。

%%%%%%%%%%%%%%%%%%%%%%%%%%%%%%%%%%%%%%%%%%%%%%%%%%%%%%%%%%%%%%%%%%%%%%%%%%%%%%%%
% #berkeleydb
%%%%%%%%%%%%%%%%%%%%%%%%%%%%%%%%%%%%%%%%%%%%%%%%%%%%%%%%%%%%%%%%%%%%%%%%%%%%%%%%
\section{BerkeleyDB}
进入build\_unix,执行\blscmd{../dist/configure},不能在dist目录执行configure。
默认make install将库安置于/usr/local/BerkeleyDB.4.7/lib。

%%%%%%%%%%%%%%%%%%%%%%%%%%%%%%%%%%%%%%%%%%%%%%%%%%%%%%%%%%%%%%%%%%%%%%%%%%%%%%%%
% #cron
%%%%%%%%%%%%%%%%%%%%%%%%%%%%%%%%%%%%%%%%%%%%%%%%%%%%%%%%%%%%%%%%%%%%%%%%%%%%%%%%
\section{cron}
\blscmd{cron}自动执行周期性任务。在RH上cron无故被叫作了crond。
cron每分钟醒来一次,会执行crontab文件中的命令,如果crontab文件被修改,
则重新加载。因为掉电或者修改系统时钟导致的任务没有执行,cron不会补过执行。

crontab文件存在于以下3个地方:
\begin{enumerate}
\item \blscmd{/var/spool/cron/}这里存放一般用户的crontab。当用户执行\blscmd{crontab -e}
时,即可编辑其所属之crontab文件,并最后存储于/var/spool/cron/username中。但一般用户不能直接
访问/var中的文件。
\item \blscmd{/etc/crontab},这个文件由系统管理员维护,手工编辑。
\item \blscmd{/etc/cron.d/}这里通常存放软件包需要的cron操作。
\end{enumerate}

除了以上路径,Linux中另有一些如\blscmd{/etc/cron.weekly}, \blscmd{/etc/cron.daily},
\blscmd{/etc/cron.monthly}的目录,里面存放shell脚本,用以cron调度执行。

\blscmd{/etc/cron.deny}是拒绝指定用户执行cron。

%%%%%%%%%%%%%%%%%%%%%%%%%%%%%%%%%%%%%%%%%%%%%%%%%%%%%%%%%%%%%%%%%%%%%%%%%%%%%%%%
% #linux
%%%%%%%%%%%%%%%%%%%%%%%%%%%%%%%%%%%%%%%%%%%%%%%%%%%%%%%%%%%%%%%%%%%%%%%%%%%%%%%%
\chapter{Linux Distrubtion}
CentOS \href{http://www.centos.org/docs/5/}{CentOS 5.x文档库}。

\begin{tabular}{ll}
Fedora  & 是一种浅顶软呢帽,是RedHat上的那顶帽子。\\
Ubuntu  & 是一个南非民族的观念,意为“人道待人”。\\
Debian  & 是Ian Murdock在1998年命名的,来自他女友Debra和他名字的混合。\\
\end{tabular}

以下是在使用各种发行版中的问题记录,基本是一些小问题,但也花费了不少时间。

\bldate{2009/03/31} 在安装Debian系统时,如果只选择“桌面系统”和“基本系统”,你可能什么事也干不了,
因为可以没有gcc, gdb, make等一系列开发工具。所以还得用\blscmd{apt-get}把这些软件都安装。
而Debian安装时的可选粒度也不如fedora, CentOS的好,因此比较麻烦一些。

\bldate{2009/03/30} 安装Debian 5.00后,用SSH登录总是失败。检查Debian网络配置,一切正常。
可以连接上互联网。最后发现默认居然没有安装sshd服务。因此不得已,\blscmd{apt-get install ssh}安装sshd,
再\blscmd{/etc/init.d/ssh start}启动sshd。如此这般,就可以ssh登录了。

%\part{UNIX System Programming}

UNIX系统编程知识的主要来源是W. Richard Stevens的\emph{Advanced Programming in the UNIX Environment}。
另外的来源是GNU的手册,包括libc手册,libstdc++的手册等。

%%%%%%%%%%%%%%%%%%%%%%%%%%%%%%%%%%%%%%%%%%%%%%%%%%%%%%%%%%%%%%%%%%%%%%%%%%%%%%%%
% GNU C/C++ Programming
%%%%%%%%%%%%%%%%%%%%%%%%%%%%%%%%%%%%%%%%%%%%%%%%%%%%%%%%%%%%%%%%%%%%%%%%%%%%%%%%
\chapter{GUN C/C++ Programming}
\section{libstdc++}
\noindent\href{http://gcc.gnu.org/onlinedocs/libstdc++/manual/spine.html}{libstdc++ Manual} 官方手册。
其中有关于g++实现的许多细节问题,关于兼容性,ABI等的说明。
\noindent\href{http://gcc.gnu.org/onlinedocs/libstdc++/manual/configure.html}{Comiple g++} 官方指导。
对编译选项的说明。
\noindent\href{http://gcc.gnu.org/onlinedocs/libstdc++/manual/abi.html}{ABI} 官方关于ABI的说明。

\subsection{C++ ABI}
C++程序通常依赖于某些特定的运行时支持,如异常抛出和捕获。
当C++程序被编译为目标文件时,对象成员的对齐方式,mangling name,虚函数的实现等,都大相径庭,
这些细节定义于编译器的ABI Specification中(Application Binary Interface)。

%%%%%%%%%%%%%%%%%%%%%%%%%%%%%%%%%%%%%%%%%%%%%%%%%%%%%%%%%%%%%%%%%%%%%%%%%%%%%%%%
% freqenct api
%%%%%%%%%%%%%%%%%%%%%%%%%%%%%%%%%%%%%%%%%%%%%%%%%%%%%%%%%%%%%%%%%%%%%%%%%%%%%%%%
\chapter{Frequent API}
\section{Date and Time}
\subsection{Sleep}
milliseconds - 毫秒,1/1000 秒。microsenconds - 微秒, 1/1000 毫秒。
让UNIX程序休眠的系统调用有:sleep(), usleep(), nanosleep()。
其中sleep()的单位是秒,usleep()的单位是微秒,nanosleep的单位是纳秒。
UNIX上似乎没有专门以毫秒为单位的sleep函数。
UNP上还介绍了用select(), poll()实现sleep函数的方法,可以精确到微秒。
在man epoll上看到,epoll\_wait()的休眠单位是毫秒。

%%%%%%%%%%%%%%%%%%%%%%%%%%%%%%%%%%%%%%%%%%%%%%%%%%%%%%%%%%%%%%%%%%%%%%%%%%%%%%%%
% singal and error handle
%%%%%%%%%%%%%%%%%%%%%%%%%%%%%%%%%%%%%%%%%%%%%%%%%%%%%%%%%%%%%%%%%%%%%%%%%%%%%%%%
\section{Signal and Error Handle}
\blscmd{man 7 signal}查看信号说明,\blscmd{man 3 errno}查看错误说明。

\blscmd{SIGKILL}, \blscmd{SIGSTOP}不能被捕获,阻塞,或者忽略。

signal的信号值和对应解释:
\blitem{SIGABRT, 6, Core}  {Abort singal from abort(3).}
\blitem{SIGSEGV, 11, Core} {Invalid memory reference.}

\blitem{SIGKILL, 9, Term}   {Kill signal}
\blitem{SIGINT, 15, Term}   {Interrupt from keyboard, i.e. Ctrl-C。}
\blitem{SIGALRM, 14, Term}  {Timer singal from alarm(2).}

errno的错误码和对应解释:
\blitem{9}   {Bad file descriptor.}
\blitem{11}  {Resource temporarily unavailable}
\blitem{104} {Connection reset by peer.}
\blitem{113} {No route to host. 如果报这个错误,可能是对端的iptables设置,禁止了tcp连接端口。}
\blitem{115} {Operation now in progress. 如果是这个错误,可能是设置非阻塞socket,在connect()时返回此错误码。}

%\part{UNIX Network Programming}

UNIX网络编程知识的主要来源是W. Richard Stevens的\emph{UNIX Networkiing Programming}。
主要是API的使用,网络编程的常用模式,各种协议的网络应用等。

这部分关注的一个领域是高性能服务器的设计、开发。包括各种I/O模型的学习,如nonblocking socket
相关的select, poll, kqueue, epoll, complete port等。包括一些优化技术技巧,如cache,thread
等的使用。

\chapter{TCP and Socket API}
\subsection{SYN}
Windows系统中,只允最大存在10个TCP半开连接(half-open tcp connections)。
因为由此限制,在开启P2P软件时,通常P2P软件会占满此半开连接的限制,
所以用户打开web页面等网络应用时会很慢。不过如果将限制打开(很多P2P软件都由此功能),
会消耗更多系统资源,也有说导致路由器梗死,防火墙瘫痪等。因此有建议半开连接数不要超过50个。
\blcomment{打开限制,使P2P软件快速拉起流量?}
修改\blscmd{C:/WINDOWS/system32/drivers/tcpip.sys}可以改变这个限制。
\href{http://half-open.com/}{half-open.com}提供了工具查看和修改这个值。

\href{http://en.wikipedia.org/wiki/SYN_flood}{SYN flood}是一种DoS攻击方式。
正常TCP连接建立经过三个阶段,client-(SYN)-server, server-(SYN/ACK)-client,
client-(ACK)-server,由此TCP连接完成。但恶意客户端可以通过两种方式对server施以SYN攻击。
其一是client不响应ACK包,其二是client填入一个虚假的IP地址,
因此server发送SYN/ACK后等待一个不存在的IP响应ACK。

\subsection{RST}
有三种情况产生RST报文。
\begin{enumerate}
\item connect到一个没有进程监听的端口。
\item 异常终止一个连接。这样做的好处在于,丢弃任何待发送数据,并立即发生复位报文;
RST的接收方可以区分另一方是异常关闭连接。Sockect API通过设置``linger on close"选项(SOL\_LINGER)
提供这种异常关闭的能力,这将导致close时发生RST报文。
\item 检查半打开连接。
\end{enumerate}

\section{Socket Options}
\subsection{SO\_KEEPALIVE}
在TCP层实现侦测无活动连接,TCP自动发送keep-alive probe给对端,对端可能出现三种响应。
其一是接收到ACK,表明对端还存在,其二是接收到RST,表示对端crashed或reboot,此时有ECONNRESET错误,
其三是对端未有任何响应,此时有更多的侦测包发送(see UNP3/e, p200),如果最后仍无响应,则有ETIMEDOUT错误。
但是如果某个keep-alive probe的响应是ICMP错误,则表明host unreachable,将有EHOSTUNREACH错误。

对于侦测时间,通常是2小时。这个时间是pre-kernel相关,而非pre-socket相关的,因此修改此值,
将影响所有的socket。这个选项的目的是,发现对端crashed或unreachable(掉线,断电等)。

此选项多用在服务器程序中,如应对half-open connection,keep-alive选项将发现half-open连接并中断它们。

\section{Nonblocking I/O}
与非阻塞IO相关的网络API包括:select, poll等传统I/O轮询函数,以及如
epoll(Linux), kqueue(FreeBSD), /dev/poll(Solaris), IOCP(Windows)等新的更为高效的函数。
\href{http://monkey.org/~provos/libevent/}{libevent}这个跨平台的事件处理库就在这些平台使用了对应的轮询函数。

\subsection{Nonblocking Socket}
\begin{enumerate}
\item Input操作,对应recv系列函数,如果是非阻塞sockect,而读取不到数据,则返回EWOULDBLOCK。
\item Output操作,对应send系列函数,如果是非阻塞,且发送缓冲已满,则返回EWOULDBLOCK。
\item accept操作,如果是非阻塞且没有新连接,返回EWOULDBLOCK。
\item connect操作,如果connect不能立即完成,则返回EINPROGRESS。
\end{enumerate}

有时select告诉我们文件描述符已可读,但recv()返回-1,errno为EWOULDBLOCK,此为正常情况,忽略。
在Linux下查看man select,其中BUGS中就提到这个问题,并举例说,当有数据到达时,select通知可读,
但此时若数据的checksum失败,则数据被抛弃,read()就将失败。而EAGAIN, EWOULDBLOCK的说明是:
the socket is marked non-blocking and the requested operation would block.
SystemV使用EAGAIN,而Berkeley使用EWOULDBLOCK,但现在大多系统都将两者设为等值。

注意,以上recv()返回-1,errno为EWOULDBLOCK。如果slect()返回可读,但recv()返回0,则说明对端关闭了socket,
此时本地socket也应该被close。

如果select()返回-1,则说明有TCP连接坏了(可能之前就接受到了SIGPIPE),要具体判断是那个连接坏了,可用以下方法。
1、对每个文件描述符调用select(),返回-1则说明此条连接已坏。
2、对每个文件描述符调用getsockopt(),取SOL\_ERROR选项的值,如果其值大于0,则说明此条连接已坏。

关于非阻塞socket的connect(),应当是比较复杂的。如果connect()返回-1且errno为EINPROGRESS,
则需要调用select()判断sockfd。且在select()返回文件可读或可写后,应用getsockopt()判断是否有错误发生。
UNP上说,非阻塞socket的connect()操作的移植性非常低。因为不同系统上的判定方法不尽相同。
以下是摘自UNP 3/e上的代码片段:

\subsection{select}
\begin{lstlisting}[language=C++]
#include <sys/select.h>
#include <sys/time.h>

int select(int maxfd, fd_set* readset, fd_set* writeset, 
           fd_set* exceptset, const struct timeval* timeout);

struct timeval{
    long tv_sec;    // seconds
    long tv_usec;   // microseconds
};

void FD_ZERO(fd_set* fdset);
void FD_SET(int fd, fd_set* fdset);
void FD_CLR(int fd, fd_set* fdset);
int  FD_ISSET(int fd, fd_set* fdset);
\end{lstlisting}

\blscmd{FD\_SETSIZE}定义于\cppheader{sys/types.h}。
在Linux上,此值一般为1024,如果修改此值,需要重新编译内核。
在Windows系统上,这个宏的值默认为64,可以直接修改。

select() return -1,表明有socket坏掉了。
select() 的最后一个参数,如果传入NULL,表明永久等待,直到有事件发生;如果传入0,表明立即返回;如果传入具体时间,
则是等待超时。

\subsection{epoll}
\begin{lstlisting}[language=C++]
#include <sys/epoll.h>

int epoll_create(int size);
int epoll_ctl(int epfd, int op, int fd,
              struct epoll_event* event);
int epoll_wait(int epfd, struct epoll_event* events,
               int maxevents, int timeout);

EPOLL_CTL_ADD   // op: add
EPOLL_CTL_MOD   // op: modify listen event
EPOLL_DEL       // op: delete

typedef union epoll_data{
    void*       ptr;
    int         fd;
    __unit32_t  u32;
    __uint64_t  u64;
}epoll_data_t;

struct epoll_event{
    __unit32_t      events;
    epoll_data_t    data;
};
\end{lstlisting}

以下是来自CU的讨论:
\blcopy{
LT(level triggered)是缺省的工作方式,并且同时支持block和no-block socket.在这种做法中,
内核告诉你一个文件描述符是否就绪了,然后你可以对这个就绪的fd进行IO操作。
如果你不作任何操作,内核还是会继续通知你的,所以,这种模式编程出错误可能性要小一点。
传统的select/poll都是这种模型的代表。}

\blcopy{ET(edge-triggered)是高速工作方式,只支持no-block socket。在这种模式下,
当描述符从未就绪变为就绪时,内核通过epoll告诉你。
然后它会假设你知道文件描述符已经就绪,并且不会再为那个文件描述符发送更多的就绪通知,
直到你做了某些操作导致那个文件描述符不再为就绪状态了(比如,你在发送,接收或者接收请求,
或者发送接收的数据少于一定量时导致了一个EWOULDBLOCK 错误)。
但是请注意,如果一直不对这个fd作IO操作(从而导致它再次变成未就绪),内核不会发送更多的通知(only once),
不过在TCP协议中,ET模式的加速效用仍需要更多的benchmark确认。}

\blcopy{在许多测试中我们会看到如果没有大量的idle-connection或者dead-connection,
epoll的效率并不会比 select/poll高很多,但是当我们遇到大量的idle-connection(例如WAN环境中存在大量的慢速连接),
就会发现epoll的效率大大高于select/poll。}

%\part{Edit and Typeset}
\chapter{Vim}
\index{Vim}
\noindent\href{http://vimcdoc.sourceforge.net/}{vimcdoc},是一个Vim帮助的中文文档。
\noindent\href{http://easwy.com/blog/}{Easwy's Blog},许多的Vim技巧与插件介绍。

%%%%%%%%%%%%%%%%%%%%%%%%%%%%%%%%%%%%%%%%%%%%%%%%%%%%%%%%%%%%%%%%%%%%%%%%%%%%%%%%
% Tricks
%%%%%%%%%%%%%%%%%%%%%%%%%%%%%%%%%%%%%%%%%%%%%%%%%%%%%%%%%%%%%%%%%%%%%%%%%%%%%%%%
\section{Tricks}
\blscmd{:r !date} 在当前行插入时间。

\blscmd{:vimgrep /pattern/[j][g] files}
在多个文件搜索。j将搜索结果输出到quickfix窗口。用\blscmd{:clist}, \blscmd{:cnext},\blscmd{:cprevious}查看。
如果files是一个正则表达式匹配某种类型的文件,使用如\blscmd{**/*.cpp}将匹配当前目录下的所有cpp文件,且是递归搜索。

\blscmd{:gegisters}查看所有的寄存器内容。所有vim寄存器以\blscmd{"}开头,如\blscmd{"x}。
常用者如选中文本,\blscmd{"x x}剪切,切换文件,再执行\blscmd{"x p}即可。

要Linux用\blscmd{Shift-K}可以查看man手册,但如果如printf,shift-K就将你带到了man(1) printf,不是你想要的C printf。
可以在Vim使用命令\blscmd{3 shift-K}即可。

转当前文件为HTML页面:\blscmd{:TOhtml}。

捕获Shell命令输出,定向到新Buffer:\blscmd{:split +enew \textbar r !ls}。
以上\textbar不是管道操作,而是命令分隔符。\blscmd{split \{+cmd\}}表示打开一个新窗口,
并在窗口中执行\blscmd{\{cmd\}}这个命令。:h中解释\blscmd{+\{cmd\}}是说打开一个窗口并执行cmd命令。

打开DOS格式的文本,通常可见行尾的\^{}M控制字符,实则为$\backslash$r。
可以用命令\blscmd{:\%s/$\backslash$r//g}清除之。

\blscmd{:cs reset}重置所有的cscope文件。
\blscmd{:se ignorecase}和\blscmd{:se noignorecase}搜索时大小写敏感设置。
\blscmd{:ls}和\blscmd{:buffers}查看缓冲区列表。
\blscmd{gv}选中上次用visual模式选择的文本块。

\blscmd{:argdo \%s/a/b/ge $|$ update}对多个文件执行替换。\blscmd{e}避免找不到匹配时输出错误信息。
与\blscmd{argdo}类似的命令有\blscmd{windo},\blscmd{bufdo}。不过执行bufdo需要小心,最好先用\blscmd{:ls}
或\blscmd{:buffers}查看缓冲区列表o


%%%%%%%%%%%%%%%%%%%%%%%%%%%%%%%%%%%%%%%%%%%%%%%%%%%%%%%%%%%%%%%%%%%%%%%%%%%%%%%%
% Tricks
%%%%%%%%%%%%%%%%%%%%%%%%%%%%%%%%%%%%%%%%%%%%%%%%%%%%%%%%%%%%%%%%%%%%%%%%%%%%%%%%
\section{Plugin}
\index{Vim!Plugin}

\subsection{Mini Plugin}host-to-host
\blitem{matrix.vim}{黑客帝国屏幕。命令:Matrix启动。如果打开标签,则:Matrix会报错说不能打开窗口。}
\blitem{ColorSchemeMenuMaker.vim}{用于快速浏览色彩机制,用命令:ColorSchemeExplorer启动。}

\blitem{mark.vim}{多种色彩,同时高亮多个word。默认命令$\backslash$m即可高亮,或取消高亮;
    $\backslash$r手工输入正则表达式,然后高亮匹配;$\backslash$n清楚光标所在高亮或所有高亮。
    高亮非常方便,不过在高亮间跳转十分麻烦。\href{http://www.vim.org/scripts/script.php?script_id=1238}{这里}
    有具体的说明。
}

\subsection{A}
\index{Vim!Plugin!A}
A插件的作用是在头文件和源文件间跳转。
在Vim中使用:A命令即可。文件的关联方式和搜索路径都可以在\~{}/.vimrc中设置。
设置文件关联方式:
\bllcmd{let g:alternateExtension\_CPP = "inc,h,H,HPP,hpp"}
设置搜索路径:
\bllcmd{let g:alternateSearchPath = "sfr:../include,sfr:../src"}
搜索路径的设置很重要,sfr:后跟的是相对“当前文件”的搜索路径。
如存在:./src/utility.cpp和./include/utility.h,则在编辑utility.h时使用:A,
想要跳转到./src/utility.cpp,则搜索路径是:../src/。

\section{Tricks}
\subsection{Search and Replace}
将一串数字,如:1234567890,替换为另一串数字,如:
0x12, 0x34, 0x56, 0x78, 0x90, 
使用命令:
\begin{lstlisting}
:s/\(\d\{2}\)/0x\1, /g
\end{lstlisting}

在选定行的每一行后面加一个反斜杠。
如在C/C++定义宏时可用:
\begin{lstlisting}
:s/$/\\/g
\end{lstlisting}
消除每行后的反斜杠则是:
\begin{lstlisting}
:s/\\$//g
\end{lstlisting}

\section{Vim Script} \label{language-vim-script}

\chapter{\LaTeX{} Notes}
\index{\LaTeX}

\section{Introduction}
\noindent\href{http://www.ctan.org}{CTAN} the Comprehensive TeX Archive Network\\
\href{http://www.ctex.org}{CTEX} 中文Tex\\
\href{http://www.ctan.org/tex-archive/info/lshort/}{lshort} Ctan上lshort各个语言的版本,包括lshort文档的源码。
这本小册子是一个德国人写的,英文名叫:The Not So Short Introduction to \LaTeXe,副标题是``112分钟学会\LaTeXe''。\\
\href{http://en.wikibooks.org/wiki/LaTeX/Packages} {wiki上的\LaTeX宏包列表。} \\
\href{http://blog.sina.com.cn/s/blog\_51e68f8d0100avil.html}{\LaTeX{} listings宏包备忘。}

\TeX{} 被称作一种typographic语言,即一种排版印刷语言。

如果往\TeX系统中添加了新文件,需要更新\TeX系统。
te\TeX和fpTeX的命令是:``texhash''。
Mik\TeX的命令是:``initexmf -\mbox{}-update-fndb''。

\section{\LaTeX{} Command}
\noindent $\backslash$stretch\{十进制小数\} 是一个弹性长度,其自然值是0pt,其伸展值等于十进制小数乘以$\backslash$fill。\\
\noindent $\backslash$texttt\{文本\} 等价于\{$\backslash$ttfamily 文本\},它使用当前系列与形状,
但是打印机(typewriter)族的字体来打印文字。\\
\noindent $\backslash$vspace\{高度\} 产生具有指定高度的竖直空白,当它位于页首或页尾时被忽略。\\
\noindent $\backslash$hspace$*$\{高度\} 产生具有指定高度的竖直空白,当它位于页首或页尾时也不被忽略。\\


\subsection{\LaTeX{} File Type}

\begin{tabular}{|l|lp{\textwidth}|}
tex     & \LaTeX源文件,用latex命令编译。   \\
sty     & \LaTeX宏包文件,可以使用$\backslash$usepackage命令将宏包文件引入到\LaTeX中。\\
dtx     & 文档化\TeX文件。是\LaTeX宏包文件的主要发布格式。\\
ins     & 对应*.dtx文件的安装文件。网上下载\LaTeX宏包,一般会包含一个*.dtx文件和一个ins文件。
            使用\LaTeX处理*.ins文件可以解开*.dtx文件。\\
cls     & 定义文档外观形式的类文件,可以通过$\backslash$documentclass命令选取。\\
fd      & 字体描述文件,告诉\LaTeX有关新字体的信息。\\
dvi     & 设备无关文件,*.tex文件编译的主要输出文件。\\
        & 可以使用DVI预览器预览,或使用dvips/dvipdf等程序输出PS或PDF文档。\\
toc     & 存储所有的章节标题,再次编译时将读取该文件生成目录。\\
lof     & 类似*.toc,用于生成图形目录。\\
lot     & 类似*.toc,用于生成表格目录。\\
aux     & 用于再次编译时传递辅助信息,如交叉引用的信息。\\
idx     & 文档中如果包含索引,\LaTeX可使用该文件存储所有的索引词条。此文件需用makeindex处理,\\
        & 详见位于\pageref{latex-tools-makeindex}页的第\ref{latex-tools-makeindex}节。\\
\end{tabular}

% end of Latex Introduction

\section{Basic Kownledge}

\subsection{\LaTeX{} Source}
\LaTeX的源代码包括:
\begin{itemize}
\item 需要排版的文本。
\item \LaTeX命令。
\end{itemize}

\subsubsection{特殊字符}
特殊字符包括:
\framebox{\# \$ \% \^{} \& \_ \{ \} \ {} } \par

反斜杠不能通过$\backslash$$\backslash$得到,可以使用命令\$$\backslash$backslash\$生成反斜杠。

显示空格符号通过命令\fbox{$\backslash$textvisiblespace}生成,如\textvisiblespace。

带圈的符号通过命令\fbox{$\backslash$textcircled\{letter\}}生成,如:
\textcircled{\scriptsize 1}
\textcircled{\scriptsize a}
\textcircled{\scriptsize c}
\textcircled{c}。

引号不能直接使用“。左引号由两个`(重音)产生,右引号由两个'(直立引号产生)。一个`和一个'产生一个单引号。

\subsubsection{空白距离}
空格和制表符等空白字符在\LaTeX中被视作相同的space,多个连续的空白被视作一个空白。
两行文本间的空白标志上下段,多个空白行等同一个空白行。

\subsubsection{\LaTeX命令}
\LaTeX命令有两种格式:
\begin{itemize}
\item 以反斜杠$\backslash$开始,命令只由字母组成。命令后的空格、数字或任何非字母的字符都标志命令的结束。
\item 由反斜杠$\backslash$和非字母的字符组成。
\end{itemize}
\LaTeX忽略命令之后的空格,因此如果需要在命令后得到空格,可以在命令后加\{\}和一个空格,
或加上一个特殊的空格命令。\{\}将阻止\LaTeX吃掉命令后的空格。
对比"\LaTeX without space"和"\LaTeX{} with space"。

命令的参数放在\{\}中,命令的可选参数放在[]中。

\subsubsection{注释}
注释放在\%后,等同C++中的//。符号\%也可以用来断开不能含有空白字符或换行符的较长输入内容。
如果注释的内容特别长,可以使用verbatim宏包提供的comment环境。comment环境中的注释不显示在文本中。

% end of Latex Source

\subsection{Environment}
\index{\LaTeX!Environment}

verbatim\index{\LaTeX!Environment!verbatim}可以将文本直接打印。
verbatim宏包可以更好的达到这个效果。\ref{latex-macro-package-verbatim}。

% end of "Latex Environment"

\subsection{Box}
\makebox[\textwidth]{central}\par
\makebox[\textwidth][s]{spread}\par
\makebox[\textwidth][r]{right}\par
\framebox[\textwidth]{这叫中规中矩}\par
\framebox[0.8\width][r]{从左边溢出了}\par
\framebox[1cm][l]{从右边溢出了}\par

命令\fbox{$\backslash$mbox\{text\}}保证把几个单词放在同一行上。
命令$\backslash$fbox和$\backslash$mbox类似,它还能围绕内容画一个框,如每个命令周边的框就是fbox的功劳。

% end of "Latex Box"

\section{Macro Package}
\index{\LaTeX!macro package}
常用的宏包:\\
\begin{tabular}{|l|l|}
\hline
宏包            & 说明 \\ \hline
makeidx         & 提供排版索引的命令。随\LaTeX提供。\\ \hline
verbatim        & 直接打印文本。\\ \hline
fancyvrb        & 美化的verbatim。\\ \hline
indentfirst     & 首行缩进。特别对中文有用。 \\ \hline
listings        & 用于在\LaTeX中引用源码。\\ \hline
xcolor          & 与listings搭配使用,源码语法高亮。 \\ \hline
beamer          & 创建演示文稿。\\\hline
\end{tabular}

\subsubsection{verbatim宏包}\index{\LaTeX!macro package!verbatim}\label{latex-macro-package-verbatim}
verbatim环境可以直接包含打印文本,而verbatim宏包则可以直接引入ASCII文本,命令
``$\backslash$verbatiminput\{filename\}''即可。
fancyvrb宏包是美化后的verbatim,它的输出和listings很相似。

\subsection{listings}\index{\LaTeX!macro package!listings}
\begin{quote}
众里寻它千百度。
\end{quote}

listings宏包用于\LaTeX中的源代码引用,效果如下:
\begin{lstlisting}
int main(int argc, char* argv[])
{
    printf("hello, world!\n");
    return 0;
}
\end{lstlisting}

listings可以和xcolor搭配使用,让源代码语法高亮。
listings的一个示例如下:
\begin{Verbatim}[frame=single, rulecolor=\color{green}]
\begin{lstlisting}[
    language={[ANSI]C},
    numbers=left,
    numberstyle=\tiny,
    keywordstyle=\color{blue!70},
    commentstyle=\color{red!50!green!50!blue!50},
    rulesepcolor=\color{red!20!green!20!blue!20},
    escapeinside='',
    xleftmargin=2em,
    xrightmargin=2em,
    frame=shadowbox,
    aboveskip=1em]
int main(int argc, char* argv[])
{
    /* comment */
    printf("hello, '中国'!\n");
}
\end{lstlisting}
\end{Verbatim}

以上,numbers表示在源码前加以行号,可以指定行号的位置,左或右;
numberstyle表示行号风格;之后是色彩的设置;因为litings不支持中文,
所以如果在源码中包含中文,需要“转义”,用escapeinside指定转义符,
以上用'单引号作为转义符;xleftmargin和xrightmargin指定了左右边距;
aboveskip指定了上边距;frame指定了源码边框,以上为shadowbox。

可以把listings的设置放在lstset\{\}中,
这样就不用每次使用listings时都设置各项参数。

\subsubsection{语言支持和语法高亮}
\paragraph{语言支持}
listings支持常见语言,如C(ANSI, Objective, Sharp), 
    C++(ANSI, GNU, ISO, Visual), Python, \TeX(\LaTeX), 
    make, XML, SQL,
    PHP, Perl, Ruby, Lisp, Java等。

\paragraph{语法高亮}
可以自由配置高亮单词,在lstset\{\}中添加如:
\begin{lstlisting}
emph={boost},emphstyle=\color{red},
emph={[2]vector,list,iterator},emphstyle={[2]\color{blue}}
\end{lstlisting}
这样的语句即可。

\paragraph{定制边框}
可以设置代码的边框,在$\backslash$begin\{lstlisting\}[]中添加frame设置。
frame的可选值为:none, leftline, topline, bottomline, lines, single, shadowbox。
lines表示在上下划线,single表示在四周划线,shadowbox表示一个阴影框。
frame的另一个表示方式是取\{trblTRBL\}的任意子集,其中大写字母表示双划线。

\subsection{beamer}\index{\LaTeX!macro package!beamer}
Beamer是Till Tantau在2003年创建的,用于演示文稿。帮助手册参考:
\verb|texmf\doc\latex\beamer\beameruserguide.pdf|。

% end of Macro Package

\newpage
\section{Large Document}
\subsection{Create Index}
\index{\LaTeX!index}
\begin{enumerate}
\item 引入宏包:"$\backslash$usepackage\{makeidx\}"。
\item 在导言中使用命令:"$\backslash$makeindex"激活索引命令。
\item 要索引的内容通过命令:"$\backslash$index\{key\}"指定,key是索引项的关键词。
\item 当\LaTeX处理源文件时,每个$\backslash$命令都会将适当的索引项和当前页面写入一个专门的文件*.idx中,因此需要使用外部命令
"makeindex\label{latex-tools-makeindex} fname.idx"来生成索引。
\item 在源文件的适当位置(一般位于文档最后),将索引文件引入源文件,使用命令"$\backslash$printindex"。
\end{enumerate}

\newpage
\section{Chinese}
\subsection{Chinese Bookmark}
在tex文档序言部分引入:
\begin{lstlisting}
\usepackage[pdftex,CJKbookmarks]{hyperref}
\end{lstlisting}

王垠提供了一段代码,也可以达到以上效果:
\begin{lstlisting}[language=TEX]
\usepackage[pdftex]{hyperref}
\pdfstringdefDisableCommands{
\let\CJK@XX\relax
\let\CJK@XXX\relax
\let\CJK@XXXp\relax
\let\CJK@XXXX\relax
\let\CJK@XXXXp\relax
}
\end{lstlisting}

生成中文书签的步骤:
\begin{lstlisting}
pdflatex cs.notes.tex
gbk2uni cs.notes.out
pdflatex cs.notes.tex
\end{lstlisting}

CTEX中带了gbk2uni这个小工具。

% end of Latex

 % vim, latex, metapost, graphzi, etc.
%\part{Protocol Notes}
\noindent\href{http://linux-ip.net/html/}{Guide to IP Layer Network Administration with Linux}介绍了许多Linux下的网络知识,
包括常用的网络软件等。

POP3(fecting mail), SMTP(sending mail), IRC(chat), LDAP(directory access).

\section{NAT}
NAT类型:
\begin{enumerate}
\item \blscmd{Full cone NAT},内部地址(iAddr:port1)映射为外部地址(eAddr:port2)。
任意从(iAddr:port1)发出的报文都将转为(eAddr:port2),
任意从(eAddr:port2)接收的报文都将送到(iAddr:port1)。
\item \blscmd{(Address) Restricted cone NAT},只有当(iAddr:port1)发送报文给(hostAddr:any),
(hostAddr:any)才可发送报文给(iAddr:port1)。此种NAT是地址限制,但port不限制之NAT。
\item \blscmd{Port-Restricted cone NAT},(iAddr:port1)通过(eAddr:port2)发送报文给(hostAddr:port3),
只有(hostAddr:port3)才可发送报文给(iAddr:port1)。
\item \blscmd{Symmetric NAT},每个从内网地址发出的报文到达不同的外网地址,都将映射出一道唯一的通往外网的通路。
\end{enumerate}
以上后三种类型的NAT,都一定要(iAddr:port1)首先发送报文,才可做NAT。

\section{BT}
\noindent\href{http://en.wikipedia.org/wiki/Comparison\_of\_BitTorrent\_software}{wiki上的BT软件列表} 这个列表
比较全面的列出了各个平台下的BT软件,并做了非常详细的对比,如支持的BT特性,使用的编程语言,以及BT库等。
\noindent\href{http://en.wikipedia.org/wiki/Libtorrent\_(Rasterbar)}{wiki上的libtorrent(Rasterbar)介绍} 这个BT库
又叫rb-libtorrent,用了Boost.Asio,以此获得跨平台性。

\subsection{libtorrent Notes}
libtorrents是一个用C++写的BT库,其中用了Boost,获取跨平台性。
Arctic Torrent就使用了该库,而Arctic Torrent的目标是使用尽量少的内存和CPU。

libtorrents中,一个piece拥有多个blocks。


%-------------------------------------------------------------------------------- 

\clearpage      % 如果是双面打印,即openrigth,则使用\cleardoublepage
\addcontentsline{toc}{chapter}{Bibliography}    % 将参考文献加入到目录中

\begin{thebibliography}{99}
\bibitem{tcpl} B.Stroustrup: The C++ Programming Lanauage(2002)
\end{thebibliography}

\clearpage      % 如果是双面打印,即openrigth,则使用\cleardoublepage
\addcontentsline{toc}{chapter}{Index}    % 将索引加入到目录中

\printindex

\end{CJK*}
\end{document}
%-------------------------------------------------------------------------------- 


% ------------------------------------------------------------------------------

\begin{CJK*}{GBK}{song}
\title{\huge \bfseries CS Notes}
\end{CJK*}

\makeindex

% ------------------------------------------------------------------------------

\begin{document}
\begin{CJK*}{GBK}{song}

\maketitle

\clearpage      % 如果是双面打印,即openrigth,则使用\cleardoublepage
\addcontentsline{toc}{chapter}{Contents}    % 将目录本身加入到目录中

\tableofcontents

% ------------------------------------------------------------------------------

% 避免影响目录
%\setlength{\parindent}{0pt}                        % 将段首缩进设置为0
\setlength{\parskip}{1ex plus 0.5ex minus 0.2ex}    % 修改段落间距

% ------------------------------------------------------------------------------

\chapter{Introduction}
CS.Notes主要是学习工作中的随记,大多来自HOWTO、man手册、各种软件自带的帮助文档,技术书籍,
另有网上看到的CS相关的技术文章或逸闻趣事。不论程序设计还是文档写作,一个诀窍是Copy and Plaster。
然而,俗话说吃亏就是占便宜,反过来讲,便宜不好占;因此本文绝大多数的内容,虽非原创,但不复制粘贴。
这样就可以避免使文档内容呈爆炸式的增长,也可以在键入每个字的时候,加深印象。
好记性不如烂笔头,书不厌读,识不厌记。

文档的内容主要是:
\begin{itemize}
\item CS随记,包括CS逸闻轶事逸言,CS读书笔记等。
\item C/C++程序设计,包括语言、库、调试技巧等。
\item UNIX程序设计,包括编译系统、系统编程、网络编程等。
\item UNIX环境,包括Bash、系统命令等。
\item UNIX网络,如iptable/netstate/tcpdump等各种网络工具的使用。
\item 网络协议,如TCP/IP/HTTP/FTP/BT等。
\item 数据结构和算法。
\item 硬件相关的笔记。
\item Python程序设计。
\item Vim编辑器。
\item 使用\LaTeX或者DocBook排版。
\end{itemize}

\begin{flushright}
-\ {} 2009/02/02
\end{flushright}

% end of "Introduction"

\chapter{Miscellanea Notes}
\section{Quote}
\begin{quote}
Make simple things easy.\\
-\ {}Larry Wall
\end{quote}

\section{Working Notes}
\emph{\date{2009/02/10}} 用GDB调试的一个问题。编译时加-g参数,编译生成a.out。
在serv-a上编译,在serv-b上调试。在GDB中用edit查看源文件,发现源文件不同;
用系统命令md5sum计算serv-a/a.out和serv-b/a.out,发现两个的MD5值却完全相同。
导致此问题的原因是:-g参数并不将源代码嵌入可执行程序,而只是在可执行程序中
嵌入调试信息,与源代码关联起来。因此在serv-b上看到程序源码,完全出于巧合,
即假设在serv-a的/home/bailing/编译./hello.cpp,而在serv-b的/home/bailing/目录下,
恰好存在一个名为hello.cpp的文件,因此被GDB发现;如果没有hello.cpp,
则GDB会抱怨说找不到源码。

\section{Working in Funshion}

\part{C/C++ Programming Language}
%\chapter{C/C++ Programming Language}
\index{C/C++}
参考TC++PL \cite{tcpl}。

\section{Standard Library}
\cppheader{string.h}中,除了memmove外,其他函数都没有定义重叠对象间的复制行为。
strerror()也在\cppheader{string.h}头文件中。

\chapter{Library-Boost}
\subsection{Boost Compile and Install}
\paragraph{Compile in Linux}
在Linux下编译Boost。

\subsection{boost::dynamic\_bitset}
boost::dynamic\_bitset相比std::bitset的好处在于,
dynamic\_bitset无需在编译时知道bitset的大小。
但dynamic\_bitset仍然是一个不完备的容器,例如它没有begin()/end()等函数返回的迭代器,
因此许多标准算法就不能用于dynamic\_bitset。

有几种方法可以将dynamic\_bitset转换为字符串:
\begin{itemize}
\item 用iostream/stringstream等输入输出流。
\item 用to\_string(dynamic\_bitset, string)函数,由boost::dynamic\_bitset库提供。
\item 用for循环,将dynmic\_bitset的每一位输出,用operator[]输出[0, size()]的每一位。
此种输出,与输出流的输出的bit流顺序正好相反。
\end{itemize}

\chapter{UNIX System and Network Programming}
W. Richard Stevens的两部书:
\begin{itemize}
\item Advanced Programming in the UNIX Environment
\item UNIX Networkiing Programming
\end{itemize}

\section{System Programming}
\subsection{Sleep}
让UNIX程序sleep的系统调用有:sleep(), usleep(), nanosleep()。
其中sleep()的单位是秒,usleep()的单位是微秒,nanosleep的单位是纳秒。
UNIX上似乎没有专门以毫秒为单位的sleep函数。


\section{Sockets Networking API}
\begin{lstlisting}[language={[ANSI]C}, frame=showbox]
#include <netinet/in.h>

uint16_t htons(uint16_t host16bitvalue);
uint32_t htonl(uint32_t host32bitvalue);
uint16_t ntohs(uint16_t host16bitvalue);
uint32_t ntohl(uint32_t host32bitvalue);
\end{lstlisting}

% end of "UNIX Environment"

\chapter{\LaTeX{} Notes}
\index{\LaTeX}

\section{Introduction}
\noindent\href{http://www.ctan.org}{CTAN} the Comprehensive TeX Archive Network\\
\href{http://www.ctex.org}{CTEX} 中文Tex\\
\href{http://www.ctan.org/tex-archive/info/lshort/}{lshort} Ctan上lshort各个语言的版本,包括lshort文档的源码。
这本小册子是一个德国人写的,英文名叫:The Not So Short Introduction to \LaTeXe,副标题是``112分钟学会\LaTeXe''。\\
\href{http://en.wikibooks.org/wiki/LaTeX/Packages} {wiki上的\LaTeX宏包列表。} \\
\href{http://blog.sina.com.cn/s/blog\_51e68f8d0100avil.html}{\LaTeX{} listings宏包备忘。}

\TeX{} 被称作一种typographic语言,即一种排版印刷语言。

如果往\TeX系统中添加了新文件,需要更新\TeX系统。
te\TeX和fpTeX的命令是:``texhash''。
Mik\TeX的命令是:``initexmf -\mbox{}-update-fndb''。

\subsection{\LaTeX{} File Type}

\begin{itemize}
\item *.tex     \LaTeX源文件,用latex命令编译。
\item *.sty     \LaTeX宏包文件,可以使用$\backslash$usepackage命令将宏包文件引入到\LaTeX中。
\item *.dtx     文档化\TeX文件。是\LaTeX宏包文件的主要发布格式。
\item *.ins     对应*.dtx文件的安装文件。网上下载\LaTeX宏包,一般会包含一个*.dtx文件和一个ins文件。
                使用\LaTeX处理*.ins文件可以解开*.dtx文件。
\item *.cls     定义文档外观形式的类文件,可以通过$\backslash$documentclass命令选取。
\item *.fd      字体描述文件,告诉\LaTeX有关新字体的信息。
\end{itemize}

以下文件是使用\LaTeX处理源文件时生成的:
\begin{itemize}
\item *.dvi   设备无关文件,*.tex文件编译的主要输出文件。
                可以使用DVI预览器预览,或使用dvips/dvipdf等程序输出PS或PDF文档。
\item *.toc   存储所有的章节标题,再次编译时将读取该文件生成目录。
\item *.lof   类似*.toc,用于生成图形目录。
\item *.lot   类似*.toc,用于生成表格目录。
\item *.aux   用于再次编译时传递辅助信息,如交叉引用的信息。
\item *.idx   文档中如果包含索引,\LaTeX可使用该文件存储所有的索引词条。此文件需用makeindex处理,
                详见位于\pageref{latex-tools-makeindex}页的第\ref{latex-tools-makeindex}节。
\end{itemize}

% end of Latex Introduction

\section{Basic Kownledge}

\subsection{\LaTeX{} Source}
\LaTeX的源代码包括:
\begin{itemize}
\item 需要排版的文本。
\item \LaTeX命令。
\end{itemize}

\subsubsection{特殊字符}
特殊字符包括:
\framebox{\# \$ \% \^{} \& \_ \{ \} \ {} } \par

反斜杠不能通过$\backslash$$\backslash$得到,可以使用命令\$backslash\$生成反斜杠。

显示空格符号通过命令\fbox{$\backslash$textvisiblespace}生成,如\textvisiblespace。

带圈的符号通过命令\fbox{$\backslash$textcircled\{letter\}}生成,如:
\textcircled{\scriptsize 1}
\textcircled{\scriptsize a}
\textcircled{\scriptsize c}
\textcircled{c}。

引号不能直接使用“。左引号由两个`(重音)产生,右引号由两个'(直立引号产生)。一个`和一个'产生一个单引号。

\subsubsection{空白距离}
空格和制表符等空白字符在\LaTeX中被视作相同的space,多个连续的空白被视作一个空白。
两行文本间的空白标志上下段,多个空白行等同一个空白行。

\subsubsection{\LaTeX命令}
\LaTeX命令有两种格式:
\begin{itemize}
\item 以反斜杠$\backslash$开始,命令只由字母组成。命令后的空格、数字或任何非字母的字符都标志命令的结束。
\item 由反斜杠$\backslash$和非字母的字符组成。
\end{itemize}
\LaTeX忽略命令之后的空格,因此如果需要在命令后得到空格,可以在命令后加\{\}和一个空格,
或加上一个特殊的空格命令。\{\}将阻止\LaTeX吃掉命令后的空格。
对比"\LaTeX without space"和"\LaTeX{} with space"。

命令的参数放在\{\}中,命令的可选参数放在[]中。

\subsubsection{注释}
注释放在\%后,等同C++中的//。符号\%也可以用来断开不能含有空白字符或换行符的较长输入内容。
如果注释的内容特别长,可以使用verbatim宏包提供的comment环境。comment环境中的注释不显示在文本中。

% end of Latex Source

\subsection{Environment}
\index{\LaTeX!Environment}

verbatim\index{\LaTeX!Environment!verbatim}可以将文本直接打印。
verbatim宏包可以更好的达到这个效果。\ref{latex-macro-package-verbatim}。

% end of "Latex Environment"

\subsection{Box}
\makebox[\textwidth]{central}\par
\makebox[\textwidth][s]{spread}\par
\makebox[\textwidth][r]{right}\par
\framebox[\textwidth]{这叫中规中矩}\par
\framebox[0.8\width][r]{从左边溢出了}\par
\framebox[1cm][l]{从右边溢出了}\par

命令\fbox{$\backslash$mbox\{text\}}保证把几个单词放在同一行上。
命令$\backslash$fbox和$\backslash$mbox类似,它还能围绕内容画一个框,如每个命令周边的框就是fbox的功劳。

% end of "Latex Box"

\section{Macro Package}
\index{\LaTeX!macro package}
常用的宏包:\\
\begin{tabular}{|l|l|}
\hline
宏包            & 说明 \\ \hline
makeidx         & 提供排版索引的命令。随\LaTeX提供。\\ \hline
verbatim        & 直接打印文本。\\ \hline
indentfirst     & 首行缩进。特别对中文有用。 \\ \hline
listings        & 用于在\LaTeX中引用源码。\\ \hline
xcolor          & 与listings搭配使用,源码语法高亮。 \\ \hline
beamer          & 创建演示文稿。\\\hline
\end{tabular}

\subsubsection{Verbatim宏包}\index{\LaTeX!macro package!verbatim}\label{latex-macro-package-verbatim}
verbatim环境可以直接包含打印文本,而verbatim宏包则可以直接引入ASCII文本,命令
"$\backslash$verbatiminput\{filename\}"即可

\subsection{listings}\index{\LaTeX!macro package!listings}
\begin{quote}
众里寻它千百度。
\end{quote}

listings宏包用于\LaTeX中的源代码引用,效果如下:
\begin{lstlisting}
int main(int argc, char* argv[])
{
    printf("hello, world!\n");
    return 0;
}
\end{lstlisting}

listings可以和xcolor搭配使用,让源代码语法高亮。
listings的一个示例如下:
\begin{verbatim}
\begin{lstlisting}[
    language={[ANSI]C},
    numbers=left,
    numberstyle=\tiny,
    keywordstyle=\color{blue!70},
    commentstyle=\color{red!50!green!50!blue!50},
    rulesepcolor=\color{red!20!green!20!blue!20},
    escapeinside='',
    xleftmargin=2em,
    xrightmargin=2em,
    frame=shadowbox,
    aboveskip=1em]
int main(int argc, char* argv[])
{
    /* comment */
    printf("hello, '中国'!\n");
}
\end{lstlisting}
\end{verbatim}

以上,numbers表示在源码前加以行号,可以指定行号的位置,左或右;
numberstyle表示行号风格;之后是色彩的设置;因为litings不支持中文,
所以如果在源码中包含中文,需要“转义”,用escapeinside指定转义符,
以上用'单引号作为转义符;xleftmargin和xrightmargin指定了左右边距;
aboveskip指定了上边距;frame指定了源码边框,以上为shadowbox。

可以把listings的设置放在lstset\{\}中,
这样就不用每次使用listings时都设置各项参数。

\subsubsection{语言支持和语法高亮}
\paragraph{语言支持}
listings支持常见语言,如C(ANSI, Objective, Sharp), 
    C++(ANSI, GNU, ISO, Visual), Python, \TeX(\LaTeX), 
    make, XML, SQL,
    PHP, Perl, Ruby, Lisp, Java等。

\paragraph{语法高亮}
可以自由配置高亮单词,在lstset\{\}中添加如:
\begin{lstlisting}
emph={boost},emphstyle=\color{red},
emph={[2]vector,list,iterator},emphstyle={[2]\color{blue}}
\end{lstlisting}
这样的语句即可。

\paragraph{定制边框}
可以设置代码的边框,在$\backslash$begin\{lstlisting\}[]中添加frame设置。
frame的可选值为:none, leftline, topline, bottomline, lines, single, shadowbox。
lines表示在上下划线,single表示在四周划线,shadowbox表示一个阴影框。
frame的另一个表示方式是取\{trblTRBL\}的任意子集,其中大写字母表示双划线。

\subsection{beamer}\index{\LaTeX!macro package!beamer}
Beamer是Till Tantau在2003年创建的,用于演示文稿。帮助手册参考:
\verb|texmf\doc\latex\beamer\beameruserguide.pdf|。

% end of Macro Package

\newpage
\section{Large Document}
\subsection{Create Index}
\index{\LaTeX!index}
\begin{enumerate}
\item 引入宏包:"$\backslash$usepackage\{makeidx\}"。
\item 在导言中使用命令:"$\backslash$makeindex"激活索引命令。
\item 要索引的内容通过命令:"$\backslash$index\{key\}"指定,key是索引项的关键词。
\item 当\LaTeX处理源文件时,每个$\backslash$命令都会将适当的索引项和当前页面写入一个专门的文件*.idx中,因此需要使用外部命令
"makeindex\label{latex-tools-makeindex} fname.idx"来生成索引。
\item 在源文件的适当位置(一般位于文档最后),将索引文件引入源文件,使用命令"$\backslash$printindex"。
\end{enumerate}

\newpage
\section{Chinese}
\subsection{Chinese Bookmark}
在tex文档序言部分引入:
\begin{lstlisting}
\usepackage[pdftex,CJKbookmarks]{hyperref}
\end{lstlisting}

王垠提供了一段代码,也可以达到以上效果:
\begin{lstlisting}[language=TEX]
\usepackage[pdftex]{hyperref}
\pdfstringdefDisableCommands{
\let\CJK@XX\relax
\let\CJK@XXX\relax
\let\CJK@XXXp\relax
\let\CJK@XXXX\relax
\let\CJK@XXXXp\relax
}
\end{lstlisting}

生成中文书签的步骤:
\begin{lstlisting}
pdflatex cs.notes.tex
gbk2uni cs.notes.out
pdflatex cs.notes.tex
\end{lstlisting}

CTEX中带了gbk2uni这个小工具。

% end of Latex

% start Metapost
\chapter{Metapost}
\noindent\href{http://ect.bell-labs.com/who/hobby/index.shtml}{John D.Hobby} Metapost作者的主页。\\
\noindent\href{http://docs.huihoo.com/homepage/shredderyin/metapost.html}{王垠写的Metapost介绍。} 其中有大量的Metapost实例。\\

Metapost是一种用于绘图的语言,源自Knuth的Metafont语言。
在CTEX提供的软件包中附带有Metapost工具:mp/mpost。较新的系统都使用mpost命令。
Metapost文件通常以.mp为后缀,编译命令为\fbox{mpost fname.mp}
生成的图形文件为\fbox{fname.1}或以某个数字为后缀。这种图形文件就是eps文件。
用CTEX软件包中的GSview就可以查看esp, ps图形文件,也可以用GSview将它们转换为其他许多格式的图像文件,如jpeg等。

一个简单的Metapost源文件是:
\begin{lstlisting}
beginfig(1)
draw (20,20)--(0,0);
endfig;
\end{lstlisting}
就画了一条直线。Metapost的源文件,每行以``;''结尾;注释与\LaTeX注释相同,使用\%作注释。

% end of "Metapost"

\chapter{DocBook}
DocBook的特殊字符和转义方法
\begin{tabular}{|l|l|}
\hline
< & <lt;    \\ \hline
> & <gt;    \\ \hline
\end{tabular}

DocBook常用inline元素

% end of "DocBook"

\chapter{Graphic Tools}
\index{Graphic Tools}

\section{Grapviz-Dot}
\index{Graphic Tools!Dot}
\noindent\href{http://www.graphviz.org/}{Graphviz} Graph Visualization Software\\

Dot是Graphviz套件中的一个工具,用于绘图。Dot擅长绘制如数据结构、继承关系之类的复杂图形。
Doxygen就是用Dot绘制类的继承关系,函数间的调用与被调用关系等。
Dot的命令非常简单:\fbox{dot -Tpng graph.dot -o graph.png}。
-T指明了输出图形的类型,Dot可以输出的图形类型包括:gif, png, svg, ps等。

定义图的代码是:
\begin{verbatim}
digraph G{
    main -> func;   /* comment */ // also comment
}
\end{verbatim}
每行代码都需要以``;''结束;用C/C++注释符号作注释,即/**/或//作注释。

在图中定义子图的代码是:
\begin{verbatim}
digraph G{
    subgraph cluster0{
        main;
    }

    subgraph cluster1{
        func;
    }

    main->func;
}
\end{verbatim}
子图必须以subgraph为关键字,且子图的名词必须以cluster打头。
不能将箭头指向子图,只能指向子图中的元素节点。

% end of "Graphic Tools"

\chapter{Protocol Notes}

\section{BT}
\noindent\href{http://en.wikipedia.org/wiki/Comparison\_of\_BitTorrent\_software}{wiki上的BT软件列表} 这个列表
比较全面的列出了各个平台下的BT软件,并做了非常详细的对比,如支持的BT特性,使用的编程语言,以及BT库等。
\noindent\href{http://en.wikipedia.org/wiki/Libtorrent\_(Rasterbar)}{wiki上的libtorrent(Rasterbar)介绍} 这个BT库
又叫rb-libtorrent,用了Boost.Asio,以此获得跨平台性。

\subsection{libtorrent Notes}
libtorrents是一个用C++写的BT库,其中用了Boost,获取跨平台性。
Arctic Torrent就使用了该库,而Arctic Torrent的目标是使用尽量少的内存和CPU。

libtorrents中,一个piece拥有多个blocks。

% end of BT

% end of Protocol Notes

\clearpage      % 如果是双面打印,即openrigth,则使用\cleardoublepage
\addcontentsline{toc}{chapter}{Bibliography}    % 将参考文献加入到目录中

\begin{thebibliography}{99}
\bibitem{tcpl} B.Stroustrup: The C++ Programming Lanauage(2002)
\end{thebibliography}

\clearpage      % 如果是双面打印,即openrigth,则使用\cleardoublepage
\addcontentsline{toc}{chapter}{Index}    % 将索引加入到目录中

\printindex

\end{CJK*}
\end{document}
