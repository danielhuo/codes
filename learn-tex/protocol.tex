\part{Protocol Notes}
\noindent\href{http://linux-ip.net/html/}{Guide to IP Layer Network Administration with Linux}介绍了许多Linux下的网络知识,
包括常用的网络软件等。

POP3(fecting mail), SMTP(sending mail), IRC(chat), LDAP(directory access).

\section{NAT}
NAT类型:
\begin{enumerate}
\item \blscmd{Full cone NAT},内部地址(iAddr:port1)映射为外部地址(eAddr:port2)。
任意从(iAddr:port1)发出的报文都将转为(eAddr:port2),
任意从(eAddr:port2)接收的报文都将送到(iAddr:port1)。
\item \blscmd{(Address) Restricted cone NAT},只有当(iAddr:port1)发送报文给(hostAddr:any),
(hostAddr:any)才可发送报文给(iAddr:port1)。此种NAT是地址限制,但port不限制之NAT。
\item \blscmd{Port-Restricted cone NAT},(iAddr:port1)通过(eAddr:port2)发送报文给(hostAddr:port3),
只有(hostAddr:port3)才可发送报文给(iAddr:port1)。
\item \blscmd{Symmetric NAT},每个从内网地址发出的报文到达不同的外网地址,都将映射出一道唯一的通往外网的通路。
\end{enumerate}
以上后三种类型的NAT,都一定要(iAddr:port1)首先发送报文,才可做NAT。

\section{BT}
\noindent\href{http://en.wikipedia.org/wiki/Comparison\_of\_BitTorrent\_software}{wiki上的BT软件列表} 这个列表
比较全面的列出了各个平台下的BT软件,并做了非常详细的对比,如支持的BT特性,使用的编程语言,以及BT库等。
\noindent\href{http://en.wikipedia.org/wiki/Libtorrent\_(Rasterbar)}{wiki上的libtorrent(Rasterbar)介绍} 这个BT库
又叫rb-libtorrent,用了Boost.Asio,以此获得跨平台性。

\subsection{libtorrent Notes}
libtorrents是一个用C++写的BT库,其中用了Boost,获取跨平台性。
Arctic Torrent就使用了该库,而Arctic Torrent的目标是使用尽量少的内存和CPU。

libtorrents中,一个piece拥有多个blocks。
