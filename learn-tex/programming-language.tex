\part{Programming Language}

本部分主要一些小型语言的记录。如shell,awk, sed,metapost, dot等。
常使用的通用编程语言,如C/C++,Python单独成章。
一些具体应用领域的语言,如GCC汇编,Vim脚本等,分别归属于这些工具的笔记章节。
不过,这里仍是所有记录的一个入口,以下给出所有语言的索引。

C/C++的笔记见第\ref{language-c-cpp}部分。Python的笔记见第\ref{language-python}章。
GCC汇编的笔记见第\ref{language-gcc-assemble}节。Vim脚本的笔记见第\ref{language-vim-script}节。

对于编程语言,在学习过程中所写的代码,大都以外部文件的形式给出,并不附加于本文档。

%%%%%%%%%%%%%%%%%%%%%%%%%%%%%%%%%%%%%%%%%%%%%%%%%%%%%%%%%%%%%%%%%%%%%%%%%%%%%%%%
% bash
%%%%%%%%%%%%%%%%%%%%%%%%%%%%%%%%%%%%%%%%%%%%%%%%%%%%%%%%%%%%%%%%%%%%%%%%%%%%%%%%
\chapter{bash}

特殊变量\\
\begin{tabular}{ll}
\$*     &   展开为参数位置(成一个list)。\$@和\$*功能一样。\\
\$0     &   展开为shell或shell脚本名。\\
\$\$    &   展开为shell的PID。\\
\$?     &   展开为最近执行的程序的退出状态。\\
\$\#    &   展开为参数个数。\\
\end{tabular}

%%%%%%%%%%%%%%%%%%%%%%%%%%%%%%%%%%%%%%%%%%%%%%%%%%%%%%%%%%%%%%%%%%%%%%%%%%%%%%%%
% awk
%%%%%%%%%%%%%%%%%%%%%%%%%%%%%%%%%%%%%%%%%%%%%%%%%%%%%%%%%%%%%%%%%%%%%%%%%%%%%%%%
\chapter{awk}
awk有与C/C++类似的命令行参数处理,分别是ARGC, ARGV。

\blscmd{NF}表示field的个数。
\blscmd{FS}表示field的分隔符。

\blscmd{getline}是一个函数。但与普通函数不同的是,getline的语法类似一个语句,不能写成
getline()。getline从输入中读入一行,可以从文件、管道中读取。它的一个常见用法是:
\bllcmd{while( (getline $<$ "fname") $>$ 0 ) \{\} }
以上$<$表示重定向,而$>$则表示逻辑判断,与0的比较;其实是判断getline的返回值。

%%%%%%%%%%%%%%%%%%%%%%%%%%%%%%%%%%%%%%%%%%%%%%%%%%%%%%%%%%%%%%%%%%%%%%%%%%%%%%%%
% sed 
%%%%%%%%%%%%%%%%%%%%%%%%%%%%%%%%%%%%%%%%%%%%%%%%%%%%%%%%%%%%%%%%%%%%%%%%%%%%%%%%
\chapter{sed}
